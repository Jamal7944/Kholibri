%%%%%%%%%%%%%%%%%%%%%%%%%%%%%%%%%%%   Packages de base   %%%%%%%%%%%%%%%%%%%%%%%%%%%%%%%%%%%


% Classe du document (book : chapitre inclus, pratique pour les gros cours (PASS, Tremplin), article (8pt) ou amsart (12pt) : pratique pour les cours sur une seule thématique)
\documentclass[a4paper, 11pt,openany]{book}% openany évite de commencer les chapitres sur une page forcément impaire (évite les pages blanches)

% Encodage, langue et font
\usepackage[utf8]{inputenc} % Pour les accents (conversion entre ces caractères accentués et les commandes d’accentuation)
\usepackage[french]{babel} % Français
\usepackage[T1]{fontenc} % Permet d'afficher et de prendre correctement en charge ces caractères accentués du point de vue du fichier de sortie

% Si on veut faire apparaître le texte avec des caractères plus larges
%\usepackage[lf]{Baskervaldx} % Texte en plus "gras"
%\usepackage[bigdelims,vvarbb]{newtxmath} % Lettre mathématiques en plus "gras"
%\usepackage[cal=boondoxo]{mathalfa} % Style pour les \mathscal
%\renewcommand*\oldstylenums[1]{\textosf{#1}}

% Mise en page
% Version automatique mais problème d'écart entre l'en-tête et le texte.
%\usepackage{fullpage} % Numérotation bas de page (et mise en page pleine).
\usepackage{setspace} % Interligne
\onehalfspacing % Interligne
\usepackage[a4paper]{geometry}% Package pour mise en page
\geometry{hscale=0.8,vscale=0.8,centering,headsep=0.3cm} % Marges, Headsep permet de ne pas coller le texte à l'en-tête

\setlength{\parindent}{0pt} % Supprimer les identations par défaut

\usepackage{fancyhdr} %Package permettant de mettre des en-têtes et des pieds de pages
\pagestyle{fancy}
\fancyhead{} % Enleve ce qui est présent par défaut
\fancyfoot{}% Enleve ce qui est présent par défaut
\usepackage{extramarks} % Pour écrire proprement le chapitre dans l'en-tête
%\renewcommand{\chaptermark}[1]{\markboth{\chaptername\ \thechapter.\ #1}{}} % Redéfinition du chapitre pour écriture dans l'en-tête sous la forme "Chapitre n. Nom du chapitre"
%\renewcommand{\chaptermark}[1]{\markboth{\thechapter.\ #1}{}} % Forme "n. Nom du chapitre"
\renewcommand{\chaptermark}[1]{\markboth{#1}{}} % Forme "Nom du chapitre"
\renewcommand{\headrulewidth}{0.6pt} % Epaisseur trait en haut
\renewcommand{\footrulewidth}{0.6pt} % Epaisseur trait en bas
\setlength{\headheight}{15pt}
% Placer les informations où on veut (à noter : E: Even page ; O: Odd page ; L: Left field ; C: Center field ; R: Right field ; H: Header ; F: Footer) :
\fancyhead[LO,LE]{\slshape  \textbf{Mathématiques P.A.S.S. 1}}
\fancyhead[RO,RE]{\slshape \textbf{\firstleftmark}}
%\fancyhead[CO,CE]{---Draft---}
\fancyfoot[C]{\thepage}
\fancyfoot[LO, LE] {\slshape Damien GOBIN}
\fancyfoot[RO, RE] {\slshape Année 2021-2022}

% Couleurs
\usepackage{xcolor}
\definecolor{BleuTresFonce}{rgb}{0.0,0.0,0.250}
%\definecolor{bleu}{rgb}{0.36, 0.54, 0.66}
\definecolor{bleu}{rgb}{0.47, 0.62, 0.8}
%\definecolor{vert}{rgb}{0.33, 0.42, 0.18}
%\definecolor{vert}{rgb}{0.52, 0.73, 0.4}
\definecolor{vert}{rgb}{0.56, 0.74, 0.56}
\definecolor{rouge}{rgb}{0.700, 0, 0}

% Liens hypertextes
\usepackage[colorlinks,final,hyperindex]{hyperref}
\hypersetup{
    pdftex,
    linkcolor=BleuTresFonce,
    citecolor=BleuTresFonce,
    filecolor=BleuTresFonce,
    urlcolor=BleuTresFonce,
    pdftitle=TemplateCours,
    pdfauthor=Damien Gobin,
    pdfsubject=,
    pdfkeywords=
}


% Inclure des figures
\usepackage{graphicx} % Permet d'utiliser includegraphics
\usepackage{pdfpages} % Permet d'insérer des fichiers pdf


%%%%%%%%%%%%%%%%%%%%%%%%%%%%%%%%%%%   Packages mathématiques  %%%%%%%%%%%%%%%%%%%%%%%%%%%%%%%%%%%

% Packages mathématiques de base
%\usepackage{amsfonts} % Permet de taper des ensembles 
\usepackage{amsmath} % Permet de taper des maths (contient cleveref)
\usepackage{amsthm} % Permet de définir une environnement pour les théorèmes
\usepackage{amssymb} % Donne accès a plus de symboles mathématiques (charge de façon automatique amsfonts)
% \usepackage{mathrsfs} % Ecriture ronde type mathcal
%\usepackage{bbold} % Permet d'avoir l'indicatrice mais change les textbb
\usepackage{stmaryrd} % Pour les intervalles entiers [[ ]]
\usepackage{pifont} % Pour avoir accès à plus de caractères

%Utiliser les symboles jeu de cartes
\DeclareSymbolFont{extraup}{U}{zavm}{m}{n}
\DeclareMathSymbol{\varheart}{\mathalpha}{extraup}{86}
\DeclareMathSymbol{\vardiamond}{\mathalpha}{extraup}{87}



% Référence dans le document
\usepackage[noabbrev,capitalize]{cleveref}

% Tableau
\usepackage{tabularx} % Tableau
\usepackage{multirow} % Pour créer des tableaux en subdivisant les lignes
\usepackage{diagbox} %Pour créer des crois dans les tableaux à double entrées
\usepackage{enumitem} %Permet de scinder une énumération et de reprendre au numéro suivant
\usepackage{multicol} % Création de document en colonne avec plusieurs colonnes
\multicolsep=5pt % supprime l'espace vertical

% Modification des itemizes
\setitemize{label=$\bullet$} % Utilise le package enumitem et met des points au lieu des tirets

% Ecrire des algorithmes en "français" (exemple issu du cours d'optimisation 2020/2021)
\usepackage[linesnumbered, french, frenchkw,ruled]{algorithm2e}
% Exemple :
%\begin{center}
%\begin{algorithm}
%\Entree{Un graphe $G = (V,E)$, $|V| = n$, $|E| = m$ et pour chaque arête $e$ de $E$ son poids $c(e)$;}
%\Sortie{Un arbre (ou une forêt) maximal $A = (V,F)$ et de poids minimum ;} 
%Trier et renuméroter les arêtes de $G$ dans l'ordre croissant de leur poids : $c(e_1) \leqslant c(e_2) \leqslant ... \leqslant c(e_m)$ \;
%Poser $F := \emptyset$ et $k = 0$\;
%\Tq{$k<m$ et $|F| < n-1$}{Si $e_{k+1}$ ne forme pas de cycle avec $F$ alors $F := F \cup \{ e_k \}$\;
%$k := k +1 $\;}
%\caption{Algorithme de Kruskal théorique (1956)}
%\end{algorithm}
%\end{center}

% Inclure du code Python avec coloration syntaxique et bloc gris (exemple issu du TP2 d'optimisation 2020/2021)
%\usepackage{xcolor} % Déjà importé plus haut
\usepackage{listings}
\lstset{backgroundcolor=\color{darkWhite},literate={á}{{\'a}}1 {é}{{\'e}}1 {í}{{\'i}}1 {ó}{{\'o}}1 {ú}{{\'u}}1{Á}{{\'A}}1 {É}{{\'E}}1 {Í}{{\'I}}1 {Ó}{{\'O}}1 {Ú}{{\'U}}1{à}{{\`a}}1 {è}{{\`e}}1 {ì}{{\`i}}1 {ò}{{\`o}}1 {ù}{{\`u}}1{À}{{\`A}}1 {È}{{\'E}}1 {Ì}{{\`I}}1 {Ò}{{\`O}}1 {Ù}{{\`U}}1{ä}{{\"a}}1 {ë}{{\"e}}1 {ï}{{\"i}}1 {ö}{{\"o}}1 {ü}{{\"u}}1{Ä}{{\"A}}1 {Ë}{{\"E}}1 {Ï}{{\"I}}1 {Ö}{{\"O}}1 {Ü}{{\"U}}1{â}{{\^a}}1 {ê}{{\^e}}1 {î}{{\^i}}1 {ô}{{\^o}}1 {û}{{\^u}}1{Â}{{\^A}}1 {Ê}{{\^E}}1 {Î}{{\^I}}1 {Ô}{{\^O}}1 {Û}{{\^U}}1{œ}{{\oe}}1 {Œ}{{\OE}}1 {æ}{{\ae}}1 {Æ}{{\AE}}1 {ß}{{\ss}}1{ű}{{\H{u}}}1 {Ű}{{\H{U}}}1 {ő}{{\H{o}}}1 {Ő}{{\H{O}}}1{ç}{{\c c}}1 {Ç}{{\c C}}1 {ø}{{\o}}1 {å}{{\r a}}1 {Å}{{\r A}}1{€}{{\EUR}}1 {£}{{\pounds}}1}
\lstdefinestyle{stylepython}{        language=Python,        basicstyle=\ttfamily,    commentstyle=\color{green},    keywordstyle=\color{blue},    stringstyle=\color{olive},    numberstyle=\tiny,        numbers=left,        stepnumber=1,         numbersep=5pt}
% Exemple :
%\begin{lstlisting}[style=stylepython]
%import networkx as nx #Cette bibliothèque permettra de manipuler des graphes
%import matplotlib.pyplot as plt #Cette bibliothèque permettra de les représenter
%import numpy as np
%\end{lstlisting}

%Package permettant de tracer des graphes, des courbes, etc.. (exemple issu du cours d'optimisation 2020/2021)
\usepackage{tikz}
\usepackage{tikz-cd}
\usepackage{tkz-tab} % Pour les tableaux de signes
%\usetikzlibrary{shapes,backgrounds}
\usetikzlibrary{
    decorations.pathmorphing,
    arrows,
    arrows.meta,
    calc,
    shapes,
    shapes.geometric,
    decorations.pathreplacing,}
\tikzcdset{arrow style=tikz, diagrams={>=stealth}}
\tikzset{>=stealth'}
\tikzstyle{rond}=[draw,circle,thick,fill=white]
% Exemple courbe :
%\begin{center}
%\begin{tikzpicture}
%   \def\shift{.5}
%   \def\xmax{6}
%   \def\ymax{7}
%   \draw[->] (-\shift,0) -- (\xmax+\shift,0) node[below] {$x$}; 
%   \foreach \x in {1,...,\xmax}{ \pgfmathtruncatemacro\xbis{100*\x}
%       \draw (\x,0) -- (\x,-.3*\shift) node[below] {$\scriptstyle{\xbis}$};
%   }
%   % axe ordonnees
%   \draw[->] (0,-\shift) -- (0,\ymax+\shift) node[left] {$y$};  
%   \foreach \y in {1,...,\ymax}{ \pgfmathtruncatemacro\ybis{100*\y}
%       \draw (0,\y) -- (-.3*\shift,\y) node[left] {$\scriptstyle{\ybis}$};
%   }
%   \draw[fill=yellow] (0,0) -- (3,0) -- (0,6) -- cycle;
%   
%   \draw[thick, purple] plot [domain=-.15:6.15, variable=\t] (\t,6-\t);
%   \node[above, text=purple] (A) at (6.9,.1) {$x+y = 600$};
%   %%
%   \draw[thick, purple] plot [domain=-.1:3.1, variable=\t] (\t,6-2*\t);
%   \node[right, text=purple] (B) at (1.1,5) {$2x+y = 600$};
%   %%
%   \draw[thick, blue] plot [domain=-.1:3.85, variable=\t] (\t,6-1.6*\t);
%   %%
%   \draw[fill=blue] (0,6) circle (3pt);
%   \node[right, text=blue] (B) at (0.1,6.3) {Solution optimale};
%\end{tikzpicture}
%\end{center}
% Exemple graphe :
%\begin{center}
%\begin{tikzpicture}[scale=0.9]
%    \draw[thick] (-1.5,0) node[rond] {$1$} coordinate (A) --
%        ++(3,0) node[rond] {$2$} coordinate (B) --
%        ++(-1,-1) node[rond] {$4$}  coordinate (D) --
%        ++(-1,0) node[rond] {$3$}  coordinate (C) --
%        ++(0,-1) node[rond] {$5$}  coordinate (E) --
%        ++(1,0) node[rond] {$6$}  coordinate (F) --
%        ++(1,-1) node[rond] {$8$}  coordinate (H) --
%        ++(-3,0) node[rond] {$7$}  coordinate (G) -- (A)
%        (A) -- (C)
%        (G) -- (E)
%        (D) -- (F)
%        (B) -- (H);
%\end{tikzpicture}
%\end{center}

%%%%%%%%%%%%%%%%%%%%%%%%%%%%%%%%%%%   Environnements  %%%%%%%%%%%%%%%%%%%%%%%%%%%%%%%%%%%

%Différents types à donner aux environnements théorèmes, définitions, etc...
%[theorem] permet de numéroter par rapport à la section en cours
%\renewcommand{\theexem}{\empty{}} permet de ne pas numéroterNe numérote pas les exemples

%\usepackage{thmtools} % Permet de créer un environnement de théorème avec des cadres
%\declaretheorem[thmbox=L]{boxtheorem L}
%\declaretheorem[thmbox=M]{boxtheorem M}
%\declaretheorem[thmbox=S]{boxtheorem S}
%
%%\usepackage[dvipsnames]{xcolor}
%%\declaretheorem[shaded={bgcolor=Lavender,textwidth=12em}]{BoxI}
%\declaretheorem[shaded={rulecolor=black,rulewidth=1pt}]{BoxII}
%
%
%\usepackage[leftmargin=0.1865cm,rightmargin=0.6cm]{thmbox}
% Voici ici pour les options https://ctan.mines-albi.fr/macros/latex/contrib/thmbox/thmbox.pdf

%% Ecrit de cette façon tout le monde a le même style : Titre en droit et texte en italique très léger.
%\newtheorem[L,leftmargin=0.1865cm,rightmargin=0.1865cm,cut=true]{thm}{Théorème}[subsection]
%%\renewcommand{\thetheorem}{\empty{}} 
%\newtheorem[M,leftmargin=0.1865cm,rightmargin=0.1865cm,cut=true]{propo}[thm]{Proposition}
%%\renewcommand{\thepropo}{\empty{}} 
%\newtheorem[M,leftmargin=0.1865cm,rightmargin=0.1865cm,cut=true]{prop}[thm]{Propriété}
%%\renewcommand{\theprop}{\empty{}} 
%\newtheorem[M,leftmargin=0.1865cm,rightmargin=0.1865cm,cut=true]{coro}[thm]{Corollaire}
%\newtheorem[M,leftmargin=0.1865cm,rightmargin=0.1865cm,cut=true]{lem}[thm]{Lemme}
%\newtheorem[M,leftmargin=0.1865cm,rightmargin=0.1865cm,cut=true]{defi}[theorem]{Définition}
%%\renewcommand{\thedefinition}{\empty{}}
%\newtheorem[S,leftmargin=0.1865cm,rightmargin=0.1865cm,cut=true]{exem}{Exemple}
%\renewcommand{\theexem}{\empty{}} 
%\newtheorem[S,leftmargin=0.1865cm,rightmargin=0.1865cm,cut=true]{exo}{Exercice}
%\renewcommand{\theexo}{\empty{}}
%\newtheorem[S,leftmargin=0.1865cm,rightmargin=0.1865cm,cut=true]{sol}{Solution}
%\renewcommand{\thesol}{\empty{}} 
%\newtheorem[S,leftmargin=0.1865cm,rightmargin=0.1865cm,cut=true]{hypo}{Hypothesis}[section]
%\newtheorem[S,leftmargin=0.1865cm,rightmargin=0.1865cm,cut=true]{remark}[thm]{Remarque}%[section]
%\newtheorem{dem}{Démonstration}
%\renewcommand{\thedem}{\empty{}} 
%\newtheorem[S]{nota}{Notation}[section]

\usepackage[leftmargin=0.175cm,rightmargin=0.175cm]{thmbox}

% Ecrit de cette façon tout le monde a le même style : Titre en droit et texte en italique très léger.
% Théorème : cadre presque complet
\newtheorem[L]{thm}{Théorème}[section]

% Cadre en L
\newtheorem[M]{propo}[thm]{Proposition}
\newtheorem[M]{prop}[thm]{Propriété}
\newtheorem[M]{coro}[thm]{Corollaire}
\newtheorem[M]{lem}[thm]{Lemme}
\newtheorem[M,bodystyle=]{defi}[thm]{Définition}
\newtheorem[M,bodystyle=]{remark}[thm]{Remarque}
\newtheorem[M,bodystyle=]{met}[thm]{Méthode}
\newtheorem[M,bodystyle=]{ret}[thm]{A retenir}
\newtheorem[M,bodystyle=]{idee}[thm]{Idée}

% Barre à gauche
\newtheorem[style=S,underline=false,bodystyle=]{exem}[thm]{Exemple}
\newtheorem[S,underline=false,bodystyle=]{exo}[thm]{Exercice}
\newtheorem[S,underline=false,bodystyle=]{appli}[thm]{Application}
\newtheorem[S,underline=false,bodystyle=]{sol}[thm]{Solution}
\newtheorem[S,underline=false,bodystyle=]{hypo}[thm]{Hypothesis}
\newtheorem[S,underline=false,bodystyle=]{nota}[thm]{Notation}
%\newtheorem[S]{dem}[thm]{Preuve} % Si on veut mettre un trait sur le côté au lieu du carré à la fin
    

% Utiliser des commentaires
\usepackage{comment}
% \excludecomment{sol} %commenter cette ligne si on veut faire apparaitre les solutions d'exercices


%%%%%%%%%%%%%%%%%%%%%%%%%%%%%%%%%%%   Raccourcis   %%%%%%%%%%%%%%%%%%%%%%%%%%%%%%%%%%%

\newtheorem{cours}{Question de cours} 
\renewcommand{\thecours}{\empty{}}  
 
% Ensembles
\newcommand{\R}{\mathbb{R}} 
\newcommand{\Q}{\mathbb{Q}}
\newcommand{\C}{\mathbb{C}}
\newcommand{\Z}{\mathbb{Z}}
\newcommand{\K}{\mathbb{K}}
\newcommand{\N}{\mathbb{N}}
\newcommand{\D}{\mathbb{D}}


% Lettres calligraphiques
\newcommand{\calP}{\mathcal{P}} 
\newcommand{\calF}{\mathcal{F}} 
\newcommand{\calD}{\mathcal{D}} 
\newcommand{\calQ}{\mathcal{Q}} 
\newcommand{\calT}{\mathcal{T}} 
\newcommand{\calC}{\mathcal{C}} 

% Noyau et image
\newcommand{\im}{\text{Im}} 
\newcommand{\re}{\text{Re}} 
\newcommand{\noy}{\text{Ker}} 
\newcommand{\card}{\text{Card}} 

%pgcd et ppcm
\newcommand{\PGCD}{\text{PGCD}}
\newcommand{\PPCM}{\text{PPCM}}


% Fonctions usuelles
%\newcommand{\un}{\mathbb{1}} % Indicatrice en utilisant le package \usepackage{bbold} mais celui modifie les mathbb donc on définit l'indicatrice à la main
\def\un{{\mathchoice {\rm 1\mskip-4mu l} {\rm 1\mskip-4mu l}
{\rm 1\mskip-4.5mu l} {\rm 1\mskip-5mu l}}} % Indicatrice

\newcommand{\fonction}[5]{\begin{array}[t]{cccc}
#1\colon & #2 & \longrightarrow & #3 \\
    & #4 & \longmapsto & #5 \end{array}}%Commande fonction


%Présentation
\newcommand{\ds}{\displaystyle}





\title{Mathématiques pour P.A.S.S. 1}
\author{Damien Gobin}

\begin{document}

\begin{titlepage}
  \begin{sffamily}
  \begin{center}

    % Upper part of the page. The '~' is needed because \\
    % only works if a paragraph has started.
%    \includegraphics[scale=0.04]{logoUN.png}~\\[1.5cm]
 
    \hspace{5.7cm}

    [5cm]

%    \textsc{\LARGE Université de Nantes}\\[2cm]

    % Title
    \rule{\linewidth}{0.5mm}\\[0.4cm]
    { \huge \bfseries Mathématiques pour le P.A.S.S 1\\[0.4cm] }

    \rule{\linewidth}{0.5mm}\\[4cm]
    

    \textsc{\Large Filière : P.A.S.S.}\\
    \textsc{\Large Année : L1.}\\    [5cm]
    
   
    % A gauche sous le logo
    \begin{minipage}{0.4\textwidth}
      \begin{flushleft} \large
        \textsc{Damien GOBIN}\\
        Mail : damien.gobin@univ-nantes.fr\\
      \end{flushleft}
    \end{minipage}
    \hspace{1cm}
    % A droite sous le logo
    \begin{minipage}{0.5\textwidth}
      \begin{flushright} \large
         Laboratoire de Mathématiques Jean Leray \\
        Université de Nantes
      \end{flushright}
    \end{minipage}
    
    
%    \vfill
%    \includegraphics[scale=0.5]{logoUN.png}
%    \hspace{3cm}
%    \includegraphics[scale=0.33]{LMJL.jpg}
%    \\[2cm]

%    % A gauche sous le logo
%    \begin{minipage}{0.4\textwidth}
%      \begin{flushleft} \large
%        Damien \textsc{GOBIN}\\
%        mail : damien.gobin@univ-nantes.fr\\
%      \end{flushleft}
%    \end{minipage}
%    % A droite sous le logo
%    \begin{minipage}{0.4\textwidth}
%      \begin{flushright} \large
%         Laboratoire de Mathématiques Jean Leray \\
%        Université de Nantes
%      \end{flushright}
%    \end{minipage}

    \vfill

    % Bas de page
    {\large Année académique 2021-2022}

  \end{center}
  \end{sffamily}
\end{titlepage}



\begin{exo}

%Difficulté : 3/5
Soient $(\alpha, \beta, n) \in \mathbb{R}^2 \times \mathbb{N}$. Calculer 
\[ \int_{\alpha}^{\beta} (t-\alpha)^n (t - \beta)^n \, dt.\]

\end{exo}


\begin{sol}

%On pose, pour $(\alpha,\beta,n,m) \in \mathbb{R}^2 \times \mathbb{N}^2$, 
%\[ I_{m,n}= \int_{alpha}^{\beta} (t-\alpha)^m(t-\beta)^n \,dt.\]
%On intègre par parties pour obtenir une relation entre $I_{m,n}$ et $I_{m-1,n+1}$, et on trouve 
%\[ I_{m,n}= \left[ (t-\alpha)^m \frac{(t-\beta)^{n+1}}{n+1} \right]_{\alpha}^{\beta} - \frac{m}{n+1} \int_{alpha}^{\beta} (t-\alpha)^{m-1} (t-\beta)^{n+1} \,dt = -\frac{m}{n+1} I_{m-1,n+1}.\]
%D'autre part, pour tout $p \in \mathbb{N}$, on a \[ I_{0,p} = \int_{\alpha}^{\beta} (t-\beta)^p \, dt= -\frac{(\alpha- \beta)^{p+1}}{p+1}.\]
%Une récurrence immédiate donne alors \[ I_{m,n} = (-1)^{m+1} \frac{m(m-1)...1}{(n+1)(n+2)...(n+m)} \times \frac{(\alpha - \beta)^{m+n+1}}{m+n+1}.\]
%En particulier, l'intégrale recherché vaut $I_{n,n}$, c'est-à-dire 
%\[I_{n,n}=(-1)^{n+1}\frac{n!}{(n+1)...(2n)}\frac{(\alpha-\beta)^{2n+1}}{2n+1} = (-1)^{n+1}\frac{(\alpha-\beta)^{2n+1}}{(2n+1) \begin{pmatrix}
%2n \\ n
%\end{pmatrix}}.\]
%
\end{sol}

\begin{exo}

%Difficulté : 4/5
%Chapitre : Fonctions usuelles
Soit $p \geqslant 2$ un entier et $0<a_1< \dots <a_p$ des nombres réels positifs.
\begin{enumerate}
\item Montrer que, pour tout $a>a_p$, l'équation 
\[ a_1^x+ \dots +a_p^x =a^x\]
 admet une unique racine $x_a$.
 \item Étudier le sens de variation de $a \mapsto x_a$.
 \item Déterminer l'existence et calculer 
 \[ \lim_{a\to + \infty} x_a \quad \text{et} \quad \lim_{ a \to + \infty} x_a \ln(a). \]
\end{enumerate}

\end{exo}


\begin{sol}

\begin{enumerate}
\item On introduit la fonction 
\[ f_a(x)= \left( \frac{a_1}{a} \right)^x + \dots + \left( \frac{a_1}{a} \right)^x = \sum_{k=1}^p e^{x \ln \left( \frac{a_k}{a} \right)}.\]
Puisque $\ln \left( \frac{a_k}{a} \right)<0$, $x \mapsto x\ln \left( \frac{a_k}{a} \right)$ est strictement décroissante, et donc $f_a$ est strictement décroissante. Or, $f_a(0)=p$ et \[ \lim_{x\to + \infty} f_a(x) = 0.\]
L'équation $f_a(x)=1$ admet donc une unique racine $x_a>0$.
 \item Soit $a<b$. En reprenant la notation de la question précédente, pour tout $x>0$, on a $f_a(x) \geqslant f_b(x)$. En particulier $f_b(x_b)=f_a(x_a)=1 \geqslant f_b(x_a)$. Par décroissance de $f_b$, on en déduit que $x_a \geqslant x_b$ et donc $a \mapsto x_a$ est décroissante.
 \item Puisque $a\mapsto x_a$ est décroissante et minorée par $0$, elle admet une limite $\ell \geqslant 0$ en $+ \infty$. Supposons $\ell >0$. Alors, en passant à la limite dans 
 \[ a_1^{x_a} + \dots + a_p^{x_a}=a^{x_a},\]
 on trouve 
  \[ a_1^{\ell} + \dots + a_p^{\ell}=+ \infty,\]
une contradiction. Donc $\ell=0$. Ainsi, il vient également
\[ x_a \ln(a) = \ln \left(a_1^{x_a} + \dots + a_p^{x_a} \right),\]
ce qui prouve que $x_a \ln(a)$ tend vers $\ln(p)$. 
\end{enumerate}

\end{sol}

\begin{exo}

%Difficulté : 4/5
%Chapitre : Nombres complexes
Soit $n \in \mathbb{N}^{\star}$. On note $\mathbb{U}_n$ l'ensemble des racines $n$-ièmes de l'unité. Calculer 
\[ \sum_{z \in \mathbb{U}_n} |z-1|.\]

\end{exo}


\begin{sol}

Soit $k \in \{0,...,n-1\}$ et soit $\omega_k =e^{ \frac{2ik\pi}{n}}$. Alors 
\[ |\omega_k-1|=|e^{\frac{2ik\pi}{n}} - e^{i0}|=2 \left| \sin \left( \frac{k \pi}{n} \right) \right| \]
en factorisant par l'angle moitié. De plus, pour $k \in \{0,...,n-1\}$, $\frac{k \pi}{n} \in [0,\pi]$ et le sinus est positif. On en déduit
\begin{align*}
\sum_{z \in \mathbb{U}_n} |z-1| & =    2 \sum_{z \in \mathbb{U}_n} \sin \left( \frac{k \pi}{n} \right) \\ 
 &= 2 \text{Im} \left(  \sum_{z \in \mathbb{U}_n} e^{\frac{ik \pi}{n}}  \right)\\ 
 &= 2 \text{Im} \left( 1 \times \frac{1 - e^{i \pi}}{1 - e^{\frac{ i\pi}{n}}} \right)\\
  &= 4 \text{Im} \left( \frac{1}{1 - e^{\frac{ i\pi}{n}}}\right)\\ 
 &= 4 \text{Im} \left( \frac{1}{-2i \sin \left( \frac{\pi}{2n} \right) e^{\frac{ i\pi}{2n}}}\right)\\ 
  &= 2 \text{Im} \left( \frac{ie^{\frac{ -i\pi}{2n}}}{\sin \left( \frac{\pi}{2n} \right)}\right)\\ 
    &= 2 \frac{\cos \left( \frac{\pi}{2n} \right)}{\sin \left( \frac{\pi}{2n} \right)}.
\end{align*}

\end{sol}

\begin{exo}

%Difficulté : 3/5
%Chapitre : Nombres complexes
Soit z un nombre complexe, $z \neq 1$. Démontrer que : 
\[ |z|=1 \quad \Leftrightarrow \quad \frac{1+z}{1-z} \in i \mathbb{R}.\]

\end{exo}


\begin{sol}

Supposons d'abord que $|z|=1$. Alors $z$ s'écrit $z=e^{i\theta}$, avec $ \theta \in \mathbb{R} \setminus 2 \pi \mathbb{Z}$. On peut alors écrire :
\[ \frac{1+z}{1-z}=\frac{1+e^{i \theta}}{1  - e^{i\theta}}=\frac{ e^{i \frac{\theta}{2}} \left( e^{-i \frac{\theta}{2}} + e^{i \frac{\theta}{2}} \right)}{e^{i \frac{\theta}{2}} \left( e^{-i \frac{\theta}{2}} - e^{i \frac{\theta}{2}} \right)} =
\frac{2\cos \left( \frac{\theta}{2} \right)}{-2i\sin \left( \frac{\theta}{2} \right)}
=i\frac{\cos \left( \frac{\theta}{2} \right)}{\sin \left( \frac{\theta}{2} \right)}\]
qui est bien un élément de $i\mathbb{R}$ (on note que $\sin \left( \frac{\theta}{2} \right) \neq 0$).\\
Réciproquement, supposons que $\frac{1+z}{1-z}=ia$, avec $a$ un réel alors,
\[ \frac{1+z}{1-z}=ia \quad \Leftrightarrow \quad 1+ z = ia(1-z) \quad \Leftrightarrow \quad z = \frac{-1 + ia}{1+ia}.\]
Ainsi,
\[ | z| = \left| \frac{-1 + ia}{1+ia} \right| = \frac{\sqrt{1+a^2}}{\sqrt{1+a^2}} = 1\]
ce qui prouve la réciproque.

\end{sol}

\begin{exo}

%Difficulté : 1/5
Calculer
\[ \int_0^3 \frac{\sqrt{x}}{x +1} \, dx.\]

\end{exo}


\begin{sol}

%On procède au changement de variable $t = \sqrt{x}$. Alors, $dt = \frac{1}{2\sqrt{x}} dx$ et $x = t^2$. Ainsi,
%\[ \int_0^3 \frac{\sqrt{x}}{x +1} \, dx = 
%\int_{0}^{\sqrt{3}} \frac{1}{t^2 + 1} \, 2 dt =  2 [ \arctan(t)]_{0}^{\sqrt{3}} = 2(\arctan(\sqrt{3}) - \arctan(0) ) = \frac{2\pi}{3}.\]
%
\end{sol}

\begin{exo}

%Difficulté : 1/5
Donner une primitive des fonctions suivantes :
\[ f(x) = \frac{\ln(x)}{x} \quad ; \quad  g(x) = \frac{2x + 1}{x^2(x + 1)^2}\]
\[ h(x) = \frac{x+2}{x+1} \quad \text{et} \quad i(x) = 2 \sin \left( \frac{x}{2} \right) \cos \left( \frac{x}{2} \right).\]

\end{exo}


\begin{sol}

%\begin{enumerate}
%\item On remarque qu'en posant $u(x)= \ln (x)$, on a $u'(x)=\frac{1}{x}$, de sorte que
%$f(x)=u'(x)\,u(x)$. On a donc 
%\[
% f(x)\,dx= u'(x)\,u(x)=
% \frac{d}{dx} \left(\frac{1}{2}\, u^{2}(x)\right)=
%\frac{d}{dx} \left( \frac{1}{2}\, \ln^{2}(x)\right).
%\]
%\item Si on développe le
%dénominateur $x^2(x+1)^2$ en $(x^2+x)^2$, on remarque qu'en posant $u(x)=x^2+x$, on a $u'(x)=2\,x+1$, de sorte
%que $g(x)=\frac{u'(x)}{u^{2}(x)}$. On a alors 
%\[
%g(x)= \frac{u'(x)}{u^{2}(x)}= 
%u'(x)\, u^{-2}(x)=\frac{d}{dx} \left(-\frac{1}{u(x)}\right)=\frac{d}{dx} \left(-\frac{1}{x^2+x}\right).
%\] 
%\item Il faut penser dans ce cas à faire une petite
%transformation assez simple dans les trois cas. Pour $h$, on a $
%x+2=(x+1)+1$, d'où $h(x)=\frac{(x+1)+1}{x+1}=1+\frac{1}{x+1}$,
%donc
%\begin{align*}
%h(x)\, &=  1+\frac{1}{x+1}=1+ \frac{v'(x)}{v(x)}\,dx\,,\quad \text{(en posant $v(x)=x+1$)}\\
%&= \frac{d}{dx} \left(x+\ln |v(x)|\right)= \frac{d}{dx}\left(x+\ln |x+1|\right).
%\end{align*}
%\item On peut remarquer d'après les formules de trigonométrie que $2\,\sin(\frac{x}{2})\cos(\frac{x}{2})=\sin x$, donc
%$ f_{24}(x) =  \frac{d}{dx}\left(-\cos x \right)$. On peut également remarquer qu'en posant
% $u(x)=\sin(\frac{x}{2})$, on a $u'(x)=\frac{1}{2}\, \cos(\frac{x}{2})$,
%d'où $i(x)=4\,u'(x)\, u(x)$. On a donc
%\[
%i (x)= 4\,  u'(x)\, u(x)= \frac{d}{dx}\left( 2\,u^2(x)\right)= \frac{d}{dx}\left(2\,\sin^{2}\left(\frac{x}{2}\right)\right).
%\]
%Il n'y a pas de contradiction entre les deux résultats puisque
%\[
%-\cos x = -2\, \cos^{2}\left(\frac{x}{2}\right)+1
%= -2+2\,\sin^2\left(\frac{x}{2}\right)+1
%=2\,\sin^{2}\left(\frac{x}{2}\right)-1.
%\]
%\end{enumerate}
%
\end{sol}
\end{document}