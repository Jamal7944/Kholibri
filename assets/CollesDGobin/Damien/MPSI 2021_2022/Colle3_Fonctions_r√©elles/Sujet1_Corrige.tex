%%%%%%%%%%%%%%%%%%%%%%%%%%%%%%%%%%%   Packages de base   %%%%%%%%%%%%%%%%%%%%%%%%%%%%%%%%%%%


% Classe du document (book : chapitre inclus, pratique pour les gros cours (PASS, Tremplin), article (8pt) ou amsart (12pt) : pratique pour les cours sur une seule thématique)
\documentclass[a4paper, 11pt,openany]{article}% openany évite de commencer les chapitres sur une page forcément impaire (évite les pages blanches)

% Encodage, langue et font
\usepackage[utf8]{inputenc} % Pour les accents (conversion entre ces caractères accentués et les commandes d’accentuation)
\usepackage[french]{babel} % Français
\usepackage[T1]{fontenc} % Permet d'afficher et de prendre correctement en charge ces caractères accentués du point de vue du fichier de sortie

% Si on veut faire apparaître le texte avec des caractères plus larges
%\usepackage[lf]{Baskervaldx} % Texte en plus "gras"
%\usepackage[bigdelims,vvarbb]{newtxmath} % Lettre mathématiques en plus "gras"
%\usepackage[cal=boondoxo]{mathalfa} % Style pour les \mathscal
%\renewcommand*\oldstylenums[1]{\textosf{#1}}

% Mise en page
% Version automatique mais problème d'écart entre l'en-tête et le texte.
%\usepackage{fullpage} % Numérotation bas de page (et mise en page pleine).
\usepackage{setspace} % Interligne
\onehalfspacing % Interligne
\usepackage[a4paper]{geometry}% Package pour mise en page
\geometry{hscale=0.8,vscale=0.8,centering,headsep=0.5cm} % Marges, Headsep permet de ne pas coller le texte à l'en-tête

\setlength{\parindent}{0pt} % Supprimer les identations par défaut

\usepackage{fancyhdr} %Package permettant de mettre des en-têtes et des pieds de pages
\pagestyle{fancy}
\fancyhead{} % Enleve ce qui est présent par défaut
\fancyfoot{}% Enleve ce qui est présent par défaut
\usepackage{extramarks} % Pour écrire proprement le chapitre dans l'en-tête
%\renewcommand{\chaptermark}[1]{\markboth{\chaptername\ \thechapter.\ #1}{}} % Redéfinition du chapitre pour écriture dans l'en-tête sous la forme "Chapitre n. Nom du chapitre"
%\renewcommand{\chaptermark}[1]{\markboth{\thechapter.\ #1}{}} % Forme "n. Nom du chapitre"
%\renewcommand{\chaptermark}[1]{\markboth{#1}{}} % Forme "Nom du chapitre"
\renewcommand{\headrulewidth}{0.6pt} % Epaisseur trait en haut
\renewcommand{\footrulewidth}{0.6pt} % Epaisseur trait en bas
\setlength{\headheight}{15pt}
% Placer les informations où on veut (à noter : E: Even page ; O: Odd page ; L: Left field ; C: Center field ; R: Right field ; H: Header ; F: Footer) :
\fancyhead[LO,LE]{\slshape  \textbf{MPSI2}}
\fancyhead[RO,RE]{\slshape \textbf{Corrigé - Colle 3 (Sujet 1)}}
%\fancyhead[CO,CE]{---Draft---}
\fancyfoot[C]{\thepage}
%\fancyfoot[LO, LE] {\slshape Damien GOBIN}
%\fancyfoot[RO, RE] {\slshape Année 2021-2022}

% Couleurs
\usepackage{xcolor}
\definecolor{BleuTresFonce}{rgb}{0.0,0.0,0.250}
%\definecolor{bleu}{rgb}{0.36, 0.54, 0.66}
\definecolor{bleu}{rgb}{0.47, 0.62, 0.8}
%\definecolor{vert}{rgb}{0.33, 0.42, 0.18}
%\definecolor{vert}{rgb}{0.52, 0.73, 0.4}
\definecolor{vert}{rgb}{0.56, 0.74, 0.56}
% Liens hypertextes
\usepackage[colorlinks,final,hyperindex]{hyperref}
\hypersetup{
	pdftex,
	linkcolor=BleuTresFonce,
	citecolor=BleuTresFonce,
	filecolor=BleuTresFonce,
	urlcolor=BleuTresFonce,
	pdftitle=TemplateCours,
	pdfauthor=Damien Gobin,
	pdfsubject=,
	pdfkeywords=
}


% Inclure des figures
\usepackage{graphicx} % Permet d'utiliser includegraphics
\usepackage{pdfpages} % Permet d'insérer des fichiers pdf

% Utiliser des commentaires
\usepackage{comment}
%\excludecomment{sol} %commenter cette ligne si on veut faire apparaitre les solutions d'exercices

%%%%%%%%%%%%%%%%%%%%%%%%%%%%%%%%%%%   Packages mathématiques  %%%%%%%%%%%%%%%%%%%%%%%%%%%%%%%%%%%

% Packages mathématiques de base
%\usepackage{amsfonts} % Permet de taper des ensembles 
\usepackage{amsmath} % Permet de taper des maths (contient cleveref)
\usepackage{amsthm} % Permet de définir une environnement pour les théorèmes
\usepackage{amssymb} % Donne accès a plus de symboles mathématiques (charge de façon automatique amsfonts)
% \usepackage{mathrsfs} % Ecriture ronde type mathcal
%\usepackage{bbold} % Permet d'avoir l'indicatrice mais change les textbb
\usepackage{stmaryrd} % Pour les intervalles entiers [[ ]]
\usepackage{pifont} % Pour avoir accès à plus de caractères

%Utiliser les symboles jeu de cartes
\DeclareSymbolFont{extraup}{U}{zavm}{m}{n}
\DeclareMathSymbol{\varheart}{\mathalpha}{extraup}{86}
\DeclareMathSymbol{\vardiamond}{\mathalpha}{extraup}{87}



% Référence dans le document
\usepackage[noabbrev,capitalize]{cleveref}

% Tableau
\usepackage{tabularx} % Tableau
\usepackage{multirow} % Pour créer des tableaux en subdivisant les lignes
\usepackage{diagbox} %Pour créer des crois dans les tableaux à double entrées
\usepackage{enumitem} %Permet de scinder une énumération et de reprendre au numéro suivant
\usepackage{multicol} % Création de document en colonne avec plusieurs colonnes
\multicolsep=5pt % supprime l'espace vertical

% Modification des itemizes
\setitemize{label=$\bullet$} % Utilise le package enumitem et met des points au lieu des tirets

% Ecrire des algorithmes en "français" (exemple issu du cours d'optimisation 2020/2021)
\usepackage[linesnumbered, french, frenchkw,ruled]{algorithm2e}
% Exemple :
%\begin{center}
%\begin{algorithm}
%\Entree{Un graphe $G = (V,E)$, $|V| = n$, $|E| = m$ et pour chaque arête $e$ de $E$ son poids $c(e)$;}
%\Sortie{Un arbre (ou une forêt) maximal $A = (V,F)$ et de poids minimum ;} 
%Trier et renuméroter les arêtes de $G$ dans l'ordre croissant de leur poids : $c(e_1) \leqslant c(e_2) \leqslant ... \leqslant c(e_m)$ \;
%Poser $F := \emptyset$ et $k = 0$\;
%\Tq{$k<m$ et $|F| < n-1$}{Si $e_{k+1}$ ne forme pas de cycle avec $F$ alors $F := F \cup \{ e_k \}$\;
%$k := k +1 $\;}
%\caption{Algorithme de Kruskal théorique (1956)}
%\end{algorithm}
%\end{center}

% Inclure du code Python avec coloration syntaxique et bloc gris (exemple issu du TP2 d'optimisation 2020/2021)
%\usepackage{xcolor} % Déjà importé plus haut
\usepackage{listings}
\lstset{backgroundcolor=\color{darkWhite},literate={á}{{\'a}}1 {é}{{\'e}}1 {í}{{\'i}}1 {ó}{{\'o}}1 {ú}{{\'u}}1{Á}{{\'A}}1 {É}{{\'E}}1 {Í}{{\'I}}1 {Ó}{{\'O}}1 {Ú}{{\'U}}1{à}{{\`a}}1 {è}{{\`e}}1 {ì}{{\`i}}1 {ò}{{\`o}}1 {ù}{{\`u}}1{À}{{\`A}}1 {È}{{\'E}}1 {Ì}{{\`I}}1 {Ò}{{\`O}}1 {Ù}{{\`U}}1{ä}{{\"a}}1 {ë}{{\"e}}1 {ï}{{\"i}}1 {ö}{{\"o}}1 {ü}{{\"u}}1{Ä}{{\"A}}1 {Ë}{{\"E}}1 {Ï}{{\"I}}1 {Ö}{{\"O}}1 {Ü}{{\"U}}1{â}{{\^a}}1 {ê}{{\^e}}1 {î}{{\^i}}1 {ô}{{\^o}}1 {û}{{\^u}}1{Â}{{\^A}}1 {Ê}{{\^E}}1 {Î}{{\^I}}1 {Ô}{{\^O}}1 {Û}{{\^U}}1{œ}{{\oe}}1 {Œ}{{\OE}}1 {æ}{{\ae}}1 {Æ}{{\AE}}1 {ß}{{\ss}}1{ű}{{\H{u}}}1 {Ű}{{\H{U}}}1 {ő}{{\H{o}}}1 {Ő}{{\H{O}}}1{ç}{{\c c}}1 {Ç}{{\c C}}1 {ø}{{\o}}1 {å}{{\r a}}1 {Å}{{\r A}}1{€}{{\EUR}}1 {£}{{\pounds}}1}
\lstdefinestyle{stylepython}{        language=Python,        basicstyle=\ttfamily,    commentstyle=\color{green},    keywordstyle=\color{blue},    stringstyle=\color{olive},    numberstyle=\tiny,        numbers=left,        stepnumber=1,         numbersep=5pt}
% Exemple :
%\begin{lstlisting}[style=stylepython]
%import networkx as nx #Cette bibliothèque permettra de manipuler des graphes
%import matplotlib.pyplot as plt #Cette bibliothèque permettra de les représenter
%import numpy as np
%\end{lstlisting}

%Package permettant de tracer des graphes, des courbes, etc.. (exemple issu du cours d'optimisation 2020/2021)
\usepackage{tikz}
\usepackage{tikz-cd}
\usepackage{tkz-tab} % Pour les tableaux de signes
%\usetikzlibrary{shapes,backgrounds}
\usetikzlibrary{
	decorations.pathmorphing,
	arrows,
	arrows.meta,
	calc,
	shapes,
	shapes.geometric,
	decorations.pathreplacing,}
\tikzcdset{arrow style=tikz, diagrams={>=stealth}}
\tikzset{>=stealth'}
\tikzstyle{rond}=[draw,circle,thick,fill=white]
% Exemple courbe :
%\begin{center}
%\begin{tikzpicture}
%	\def\shift{.5}
%	\def\xmax{6}
%	\def\ymax{7}
%	\draw[->] (-\shift,0) -- (\xmax+\shift,0) node[below] {$x$}; 
%	\foreach \x in {1,...,\xmax}{ \pgfmathtruncatemacro\xbis{100*\x}
%		\draw (\x,0) -- (\x,-.3*\shift) node[below] {$\scriptstyle{\xbis}$};
%	}
%	% axe ordonnees
%	\draw[->] (0,-\shift) -- (0,\ymax+\shift) node[left] {$y$};  
%	\foreach \y in {1,...,\ymax}{ \pgfmathtruncatemacro\ybis{100*\y}
%		\draw (0,\y) -- (-.3*\shift,\y) node[left] {$\scriptstyle{\ybis}$};
%	}
%	\draw[fill=yellow] (0,0) -- (3,0) -- (0,6) -- cycle;
%	
%	\draw[thick, purple] plot [domain=-.15:6.15, variable=\t] (\t,6-\t);
%	\node[above, text=purple] (A) at (6.9,.1) {$x+y = 600$};
%	%%
%	\draw[thick, purple] plot [domain=-.1:3.1, variable=\t] (\t,6-2*\t);
%	\node[right, text=purple] (B) at (1.1,5) {$2x+y = 600$};
%	%%
%	\draw[thick, blue] plot [domain=-.1:3.85, variable=\t] (\t,6-1.6*\t);
%	%%
%	\draw[fill=blue] (0,6) circle (3pt);
%	\node[right, text=blue] (B) at (0.1,6.3) {Solution optimale};
%\end{tikzpicture}
%\end{center}
% Exemple graphe :
%\begin{center}
%\begin{tikzpicture}[scale=0.9]
%    \draw[thick] (-1.5,0) node[rond] {$1$} coordinate (A) --
%        ++(3,0) node[rond] {$2$} coordinate (B) --
%        ++(-1,-1) node[rond] {$4$}  coordinate (D) --
%        ++(-1,0) node[rond] {$3$}  coordinate (C) --
%        ++(0,-1) node[rond] {$5$}  coordinate (E) --
%        ++(1,0) node[rond] {$6$}  coordinate (F) --
%        ++(1,-1) node[rond] {$8$}  coordinate (H) --
%        ++(-3,0) node[rond] {$7$}  coordinate (G) -- (A)
%        (A) -- (C)
%        (G) -- (E)
%        (D) -- (F)
%        (B) -- (H);
%\end{tikzpicture}
%\end{center}

%%%%%%%%%%%%%%%%%%%%%%%%%%%%%%%%%%%   Environnements  %%%%%%%%%%%%%%%%%%%%%%%%%%%%%%%%%%%

%Différents types à donner aux environnements théorèmes, définitions, etc...
%[theorem] permet de numéroter par rapport à la section en cours
%\renewcommand{\theexem}{\empty{}} permet de ne pas numéroterNe numérote pas les exemples

%\usepackage{thmtools} % Permet de créer un environnement de théorème avec des cadres
%\declaretheorem[thmbox=L]{boxtheorem L}
%\declaretheorem[thmbox=M]{boxtheorem M}
%\declaretheorem[thmbox=S]{boxtheorem S}
%
%%\usepackage[dvipsnames]{xcolor}
%%\declaretheorem[shaded={bgcolor=Lavender,textwidth=12em}]{BoxI}
%\declaretheorem[shaded={rulecolor=black,rulewidth=1pt}]{BoxII}
%
%
%\usepackage[leftmargin=0.1865cm,rightmargin=0.6cm]{thmbox}
% Voici ici pour les options https://ctan.mines-albi.fr/macros/latex/contrib/thmbox/thmbox.pdf

%% Ecrit de cette façon tout le monde a le même style : Titre en droit et texte en italique très léger.
%\newtheorem[L,leftmargin=0.1865cm,rightmargin=0.1865cm,cut=true]{thm}{Théorème}[subsection]
%%\renewcommand{\thetheorem}{\empty{}} 
%\newtheorem[M,leftmargin=0.1865cm,rightmargin=0.1865cm,cut=true]{propo}[thm]{Proposition}
%%\renewcommand{\thepropo}{\empty{}} 
%\newtheorem[M,leftmargin=0.1865cm,rightmargin=0.1865cm,cut=true]{prop}[thm]{Propriété}
%%\renewcommand{\theprop}{\empty{}} 
%\newtheorem[M,leftmargin=0.1865cm,rightmargin=0.1865cm,cut=true]{coro}[thm]{Corollaire}
%\newtheorem[M,leftmargin=0.1865cm,rightmargin=0.1865cm,cut=true]{lem}[thm]{Lemme}
%\newtheorem[M,leftmargin=0.1865cm,rightmargin=0.1865cm,cut=true]{defi}[theorem]{Définition}
%%\renewcommand{\thedefinition}{\empty{}}
%\newtheorem[S,leftmargin=0.1865cm,rightmargin=0.1865cm,cut=true]{exem}{Exemple}
%\renewcommand{\theexem}{\empty{}} 
%\newtheorem[S,leftmargin=0.1865cm,rightmargin=0.1865cm,cut=true]{exo}{Exercice}
%\renewcommand{\theexo}{\empty{}}
%\newtheorem[S,leftmargin=0.1865cm,rightmargin=0.1865cm,cut=true]{sol}{Solution}
%\renewcommand{\thesol}{\empty{}} 
%\newtheorem[S,leftmargin=0.1865cm,rightmargin=0.1865cm,cut=true]{hypo}{Hypothesis}[section]
%\newtheorem[S,leftmargin=0.1865cm,rightmargin=0.1865cm,cut=true]{remark}[thm]{Remarque}%[section]
%\newtheorem{dem}{Démonstration}
%\renewcommand{\thedem}{\empty{}} 
%\newtheorem[S]{nota}{Notation}[section]

\theoremstyle{plain}
\newtheorem{theorem}{Théorème}[subsection]
%\renewcommand{\thetheorem}{\empty{}} 
\newtheorem{propo}[theorem]{Proposition}
%\renewcommand{\thepropo}{\empty{}} 
\newtheorem{prop}[theorem]{Propriété}
%\renewcommand{\theprop}{\empty{}} 
\newtheorem{coro}[theorem]{Corollaire}
\newtheorem{lemma}[theorem]{Lemme}

\theoremstyle{definition}
\newtheorem{definition}[theorem]{Définition}
%\renewcommand{\thedefinition}{\empty{}}
\newtheorem{exem}{Exemple}
\renewcommand{\theexem}{\empty{}}
\newtheorem{cours}{Question de cours}
\renewcommand{\thecours}{\empty{}} 
\newtheorem{exo}{Exercice}
%\renewcommand{\theexo}{\empty{}} 
\newtheorem{sol}{Solution de l'exercice}
%\renewcommand{\thesol}{\empty{}} 
\newtheorem{hypo}{Hypothesis}[section]
\newtheorem{remark}[theorem]{Remarque}%[section]


\theoremstyle{remark}
\newtheorem{dem}{Démonstration}
\renewcommand{\thedem}{\empty{}} 
\newtheorem{nota}{Notation}[section]

%%%%%%%%%%%%%%%%%%%%%%%%%%%%%%%%%%%   Raccourcis   %%%%%%%%%%%%%%%%%%%%%%%%%%%%%%%%%%%
 
% Ensembles
\newcommand{\R}{\mathbb{R}} 
\newcommand{\Q}{\mathbb{Q}}
\newcommand{\C}{\mathbb{C}}
\newcommand{\Z}{\mathbb{Z}}
\newcommand{\K}{\mathbb{K}}
\newcommand{\N}{\mathbb{N}}
\newcommand{\D}{\mathbb{D}}


% Lettres calligraphiques
\newcommand{\calP}{\mathcal{P}} 
\newcommand{\calF}{\mathcal{F}} 
\newcommand{\calD}{\mathcal{D}} 
\newcommand{\calQ}{\mathcal{Q}} 
\newcommand{\calT}{\mathcal{T}} 

% Noyau et image
\newcommand{\im}{\text{Im}} 
\newcommand{\noy}{\text{Ker}} 
\newcommand{\card}{\text{Card}} 

%Epsilon
\newcommand{\vare}{\varepsilon} 


% Fonctions usuelles
%\newcommand{\un}{\mathbb{1}} % Indicatrice en utilisant le package \usepackage{bbold} mais celui modifie les mathbb donc on définit l'indicatrice à la main
\def\un{{\mathchoice {\rm 1\mskip-4mu l} {\rm 1\mskip-4mu l}
{\rm 1\mskip-4.5mu l} {\rm 1\mskip-5mu l}}} % Indicatrice






\title{Corrigé - Colle 3 (Sujet 1)}
\author{MPSI2\\
Année 2021-2022}
%\author{Mathématiques P.A.S.S. 1}
\date{5 octobre 2021}

\begin{document}


   \maketitle
%      \rule{\linewidth}{0.5mm}
      \rule{\linewidth}{0.5mm}
%  \begin{center}
%  %    \rule{\linewidth}{0.5mm}\\[0.4cm]
%       { \huge \bfseries Mathématiques pour le P.A.S.S 1\\[0.4cm] }
%    \rule{\linewidth}{0.5mm}\\[4cm]
%     \end{center}

\begin{cours}
Énoncer et démontrer l'inégalité triangulaire.
\end{cours}

\begin{exo}
%Difficulté : 1/5
%Chapitre : Etude de fonctions
On considère la fonction $f$ définie sur $\R$ par
\[ f(x)=\cos(3x) \cos(x)^3.\]
\begin{enumerate}
\item Pour $x \in \R$, exprimer $f(-x)$ et $f(x + \pi)$ en fonction de $f(x)$. Sur quel intervalle $I$ peut-on se contenter d'étudier $f$ ?
\item Vérifier que $f'(x)$ est du signe de $-\sin(4x)$ et en déduire le sens de variation de $f$ sur $I$.
\item Tracer la courbe représentative de $f$.
\end{enumerate}
\end{exo}


\begin{sol}
\begin{enumerate}
\item On a 
\[ f(-x)=\cos(-3x) (\cos(-x))^3= \cos(3x)\cos(x)^3=f(x).\] La fonction $f$ est donc paire. De plus, \[ f(x+\pi)=\cos(3x+3 \pi)\cos(x+\pi)^3=-\cos(3x)(-cos(x))^3=f(x).\]
$f$ est donc $\pi$-périodique. Finalement, on peut se contenter d'étudier $f$ sur l'intervalle $I= \left[0, \frac{\pi}{2} \right]$. On obtiendra alors la courbe de $f$ sur $ \left[ -\frac{\pi}{2} , \frac{\pi}{2} \right]$ par parité. Cet intervalle est de longueur $\pi$ et la fonction est $\pi$-périodique. On va donc déduire le reste de la courbe par des translations de vecteur $k\pi \overline{i}$, $k \in \Z$.
\item $f$ est dérivable sur $I$ et pour tout $x \in I$, on a 
\[ f'(x)=-3 \sin(3x) \cos(x)^3 -3\cos(3x)\sin(x) \cos (x)^2=-3cos(x)^2(\sin(3x)\cos(x)+\sin(x)\cos(3x))\]
et donc 
\[ f'(x) =-3 \cos(x)^2 \sin(4x).\] Puisque $\cos(x)^2 \geqslant 0$, $f'$ est bien du signe de $-\sin(4x)$ sur l'intervalle $\left[ 0 , \frac{\pi}{2} \right]$. En particulier, si $x \in \left[ 0, \frac{\pi}{4} \right]$, $f'(x) \leqslant 0$ et $f$ est décroissante et si $x \in \left[ \frac{\pi}{4} , \frac{\pi}{2} \right]$, $f'(x) \geqslant 0$ et $f$ est croissante. 
\item On obtient le dessin suivant : 
\begin{center}
\begin{tikzpicture}[scale=1]
     \draw [very thick, ->] (-5,0)--(5,0)node[below right]{$x$};
     \draw [very thick, ->] (0,-1)--(0,3)node[above left]{$y$};
     \draw (0,0) node[below right]{$0$};
     \draw (1,0) node[below right]{$1$};
     \draw (0,1) node[above left]{$1$};
     \draw [help lines] (-5,-1) grid (5,3);
     \draw [smooth,very thick, samples=100,domain=-5:5,color=bleu!150]plot({\x},{cos(3*\x r)*(cos(\x r))^3});
    \end{tikzpicture}
    \end{center}
\end{enumerate}
\end{sol}



\begin{exo}
%Difficulté : 2/5
%Chapitre : Fonctions usuelles
Discuter, selon les valeurs de $a\in \R$, le nombre de solutions de l'équation 
\[ \frac{1}{x-1} + \frac{1}{2} \ln \left| \frac{1+x}{1-x} \right| =a.\]
\end{exo}
%
\begin{sol}
Posons, pour $x \neq \pm 1$, 
\[ f(x) = \frac{1}{x-1} + \frac{1}{2} \ln \left| \frac{1+x}{1-x} \right| -a.\]
 La fonction $f$ est dérivable sur $\R \setminus \{ \pm 1 \}$. De plus, sa dérivée (qui ne dépend pas du signe de la quantité à l'intérieur de la valeur absolue dans le logarithme) est égale, après mise au même dénominateur, à 
 \[ f'(x)= \frac{-2x}{(1-x)^2(1+x)}.\]
On en déduit le tableau de variations suivant pour la fonction (le calcul des limites ne pose pas de difficultés particulières; en particulier, il n'y a pas de formes indéterminées) :
\[
	\begin{tikzpicture}
		\tkzTabInit{$x$ /.8 , $f'(x)$ /.8 , $f(x)$ /2}{$-\infty$,$-1$,$0$,$1$,$+\infty$}
		   \tkzTabLine{,-,d,+,0,-,d,-,}
		\tkzTabVar{ +/$-a$,  -D-/ $-\infty$, +/$-a-1$, -D+/ $-\infty$ /$+\infty$, -/$-a$ /}
	\end{tikzpicture}
\]
Par continuité de $f$, en utilisant de plus sa stricte monotonie sur les intervalles $]- \infty,-1[$, $]-1,0[$, $]0,1[$ et $]1,+\infty[$, on discute le nombre de solutions suivant la valeur de $a$ :
\begin{itemize}
\item Si $a=0$, l'équation n'admet pas de solutions.
\item Si $a>0$, l'équation admet une unique solution qui est située dans l'intervalle $]1,+\infty[$.
\item Si $a\in ]-1,0[$, l'équation admet une unique solution qui est située dans l'intervalle $]-\infty ,-1[$.
\item Si $a=-1$, l'équation admet deux solutions. L'une de ces solutions est $0$, l'autre est située dans l'intervalle $]-\infty,-1[$.
\item Si $a<-1$, l'équation admet exactement trois solutions. L'une est située dans l'intervalle $]-\infty,-1[$, la seconde dans l'intervalle $]-1,0[$ et la troisième dans l'intervalle $]0,1[$.
\end{itemize}
\end{sol}


\begin{exo}
%Difficulté : 2/5
%Chapitre : Fonctions usuelles
Soit $g:\R^+ \to \R$ définie par $g(x)=(x-2)e^x+(x+2)$. Démontrer que $g$ est positive ou nulle sur $\R^+$.
\end{exo}

\begin{sol}
On va étudier $g$. Pour cela, il faut aller jusqu'à la dérivée seconde! En effet, $g$ est de classe $C^{\infty}$ sur $\R^+$, avec $g'(x)=(x-1)e^x+1$. Il ne semble pas facile d'étudier directement le signe de $g'$. On va donc calculer la dérivée de $g'$, qui est $g''(x)=xe^x$, $x \in \R^+$. $g''$ est positive sur $\R^+$, donc $g'$ est croissante sur cet intervalle. De plus, $g'(0)=0$ donc $g'$ est positive sur $\R^+$. Ainsi, $g$ est croissante sur $\R^+$ et comme $g(0)=0$, $g$ est positive sur $\R^+$.
\end{sol}


\begin{exo}
%Difficulté : 4/5
%Chapitre : Fonctions usuelles
Soit $p \geqslant 2$ un entier et $0<a_1< \dots <a_p$ des nombres réels positifs.
\begin{enumerate}
\item Montrer que, pour tout $a>a_p$, l'équation 
\[ a_1^x+ \dots +a_p^x =a^x\]
 admet une unique racine $x_a$.
 \item Étudier le sens de variation de $a \mapsto x_a$.
 \item Déterminer l'existence et calculer 
 \[ \lim_{a\to + \infty} x_a \quad \text{et} \quad \lim_{ a \to + \infty} x_a \ln(a). \]
\end{enumerate}
\end{exo}

\begin{sol}
\begin{enumerate}
\item On introduit la fonction 
\[ f_a(x)= \left( \frac{a_1}{a} \right)^x + \dots + \left( \frac{a_1}{a} \right)^x = \sum_{k=1}^p e^{x \ln \left( \frac{a_k}{a} \right)}.\]
Puisque $\ln \left( \frac{a_k}{a} \right)<0$, $x \mapsto x\ln \left( \frac{a_k}{a} \right)$ est strictement décroissante, et donc $f_a$ est strictement décroissante. Or, $f_a(0)=p$ et \[ \lim_{x\to + \infty} f_a(x) = 0.\]
L'équation $f_a(x)=1$ admet donc une unique racine $x_a>0$.
 \item Soit $a<b$. En reprenant la notation de la question précédente, pour tout $x>0$, on a $f_a(x) \geqslant f_b(x)$. En particulier $f_b(x_b)=f_a(x_a)=1 \geqslant f_b(x_a)$. Par décroissance de $f_b$, on en déduit que $x_a \geqslant x_b$ et donc $a \mapsto x_a$ est décroissante.
 \item Puisque $a\mapsto x_a$ est décroissante et minorée par $0$, elle admet une limite $\ell \geqslant 0$ en $+ \infty$. Supposons $\ell >0$. Alors, en passant à la limite dans 
 \[ a_1^{x_a} + \dots + a_p^{x_a}=a^{x_a},\]
 on trouve 
  \[ a_1^{\ell} + \dots + a_p^{\ell}=+ \infty,\]
une contradiction. Donc $\ell=0$. Ainsi, il vient également
\[ x_a \ln(a) = \ln \left(a_1^{x_a} + \dots + a_p^{x_a} \right),\]
ce qui prouve que $x_a \ln(a)$ tend vers $\ln(p)$. 
\end{enumerate}
\end{sol}

















\end{document}
