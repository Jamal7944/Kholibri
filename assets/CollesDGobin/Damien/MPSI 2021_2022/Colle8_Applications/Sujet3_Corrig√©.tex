%%%%%%%%%%%%%%%%%%%%%%%%%%%%%%%%%%%   Packages de base   %%%%%%%%%%%%%%%%%%%%%%%%%%%%%%%%%%%


% Classe du document (book : chapitre inclus, pratique pour les gros cours (PASS, Tremplin), article (8pt) ou amsart (12pt) : pratique pour les cours sur une seule thématique)
\documentclass[a4paper, 11pt,openany]{article}% openany évite de commencer les chapitres sur une page forcément impaire (évite les pages blanches)

% Encodage, langue et font
\usepackage[utf8]{inputenc} % Pour les accents (conversion entre ces caractères accentués et les commandes d’accentuation)
\usepackage[french]{babel} % Français
\usepackage[T1]{fontenc} % Permet d'afficher et de prendre correctement en charge ces caractères accentués du point de vue du fichier de sortie

% Si on veut faire apparaître le texte avec des caractères plus larges
%\usepackage[lf]{Baskervaldx} % Texte en plus "gras"
%\usepackage[bigdelims,vvarbb]{newtxmath} % Lettre mathématiques en plus "gras"
%\usepackage[cal=boondoxo]{mathalfa} % Style pour les \mathscal
%\renewcommand*\oldstylenums[1]{\textosf{#1}}

% Mise en page
% Version automatique mais problème d'écart entre l'en-tête et le texte.
%\usepackage{fullpage} % Numérotation bas de page (et mise en page pleine).
\usepackage{setspace} % Interligne
\onehalfspacing % Interligne
\usepackage[a4paper]{geometry}% Package pour mise en page
\geometry{hscale=0.8,vscale=0.8,centering,headsep=0.5cm} % Marges, Headsep permet de ne pas coller le texte à l'en-tête

\setlength{\parindent}{0pt} % Supprimer les identations par défaut

\usepackage{fancyhdr} %Package permettant de mettre des en-têtes et des pieds de pages
\pagestyle{fancy}
\fancyhead{} % Enleve ce qui est présent par défaut
\fancyfoot{}% Enleve ce qui est présent par défaut
\usepackage{extramarks} % Pour écrire proprement le chapitre dans l'en-tête
%\renewcommand{\chaptermark}[1]{\markboth{\chaptername\ \thechapter.\ #1}{}} % Redéfinition du chapitre pour écriture dans l'en-tête sous la forme "Chapitre n. Nom du chapitre"
%\renewcommand{\chaptermark}[1]{\markboth{\thechapter.\ #1}{}} % Forme "n. Nom du chapitre"
%\renewcommand{\chaptermark}[1]{\markboth{#1}{}} % Forme "Nom du chapitre"
\renewcommand{\headrulewidth}{0.6pt} % Epaisseur trait en haut
\renewcommand{\footrulewidth}{0.6pt} % Epaisseur trait en bas
\setlength{\headheight}{15pt}
% Placer les informations où on veut (à noter : E: Even page ; O: Odd page ; L: Left field ; C: Center field ; R: Right field ; H: Header ; F: Footer) :
\fancyhead[LO,LE]{\slshape  \textbf{MPSI2}}
\fancyhead[RO,RE]{\slshape \textbf{Corrigé - Colle 8 (Sujet 3)}}
%\fancyhead[CO,CE]{---Draft---}
\fancyfoot[C]{\thepage}
%\fancyfoot[LO, LE] {\slshape Damien GOBIN}
%\fancyfoot[RO, RE] {\slshape Année 2021-2022}

% Couleurs
\usepackage{xcolor}
\definecolor{BleuTresFonce}{rgb}{0.0,0.0,0.250}
%\definecolor{bleu}{rgb}{0.36, 0.54, 0.66}
\definecolor{bleu}{rgb}{0.47, 0.62, 0.8}
%\definecolor{vert}{rgb}{0.33, 0.42, 0.18}
%\definecolor{vert}{rgb}{0.52, 0.73, 0.4}
\definecolor{vert}{rgb}{0.56, 0.74, 0.56}
% Liens hypertextes
\usepackage[colorlinks,final,hyperindex]{hyperref}
\hypersetup{
	pdftex,
	linkcolor=BleuTresFonce,
	citecolor=BleuTresFonce,
	filecolor=BleuTresFonce,
	urlcolor=BleuTresFonce,
	pdftitle=TemplateCours,
	pdfauthor=Damien Gobin,
	pdfsubject=,
	pdfkeywords=
}


% Inclure des figures
\usepackage{graphicx} % Permet d'utiliser includegraphics
\usepackage{pdfpages} % Permet d'insérer des fichiers pdf

% Utiliser des commentaires
\usepackage{comment}
%\excludecomment{sol} %commenter cette ligne si on veut faire apparaitre les solutions d'exercices

%%%%%%%%%%%%%%%%%%%%%%%%%%%%%%%%%%%   Packages mathématiques  %%%%%%%%%%%%%%%%%%%%%%%%%%%%%%%%%%%

% Packages mathématiques de base
%\usepackage{amsfonts} % Permet de taper des ensembles 
\usepackage{amsmath} % Permet de taper des maths (contient cleveref)
\usepackage{amsthm} % Permet de définir une environnement pour les théorèmes
\usepackage{amssymb} % Donne accès a plus de symboles mathématiques (charge de façon automatique amsfonts)
% \usepackage{mathrsfs} % Ecriture ronde type mathcal
%\usepackage{bbold} % Permet d'avoir l'indicatrice mais change les textbb
\usepackage{stmaryrd} % Pour les intervalles entiers [[ ]]
\usepackage{pifont} % Pour avoir accès à plus de caractères

%Utiliser les symboles jeu de cartes
\DeclareSymbolFont{extraup}{U}{zavm}{m}{n}
\DeclareMathSymbol{\varheart}{\mathalpha}{extraup}{86}
\DeclareMathSymbol{\vardiamond}{\mathalpha}{extraup}{87}



% Référence dans le document
\usepackage[noabbrev,capitalize]{cleveref}

% Tableau
\usepackage{tabularx} % Tableau
\usepackage{multirow} % Pour créer des tableaux en subdivisant les lignes
\usepackage{diagbox} %Pour créer des crois dans les tableaux à double entrées
\usepackage{enumitem} %Permet de scinder une énumération et de reprendre au numéro suivant
\usepackage{multicol} % Création de document en colonne avec plusieurs colonnes
\multicolsep=5pt % supprime l'espace vertical

% Modification des itemizes
\setitemize{label=$\bullet$} % Utilise le package enumitem et met des points au lieu des tirets

% Ecrire des algorithmes en "français" (exemple issu du cours d'optimisation 2020/2021)
\usepackage[linesnumbered, french, frenchkw,ruled]{algorithm2e}
% Exemple :
%\begin{center}
%\begin{algorithm}
%\Entree{Un graphe $G = (V,E)$, $|V| = n$, $|E| = m$ et pour chaque arête $e$ de $E$ son poids $c(e)$;}
%\Sortie{Un arbre (ou une forêt) maximal $A = (V,F)$ et de poids minimum ;} 
%Trier et renuméroter les arêtes de $G$ dans l'ordre croissant de leur poids : $c(e_1) \leqslant c(e_2) \leqslant ... \leqslant c(e_m)$ \;
%Poser $F := \emptyset$ et $k = 0$\;
%\Tq{$k<m$ et $|F| < n-1$}{Si $e_{k+1}$ ne forme pas de cycle avec $F$ alors $F := F \cup \{ e_k \}$\;
%$k := k +1 $\;}
%\caption{Algorithme de Kruskal théorique (1956)}
%\end{algorithm}
%\end{center}

% Inclure du code Python avec coloration syntaxique et bloc gris (exemple issu du TP2 d'optimisation 2020/2021)
%\usepackage{xcolor} % Déjà importé plus haut
\usepackage{listings}
\lstset{backgroundcolor=\color{darkWhite},literate={á}{{\'a}}1 {é}{{\'e}}1 {í}{{\'i}}1 {ó}{{\'o}}1 {ú}{{\'u}}1{Á}{{\'A}}1 {É}{{\'E}}1 {Í}{{\'I}}1 {Ó}{{\'O}}1 {Ú}{{\'U}}1{à}{{\`a}}1 {è}{{\`e}}1 {ì}{{\`i}}1 {ò}{{\`o}}1 {ù}{{\`u}}1{À}{{\`A}}1 {È}{{\'E}}1 {Ì}{{\`I}}1 {Ò}{{\`O}}1 {Ù}{{\`U}}1{ä}{{\"a}}1 {ë}{{\"e}}1 {ï}{{\"i}}1 {ö}{{\"o}}1 {ü}{{\"u}}1{Ä}{{\"A}}1 {Ë}{{\"E}}1 {Ï}{{\"I}}1 {Ö}{{\"O}}1 {Ü}{{\"U}}1{â}{{\^a}}1 {ê}{{\^e}}1 {î}{{\^i}}1 {ô}{{\^o}}1 {û}{{\^u}}1{Â}{{\^A}}1 {Ê}{{\^E}}1 {Î}{{\^I}}1 {Ô}{{\^O}}1 {Û}{{\^U}}1{œ}{{\oe}}1 {Œ}{{\OE}}1 {æ}{{\ae}}1 {Æ}{{\AE}}1 {ß}{{\ss}}1{ű}{{\H{u}}}1 {Ű}{{\H{U}}}1 {ő}{{\H{o}}}1 {Ő}{{\H{O}}}1{ç}{{\c c}}1 {Ç}{{\c C}}1 {ø}{{\o}}1 {å}{{\r a}}1 {Å}{{\r A}}1{€}{{\EUR}}1 {£}{{\pounds}}1}
\lstdefinestyle{stylepython}{        language=Python,        basicstyle=\ttfamily,    commentstyle=\color{green},    keywordstyle=\color{blue},    stringstyle=\color{olive},    numberstyle=\tiny,        numbers=left,        stepnumber=1,         numbersep=5pt}
% Exemple :
%\begin{lstlisting}[style=stylepython]
%import networkx as nx #Cette bibliothèque permettra de manipuler des graphes
%import matplotlib.pyplot as plt #Cette bibliothèque permettra de les représenter
%import numpy as np
%\end{lstlisting}

%Package permettant de tracer des graphes, des courbes, etc.. (exemple issu du cours d'optimisation 2020/2021)
\usepackage{tikz}
\usepackage{tikz-cd}
\usepackage{tkz-tab} % Pour les tableaux de signes
%\usetikzlibrary{shapes,backgrounds}
\usetikzlibrary{
	decorations.pathmorphing,
	arrows,
	arrows.meta,
	calc,
	shapes,
	shapes.geometric,
	decorations.pathreplacing,}
\tikzcdset{arrow style=tikz, diagrams={>=stealth}}
\tikzset{>=stealth'}
\tikzstyle{rond}=[draw,circle,thick,fill=white]
% Exemple courbe :
%\begin{center}
%\begin{tikzpicture}
%	\def\shift{.5}
%	\def\xmax{6}
%	\def\ymax{7}
%	\draw[->] (-\shift,0) -- (\xmax+\shift,0) node[below] {$x$}; 
%	\foreach \x in {1,...,\xmax}{ \pgfmathtruncatemacro\xbis{100*\x}
%		\draw (\x,0) -- (\x,-.3*\shift) node[below] {$\scriptstyle{\xbis}$};
%	}
%	% axe ordonnees
%	\draw[->] (0,-\shift) -- (0,\ymax+\shift) node[left] {$y$};  
%	\foreach \y in {1,...,\ymax}{ \pgfmathtruncatemacro\ybis{100*\y}
%		\draw (0,\y) -- (-.3*\shift,\y) node[left] {$\scriptstyle{\ybis}$};
%	}
%	\draw[fill=yellow] (0,0) -- (3,0) -- (0,6) -- cycle;
%	
%	\draw[thick, purple] plot [domain=-.15:6.15, variable=\t] (\t,6-\t);
%	\node[above, text=purple] (A) at (6.9,.1) {$x+y = 600$};
%	%%
%	\draw[thick, purple] plot [domain=-.1:3.1, variable=\t] (\t,6-2*\t);
%	\node[right, text=purple] (B) at (1.1,5) {$2x+y = 600$};
%	%%
%	\draw[thick, blue] plot [domain=-.1:3.85, variable=\t] (\t,6-1.6*\t);
%	%%
%	\draw[fill=blue] (0,6) circle (3pt);
%	\node[right, text=blue] (B) at (0.1,6.3) {Solution optimale};
%\end{tikzpicture}
%\end{center}
% Exemple graphe :
%\begin{center}
%\begin{tikzpicture}[scale=0.9]
%    \draw[thick] (-1.5,0) node[rond] {$1$} coordinate (A) --
%        ++(3,0) node[rond] {$2$} coordinate (B) --
%        ++(-1,-1) node[rond] {$4$}  coordinate (D) --
%        ++(-1,0) node[rond] {$3$}  coordinate (C) --
%        ++(0,-1) node[rond] {$5$}  coordinate (E) --
%        ++(1,0) node[rond] {$6$}  coordinate (F) --
%        ++(1,-1) node[rond] {$8$}  coordinate (H) --
%        ++(-3,0) node[rond] {$7$}  coordinate (G) -- (A)
%        (A) -- (C)
%        (G) -- (E)
%        (D) -- (F)
%        (B) -- (H);
%\end{tikzpicture}
%\end{center}

%%%%%%%%%%%%%%%%%%%%%%%%%%%%%%%%%%%   Environnements  %%%%%%%%%%%%%%%%%%%%%%%%%%%%%%%%%%%

%Différents types à donner aux environnements théorèmes, définitions, etc...
%[theorem] permet de numéroter par rapport à la section en cours
%\renewcommand{\theexem}{\empty{}} permet de ne pas numéroterNe numérote pas les exemples

%\usepackage{thmtools} % Permet de créer un environnement de théorème avec des cadres
%\declaretheorem[thmbox=L]{boxtheorem L}
%\declaretheorem[thmbox=M]{boxtheorem M}
%\declaretheorem[thmbox=S]{boxtheorem S}
%
%%\usepackage[dvipsnames]{xcolor}
%%\declaretheorem[shaded={bgcolor=Lavender,textwidth=12em}]{BoxI}
%\declaretheorem[shaded={rulecolor=black,rulewidth=1pt}]{BoxII}
%
%
%\usepackage[leftmargin=0.1865cm,rightmargin=0.6cm]{thmbox}
% Voici ici pour les options https://ctan.mines-albi.fr/macros/latex/contrib/thmbox/thmbox.pdf

%% Ecrit de cette façon tout le monde a le même style : Titre en droit et texte en italique très léger.
%\newtheorem[L,leftmargin=0.1865cm,rightmargin=0.1865cm,cut=true]{thm}{Théorème}[subsection]
%%\renewcommand{\thetheorem}{\empty{}} 
%\newtheorem[M,leftmargin=0.1865cm,rightmargin=0.1865cm,cut=true]{propo}[thm]{Proposition}
%%\renewcommand{\thepropo}{\empty{}} 
%\newtheorem[M,leftmargin=0.1865cm,rightmargin=0.1865cm,cut=true]{prop}[thm]{Propriété}
%%\renewcommand{\theprop}{\empty{}} 
%\newtheorem[M,leftmargin=0.1865cm,rightmargin=0.1865cm,cut=true]{coro}[thm]{Corollaire}
%\newtheorem[M,leftmargin=0.1865cm,rightmargin=0.1865cm,cut=true]{lem}[thm]{Lemme}
%\newtheorem[M,leftmargin=0.1865cm,rightmargin=0.1865cm,cut=true]{defi}[theorem]{Définition}
%%\renewcommand{\thedefinition}{\empty{}}
%\newtheorem[S,leftmargin=0.1865cm,rightmargin=0.1865cm,cut=true]{exem}{Exemple}
%\renewcommand{\theexem}{\empty{}} 
%\newtheorem[S,leftmargin=0.1865cm,rightmargin=0.1865cm,cut=true]{exo}{Exercice}
%\renewcommand{\theexo}{\empty{}}
%\newtheorem[S,leftmargin=0.1865cm,rightmargin=0.1865cm,cut=true]{sol}{Solution}
%\renewcommand{\thesol}{\empty{}} 
%\newtheorem[S,leftmargin=0.1865cm,rightmargin=0.1865cm,cut=true]{hypo}{Hypothesis}[section]
%\newtheorem[S,leftmargin=0.1865cm,rightmargin=0.1865cm,cut=true]{remark}[thm]{Remarque}%[section]
%\newtheorem{dem}{Démonstration}
%\renewcommand{\thedem}{\empty{}} 
%\newtheorem[S]{nota}{Notation}[section]

\theoremstyle{plain}
\newtheorem{theorem}{Théorème}[subsection]
%\renewcommand{\thetheorem}{\empty{}} 
\newtheorem{propo}[theorem]{Proposition}
%\renewcommand{\thepropo}{\empty{}} 
\newtheorem{prop}[theorem]{Propriété}
%\renewcommand{\theprop}{\empty{}} 
\newtheorem{coro}[theorem]{Corollaire}
\newtheorem{lemma}[theorem]{Lemme}

\theoremstyle{definition}
\newtheorem{definition}[theorem]{Définition}
%\renewcommand{\thedefinition}{\empty{}}
\newtheorem{exem}{Exemple}
\renewcommand{\theexem}{\empty{}} 
\newtheorem{cours}{Question de cours}
\renewcommand{\thecours}{\empty{}} 
\newtheorem{exo}{Exercice}
%\renewcommand{\theexo}{\empty{}} 
\newtheorem{sol}{Solution de l'exercice}
%\renewcommand{\thesol}{\empty{}} 
\newtheorem{hypo}{Hypothesis}[section]
\newtheorem{remark}[theorem]{Remarque}%[section]


\theoremstyle{remark}
\newtheorem{dem}{Démonstration}
\renewcommand{\thedem}{\empty{}} 
\newtheorem{nota}{Notation}[section]

%%%%%%%%%%%%%%%%%%%%%%%%%%%%%%%%%%%   Raccourcis   %%%%%%%%%%%%%%%%%%%%%%%%%%%%%%%%%%%
 
% Ensembles
\newcommand{\R}{\mathbb{R}} 
\newcommand{\Q}{\mathbb{Q}}
\newcommand{\C}{\mathbb{C}}
\newcommand{\Z}{\mathbb{Z}}
\newcommand{\K}{\mathbb{K}}
\newcommand{\N}{\mathbb{N}}
\newcommand{\D}{\mathbb{D}}


% Lettres calligraphiques
\newcommand{\calP}{\mathcal{P}} 
\newcommand{\calF}{\mathcal{F}} 
\newcommand{\calD}{\mathcal{D}} 
\newcommand{\calQ}{\mathcal{Q}} 
\newcommand{\calT}{\mathcal{T}} 

% Noyau et image
\newcommand{\im}{\text{Im}} 
\newcommand{\noy}{\text{Ker}} 
\newcommand{\card}{\text{Card}} 

%Epsilon
\newcommand{\vare}{\varepsilon} 


% Fonctions usuelles
%\newcommand{\un}{\mathbb{1}} % Indicatrice en utilisant le package \usepackage{bbold} mais celui modifie les mathbb donc on définit l'indicatrice à la main
\def\un{{\mathchoice {\rm 1\mskip-4mu l} {\rm 1\mskip-4mu l}
{\rm 1\mskip-4.5mu l} {\rm 1\mskip-5mu l}}} % Indicatrice


\title{Corrigé - Colle 8 (Sujet 3)}
\author{MPSI2\\
Année 2021-2022}
%\author{Mathématiques P.A.S.S. 1}
\date{23 novembre 2021}


\begin{document}


   \maketitle
%      \rule{\linewidth}{0.5mm}
      \rule{\linewidth}{0.5mm}
%  \begin{center}
%  %    \rule{\linewidth}{0.5mm}\\[0.4cm]
%       { \huge \bfseries Mathématiques pour le P.A.S.S 1\\[0.4cm] }
%    \rule{\linewidth}{0.5mm}\\[4cm]
%     \end{center}

\begin{cours}
Soient $f : E \to F$ et $g : F \to E$. Montrer que si $f \circ g = Id_F$ et $g \circ f = Id_E$ alors $f$ est une bijection et $f^{-1} = g$.
\end{cours}


\begin{exo}
%Difficulté : 1/5
%Chapitre : Applications
Soit $f\colon x \mapsto \dfrac{x}{x+1}$. Exprimer $f^n(x) = \underbrace{f\circ f \circ \ldots \circ f}_{n\text{ fois}}(x)$ pour tout entier naturel $n \geqslant 1$.
\end{exo}

\begin{sol}
Montrons par récurrence que la propriété $f^n(x) = \frac{x}{nx+1}$ est vraie pour tout $n \geqslant 1$.
\begin{itemize}
\item \textbf{Initialisation.} Pour $n=1$ le résultat provient immédiatement de la définition de $f$.
\item \textbf{Hérédité.} Soit $n \geqslant 1$ fixé. Supposons que le résultat est vrai au rang $n$, i.e. que $f^n(x) = \frac{x}{nx+1}$. Alors,
\[ f^{n+1}(x) = f(f^n(x)) = f \left( \frac{x}{nx+1} \right) = \frac{\frac{x}{nx+1}}{\frac{x}{nx+1} +1 } = \frac{x}{x+(nx+1)} = \frac{x}{(n+1)x+1}.\]
Le résultat est donc vrai au rang $n+1$.
\item \textbf{Conclusion.} On a donc démontré par récurrence que pour tout $n \geqslant 1$, $f^n(x) = \frac{x}{nx+1}$.
\end{itemize}
\end{sol}


\begin{exo}
%Difficulté : 1/5
%Chapitre : Ensembles
On définit les ensembles
\[ K = [2,5] \times [-1,4] \quad \text{et} \quad D = \{(x,y) \in \R^2, y \leqslant x \}.\]
\begin{enumerate}
\item Représenter dans le plan le domaine des points $M(x,y)$ appartenant à $K$ et le domaine des points $N(x,y)$ appartenant à $D$.
\item L'implication suivante est-elle vraie ?
\[ \forall (x,y) \in \R^2, \quad (x,y) \in K \quad \Rightarrow \quad (x + 1,y - 1)\in D.\]
\item La réciproque est-elle vraie ?
\item Montrer que $K \cap D \neq \emptyset$.
\end{enumerate}
\end{exo}

\begin{sol}
\begin{enumerate}
\item On a les représentations suivantes (où on a mis en gras les "bords" inclus dans le domaine) :
  \begin{center}
\begin{tikzpicture}[scale=0.8]
      \draw [very thick, ->] (-1,0)--(6,0);
     \draw [very thick, ->] (0,-2)--(0,5);
     \draw [help lines] (-1,-2) grid (6,5);
\fill[bleu!100,opacity=0.2] (2,-1) rectangle (5,4);
     \draw [very thick,bleu!150] (2,-1)--(5,-1);
     \draw [very thick,bleu!150] (2,4)--(5,4);
     \draw [very thick,bleu!150] (2,-1)--(2,4);
     \draw [very thick,bleu!150] (5,-1)--(5,4);
     \draw (1.2,-0.3) node {$1$};
     \draw (-0.2,1.2) node {$1$};
     \draw (2.5,-2.5) node {\textbf{Ensemble $K$}};
    \end{tikzpicture}
\hspace{1.5cm} \text{et} \hspace{1.5cm}
\begin{tikzpicture}[scale=0.8]
      \draw [very thick, ->] (-1,0)--(6,0);
     \draw [very thick, ->] (0,-2)--(0,5);
     \draw [help lines] (-1,-2) grid (6,5);
\fill[bleu!100,opacity=0.2]  (-1,-1) -- (-1,-2) -- (6,-2)-- (6,5) -- (5,5) -- cycle;
     \draw [very thick,bleu!150] (-1,-1)--(5,5);
     \draw (1.2,-0.3) node {$1$};
     \draw (-0.2,1.2) node {$1$};
     \draw (2.5,-2.5) node {\textbf{Ensemble $D$}};
    \end{tikzpicture}
    \end{center}
\item Cette implication est vraie. En effet, si $(x,y) \in K$, alors $2 \leqslant x \leqslant 5$ et $-1 \leqslant y \leqslant 4$. Ainsi, $3 \leqslant x+1 \leqslant 6$ et $ -2 \leqslant y-1\leqslant 3$ et donc $y-1 \leqslant x+1$ ce qui signifie que $(x+1,y-1) \in D$.
\item La réciproque est fausse. En effet, par exemple $(10,9) \in D$ mais $(10 - 1,9 + 1) = (9,10) \notin K$.
\item $K \cap D \neq \emptyset$ car $(3,3) \in K \cap D$.
\end{enumerate}
\end{sol}



\begin{exo}
%Difficulté : 2/5
%Chapitre : Applications
On considère la fonction 
$ h : x \mapsto \frac{x^2 +1}{2x - 1}$.
\begin{enumerate}
\item Déterminer l'ensemble de définition $\calD_h$ de $h$.
\item Déterminer l'image de $h$, i.e. $f(\calD_h)$.
\item L'application $h$ est-elle injective de $\calD_h$ dans $\R$ ? Surjective ? Bijective ?
\end{enumerate}
\end{exo}

\begin{sol}
\begin{enumerate}
\item L'ensemble de définition de $h$ est $\calD_h = \R \setminus \left\{ \frac{1}{2} \right\}$.
\item L'image de $h$ n'est pas si évidente à trouver. Soit $y \in \R$. Supposons qu'il existe $x$ tel que $h(x) =y$. Alors,
\[ \frac{x^2 +1}{2x - 1} = y \quad \Leftrightarrow \quad x^2 +1 = y(2x - 1) \quad \Leftrightarrow \quad x^2 - 2yx +1 + y =0.\]
On obtient donc un trinôme en $x$. Pour déterminer si ce trinôme possède des racines (et donc si $y$ possède un antécédent), on calcule son discriminant. On a
\[ \Delta = 4y^2 - 4(1+y) = 4y^2 - 4y - 4.\]
Ce discriminant est à nouveau un trinôme. Pour déterminer son signe, nous devons à présent calculer le discriminant de ce dernier trinôme. On obtient
\[ \Delta' = 16 + 64 = 80 > 0.\]
Le trinôme $4y^2 - 4y - 4$ possède donc deux racines réelles distinctes
\[ y_1 = \frac{4 - \sqrt{80}}{8} = \frac{1 - \sqrt{5}}{2} \quad \text{et} \quad y_2 = \frac{4 + 4\sqrt{80}}{8} = \frac{1 +\sqrt{5}}{2}.\]
Ainsi, on a deux situations possibles :
\begin{itemize}
\item Si $y \in \left]  \frac{1 - \sqrt{5}}{2} ,  \frac{1 + \sqrt{5}}{2} \right[$ alors $\Delta$ est strictement négatif (car il est du signe de $a = 4$ sauf entre les racines). Dans ce cas, l'équation $x^2 - 2yx +1 + y =0$ ne possède pas de solution et $y$ n'a donc pas d'antécédent.
\item Si $y \notin \left]  \frac{1 - \sqrt{5}}{2} ,  \frac{1 + \sqrt{5}}{2} \right[$ alors $\Delta \geqslant 0$ et dans ce cas, l'équation $x^2 - 2yx +1 + y =0$ possède une (si $y = y_1$ ou $y = y_2$ et donc $\Delta = 0$) ou deux solutions et $y$ a donc bien d'antécédent.
\end{itemize}
En conclusion, on a
\[ \im(h)= \left] - \infty ,  \frac{1 - \sqrt{5}}{2} \right] \cup \left[  \frac{1 + \sqrt{5}}{2} , + \infty \right[.\]
\item $h$ n'est pas injective de $\calD_h$ dans $\R$ car $h(1) = 2 = h(3)$. On pourrait utiliser le fait que l'image de $h$ n'est pas $\R$ pour conclure immédiatement que $h$ n'est pas surjective de $\mathcal{D}_h$ dans $\R$. Néanmoins, pour montrer que l'application $h$ n'est pas surjective de $\calD_h$ dans $\R$ il n'est pas utile de connaître son image. En effet, $0$ n'a pas d'antécédent par $h$ car $h(x) = 0$ équivaut à $x^2 +1 =0$ qui n'a pas de solution réelle. $h$ n'est pas bijective car non surjective.
\end{enumerate}
\end{sol}

\begin{exo}
%Difficulté : 2/5
%Chapitre : Applications
Soit $f :E \to F$ et $A \subset F$. Montrer que $ A \subset f^{-1}(f(A))$. Trouver un contre-exemple pour l'autre inclusion. Que peut-on dire si $f$ est de plus injective ?
\end{exo}

\begin{sol}
\begin{enumerate}
 \item Soit $x\in A$. Alors, $x\in f^{-1}(f(A))$ équivaut à $f(x)\in f(A)$. Or $f(x)\in f(A)$ puisque $x\in A$ donc on a bien $A \subset f^{-1}(f(A)$. 
  \item Donnons un contre-exemple lorsque $f$ n'est pas injective. On peut choisir $f\colon \mathbb R \to \mathbb R$ définie par $f(x)=\cos(x)$. Pour $A=[0,2\pi]$ on a $f(A)=[-1,1]$ et $f^{-1}f(A))=\mathbb R$. Donc $f^{-1}(f(A))$ n'est pas inclu dans $A$.
 \item Supposons que $f$ est injective. Soit $x\in f^{-1}(f(A))$. Alors, $f(x)\in f(A)$. Donc il existe $x' \in A$ tel que $f(x)=f(x')$. Par injectivité de $f$, on a donc $x=x'$. D'où $x\in A$ et $f^{-1}(f(A)) \subset A$. Ainsi, lorsque $f$ est injective on a  $f^{-1}(f(A)) = A$.
\end{enumerate}
\end{sol}
   

\begin{exo}
%Difficulté : 2/5
%Chapitre : Applications
Démontrer que la fonction $f: \R \to \R_+^{\star}$ définie par 
\[ f(x)=\frac{e^x+2}{e^{-x}}\]
est bijective. Calculer sa bijection réciproque $f^{-1}$.
\end{exo}

\begin{sol}
La première chose à remarquer est que $f$ s'écrit plus facilement 
\[ f(x)=e^{2x}+2e^x.\]
Soit $y >0$. Alors on a 
\[ y=f(x) \quad \Leftrightarrow \quad e^{2x}+2e^{x}-y=0.\]
On pose alors $X=e^x$ et l'équation est équivalente à l'équation du second degré \[ X^2+2X-y=0\]
dont les racines sont 
\[ X_1=-1- \sqrt{1+y} \quad \text{et} \quad X_2=-1+\sqrt{1+y}.\]
Mais $e^x>0$, et donc on a \[ y=f(x) \quad \Leftrightarrow \quad e^x=-1+\sqrt{1+y} \quad \Leftrightarrow \quad x=\ln(-1+\sqrt{1+y}).\] L'équation $y=f(x)$ admet donc toujours une unique solution. La fonction $f$ est bijective et 
\[ f^{-1}(y)=\ln(-1+\sqrt{1+y}).\]
\end{sol}















\end{document}
