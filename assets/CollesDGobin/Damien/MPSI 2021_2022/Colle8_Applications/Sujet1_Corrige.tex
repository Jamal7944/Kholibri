%%%%%%%%%%%%%%%%%%%%%%%%%%%%%%%%%%%   Packages de base   %%%%%%%%%%%%%%%%%%%%%%%%%%%%%%%%%%%


% Classe du document (book : chapitre inclus, pratique pour les gros cours (PASS, Tremplin), article (8pt) ou amsart (12pt) : pratique pour les cours sur une seule thématique)
\documentclass[a4paper, 11pt,openany]{article}% openany évite de commencer les chapitres sur une page forcément impaire (évite les pages blanches)

% Encodage, langue et font
\usepackage[utf8]{inputenc} % Pour les accents (conversion entre ces caractères accentués et les commandes d’accentuation)
\usepackage[french]{babel} % Français
\usepackage[T1]{fontenc} % Permet d'afficher et de prendre correctement en charge ces caractères accentués du point de vue du fichier de sortie

% Si on veut faire apparaître le texte avec des caractères plus larges
%\usepackage[lf]{Baskervaldx} % Texte en plus "gras"
%\usepackage[bigdelims,vvarbb]{newtxmath} % Lettre mathématiques en plus "gras"
%\usepackage[cal=boondoxo]{mathalfa} % Style pour les \mathscal
%\renewcommand*\oldstylenums[1]{\textosf{#1}}

% Mise en page
% Version automatique mais problème d'écart entre l'en-tête et le texte.
%\usepackage{fullpage} % Numérotation bas de page (et mise en page pleine).
\usepackage{setspace} % Interligne
\onehalfspacing % Interligne
\usepackage[a4paper]{geometry}% Package pour mise en page
\geometry{hscale=0.8,vscale=0.8,centering,headsep=0.5cm} % Marges, Headsep permet de ne pas coller le texte à l'en-tête

\setlength{\parindent}{0pt} % Supprimer les identations par défaut

\usepackage{fancyhdr} %Package permettant de mettre des en-têtes et des pieds de pages
\pagestyle{fancy}
\fancyhead{} % Enleve ce qui est présent par défaut
\fancyfoot{}% Enleve ce qui est présent par défaut
\usepackage{extramarks} % Pour écrire proprement le chapitre dans l'en-tête
%\renewcommand{\chaptermark}[1]{\markboth{\chaptername\ \thechapter.\ #1}{}} % Redéfinition du chapitre pour écriture dans l'en-tête sous la forme "Chapitre n. Nom du chapitre"
%\renewcommand{\chaptermark}[1]{\markboth{\thechapter.\ #1}{}} % Forme "n. Nom du chapitre"
%\renewcommand{\chaptermark}[1]{\markboth{#1}{}} % Forme "Nom du chapitre"
\renewcommand{\headrulewidth}{0.6pt} % Epaisseur trait en haut
\renewcommand{\footrulewidth}{0.6pt} % Epaisseur trait en bas
\setlength{\headheight}{15pt}
% Placer les informations où on veut (à noter : E: Even page ; O: Odd page ; L: Left field ; C: Center field ; R: Right field ; H: Header ; F: Footer) :
\fancyhead[LO,LE]{\slshape  \textbf{MPSI2}}
\fancyhead[RO,RE]{\slshape \textbf{Corrigé - Colle 8 (Sujet 1)}}
%\fancyhead[CO,CE]{---Draft---}
\fancyfoot[C]{\thepage}
%\fancyfoot[LO, LE] {\slshape Damien GOBIN}
%\fancyfoot[RO, RE] {\slshape Année 2021-2022}

% Couleurs
\usepackage{xcolor}
\definecolor{BleuTresFonce}{rgb}{0.0,0.0,0.250}
%\definecolor{bleu}{rgb}{0.36, 0.54, 0.66}
\definecolor{bleu}{rgb}{0.47, 0.62, 0.8}
%\definecolor{vert}{rgb}{0.33, 0.42, 0.18}
%\definecolor{vert}{rgb}{0.52, 0.73, 0.4}
\definecolor{vert}{rgb}{0.56, 0.74, 0.56}
% Liens hypertextes
\usepackage[colorlinks,final,hyperindex]{hyperref}
\hypersetup{
	pdftex,
	linkcolor=BleuTresFonce,
	citecolor=BleuTresFonce,
	filecolor=BleuTresFonce,
	urlcolor=BleuTresFonce,
	pdftitle=TemplateCours,
	pdfauthor=Damien Gobin,
	pdfsubject=,
	pdfkeywords=
}


% Inclure des figures
\usepackage{graphicx} % Permet d'utiliser includegraphics
\usepackage{pdfpages} % Permet d'insérer des fichiers pdf

% Utiliser des commentaires
\usepackage{comment}
%\excludecomment{sol} %commenter cette ligne si on veut faire apparaitre les solutions d'exercices

%%%%%%%%%%%%%%%%%%%%%%%%%%%%%%%%%%%   Packages mathématiques  %%%%%%%%%%%%%%%%%%%%%%%%%%%%%%%%%%%

% Packages mathématiques de base
%\usepackage{amsfonts} % Permet de taper des ensembles 
\usepackage{amsmath} % Permet de taper des maths (contient cleveref)
\usepackage{amsthm} % Permet de définir une environnement pour les théorèmes
\usepackage{amssymb} % Donne accès a plus de symboles mathématiques (charge de façon automatique amsfonts)
% \usepackage{mathrsfs} % Ecriture ronde type mathcal
%\usepackage{bbold} % Permet d'avoir l'indicatrice mais change les textbb
\usepackage{stmaryrd} % Pour les intervalles entiers [[ ]]
\usepackage{pifont} % Pour avoir accès à plus de caractères

%Utiliser les symboles jeu de cartes
\DeclareSymbolFont{extraup}{U}{zavm}{m}{n}
\DeclareMathSymbol{\varheart}{\mathalpha}{extraup}{86}
\DeclareMathSymbol{\vardiamond}{\mathalpha}{extraup}{87}



% Référence dans le document
\usepackage[noabbrev,capitalize]{cleveref}

% Tableau
\usepackage{tabularx} % Tableau
\usepackage{multirow} % Pour créer des tableaux en subdivisant les lignes
\usepackage{diagbox} %Pour créer des crois dans les tableaux à double entrées
\usepackage{enumitem} %Permet de scinder une énumération et de reprendre au numéro suivant
\usepackage{multicol} % Création de document en colonne avec plusieurs colonnes
\multicolsep=5pt % supprime l'espace vertical

% Modification des itemizes
\setitemize{label=$\bullet$} % Utilise le package enumitem et met des points au lieu des tirets

% Ecrire des algorithmes en "français" (exemple issu du cours d'optimisation 2020/2021)
\usepackage[linesnumbered, french, frenchkw,ruled]{algorithm2e}
% Exemple :
%\begin{center}
%\begin{algorithm}
%\Entree{Un graphe $G = (V,E)$, $|V| = n$, $|E| = m$ et pour chaque arête $e$ de $E$ son poids $c(e)$;}
%\Sortie{Un arbre (ou une forêt) maximal $A = (V,F)$ et de poids minimum ;} 
%Trier et renuméroter les arêtes de $G$ dans l'ordre croissant de leur poids : $c(e_1) \leqslant c(e_2) \leqslant ... \leqslant c(e_m)$ \;
%Poser $F := \emptyset$ et $k = 0$\;
%\Tq{$k<m$ et $|F| < n-1$}{Si $e_{k+1}$ ne forme pas de cycle avec $F$ alors $F := F \cup \{ e_k \}$\;
%$k := k +1 $\;}
%\caption{Algorithme de Kruskal théorique (1956)}
%\end{algorithm}
%\end{center}

% Inclure du code Python avec coloration syntaxique et bloc gris (exemple issu du TP2 d'optimisation 2020/2021)
%\usepackage{xcolor} % Déjà importé plus haut
\usepackage{listings}
\lstset{backgroundcolor=\color{darkWhite},literate={á}{{\'a}}1 {é}{{\'e}}1 {í}{{\'i}}1 {ó}{{\'o}}1 {ú}{{\'u}}1{Á}{{\'A}}1 {É}{{\'E}}1 {Í}{{\'I}}1 {Ó}{{\'O}}1 {Ú}{{\'U}}1{à}{{\`a}}1 {è}{{\`e}}1 {ì}{{\`i}}1 {ò}{{\`o}}1 {ù}{{\`u}}1{À}{{\`A}}1 {È}{{\'E}}1 {Ì}{{\`I}}1 {Ò}{{\`O}}1 {Ù}{{\`U}}1{ä}{{\"a}}1 {ë}{{\"e}}1 {ï}{{\"i}}1 {ö}{{\"o}}1 {ü}{{\"u}}1{Ä}{{\"A}}1 {Ë}{{\"E}}1 {Ï}{{\"I}}1 {Ö}{{\"O}}1 {Ü}{{\"U}}1{â}{{\^a}}1 {ê}{{\^e}}1 {î}{{\^i}}1 {ô}{{\^o}}1 {û}{{\^u}}1{Â}{{\^A}}1 {Ê}{{\^E}}1 {Î}{{\^I}}1 {Ô}{{\^O}}1 {Û}{{\^U}}1{œ}{{\oe}}1 {Œ}{{\OE}}1 {æ}{{\ae}}1 {Æ}{{\AE}}1 {ß}{{\ss}}1{ű}{{\H{u}}}1 {Ű}{{\H{U}}}1 {ő}{{\H{o}}}1 {Ő}{{\H{O}}}1{ç}{{\c c}}1 {Ç}{{\c C}}1 {ø}{{\o}}1 {å}{{\r a}}1 {Å}{{\r A}}1{€}{{\EUR}}1 {£}{{\pounds}}1}
\lstdefinestyle{stylepython}{        language=Python,        basicstyle=\ttfamily,    commentstyle=\color{green},    keywordstyle=\color{blue},    stringstyle=\color{olive},    numberstyle=\tiny,        numbers=left,        stepnumber=1,         numbersep=5pt}
% Exemple :
%\begin{lstlisting}[style=stylepython]
%import networkx as nx #Cette bibliothèque permettra de manipuler des graphes
%import matplotlib.pyplot as plt #Cette bibliothèque permettra de les représenter
%import numpy as np
%\end{lstlisting}

%Package permettant de tracer des graphes, des courbes, etc.. (exemple issu du cours d'optimisation 2020/2021)
\usepackage{tikz}
\usepackage{tikz-cd}
\usepackage{tkz-tab} % Pour les tableaux de signes
%\usetikzlibrary{shapes,backgrounds}
\usetikzlibrary{
	decorations.pathmorphing,
	arrows,
	arrows.meta,
	calc,
	shapes,
	shapes.geometric,
	decorations.pathreplacing,}
\tikzcdset{arrow style=tikz, diagrams={>=stealth}}
\tikzset{>=stealth'}
\tikzstyle{rond}=[draw,circle,thick,fill=white]
% Exemple courbe :
%\begin{center}
%\begin{tikzpicture}
%	\def\shift{.5}
%	\def\xmax{6}
%	\def\ymax{7}
%	\draw[->] (-\shift,0) -- (\xmax+\shift,0) node[below] {$x$}; 
%	\foreach \x in {1,...,\xmax}{ \pgfmathtruncatemacro\xbis{100*\x}
%		\draw (\x,0) -- (\x,-.3*\shift) node[below] {$\scriptstyle{\xbis}$};
%	}
%	% axe ordonnees
%	\draw[->] (0,-\shift) -- (0,\ymax+\shift) node[left] {$y$};  
%	\foreach \y in {1,...,\ymax}{ \pgfmathtruncatemacro\ybis{100*\y}
%		\draw (0,\y) -- (-.3*\shift,\y) node[left] {$\scriptstyle{\ybis}$};
%	}
%	\draw[fill=yellow] (0,0) -- (3,0) -- (0,6) -- cycle;
%	
%	\draw[thick, purple] plot [domain=-.15:6.15, variable=\t] (\t,6-\t);
%	\node[above, text=purple] (A) at (6.9,.1) {$x+y = 600$};
%	%%
%	\draw[thick, purple] plot [domain=-.1:3.1, variable=\t] (\t,6-2*\t);
%	\node[right, text=purple] (B) at (1.1,5) {$2x+y = 600$};
%	%%
%	\draw[thick, blue] plot [domain=-.1:3.85, variable=\t] (\t,6-1.6*\t);
%	%%
%	\draw[fill=blue] (0,6) circle (3pt);
%	\node[right, text=blue] (B) at (0.1,6.3) {Solution optimale};
%\end{tikzpicture}
%\end{center}
% Exemple graphe :
%\begin{center}
%\begin{tikzpicture}[scale=0.9]
%    \draw[thick] (-1.5,0) node[rond] {$1$} coordinate (A) --
%        ++(3,0) node[rond] {$2$} coordinate (B) --
%        ++(-1,-1) node[rond] {$4$}  coordinate (D) --
%        ++(-1,0) node[rond] {$3$}  coordinate (C) --
%        ++(0,-1) node[rond] {$5$}  coordinate (E) --
%        ++(1,0) node[rond] {$6$}  coordinate (F) --
%        ++(1,-1) node[rond] {$8$}  coordinate (H) --
%        ++(-3,0) node[rond] {$7$}  coordinate (G) -- (A)
%        (A) -- (C)
%        (G) -- (E)
%        (D) -- (F)
%        (B) -- (H);
%\end{tikzpicture}
%\end{center}

%%%%%%%%%%%%%%%%%%%%%%%%%%%%%%%%%%%   Environnements  %%%%%%%%%%%%%%%%%%%%%%%%%%%%%%%%%%%

%Différents types à donner aux environnements théorèmes, définitions, etc...
%[theorem] permet de numéroter par rapport à la section en cours
%\renewcommand{\theexem}{\empty{}} permet de ne pas numéroterNe numérote pas les exemples

%\usepackage{thmtools} % Permet de créer un environnement de théorème avec des cadres
%\declaretheorem[thmbox=L]{boxtheorem L}
%\declaretheorem[thmbox=M]{boxtheorem M}
%\declaretheorem[thmbox=S]{boxtheorem S}
%
%%\usepackage[dvipsnames]{xcolor}
%%\declaretheorem[shaded={bgcolor=Lavender,textwidth=12em}]{BoxI}
%\declaretheorem[shaded={rulecolor=black,rulewidth=1pt}]{BoxII}
%
%
%\usepackage[leftmargin=0.1865cm,rightmargin=0.6cm]{thmbox}
% Voici ici pour les options https://ctan.mines-albi.fr/macros/latex/contrib/thmbox/thmbox.pdf

%% Ecrit de cette façon tout le monde a le même style : Titre en droit et texte en italique très léger.
%\newtheorem[L,leftmargin=0.1865cm,rightmargin=0.1865cm,cut=true]{thm}{Théorème}[subsection]
%%\renewcommand{\thetheorem}{\empty{}} 
%\newtheorem[M,leftmargin=0.1865cm,rightmargin=0.1865cm,cut=true]{propo}[thm]{Proposition}
%%\renewcommand{\thepropo}{\empty{}} 
%\newtheorem[M,leftmargin=0.1865cm,rightmargin=0.1865cm,cut=true]{prop}[thm]{Propriété}
%%\renewcommand{\theprop}{\empty{}} 
%\newtheorem[M,leftmargin=0.1865cm,rightmargin=0.1865cm,cut=true]{coro}[thm]{Corollaire}
%\newtheorem[M,leftmargin=0.1865cm,rightmargin=0.1865cm,cut=true]{lem}[thm]{Lemme}
%\newtheorem[M,leftmargin=0.1865cm,rightmargin=0.1865cm,cut=true]{defi}[theorem]{Définition}
%%\renewcommand{\thedefinition}{\empty{}}
%\newtheorem[S,leftmargin=0.1865cm,rightmargin=0.1865cm,cut=true]{exem}{Exemple}
%\renewcommand{\theexem}{\empty{}} 
%\newtheorem[S,leftmargin=0.1865cm,rightmargin=0.1865cm,cut=true]{exo}{Exercice}
%\renewcommand{\theexo}{\empty{}}
%\newtheorem[S,leftmargin=0.1865cm,rightmargin=0.1865cm,cut=true]{sol}{Solution}
%\renewcommand{\thesol}{\empty{}} 
%\newtheorem[S,leftmargin=0.1865cm,rightmargin=0.1865cm,cut=true]{hypo}{Hypothesis}[section]
%\newtheorem[S,leftmargin=0.1865cm,rightmargin=0.1865cm,cut=true]{remark}[thm]{Remarque}%[section]
%\newtheorem{dem}{Démonstration}
%\renewcommand{\thedem}{\empty{}} 
%\newtheorem[S]{nota}{Notation}[section]

\theoremstyle{plain}
\newtheorem{theorem}{Théorème}[subsection]
%\renewcommand{\thetheorem}{\empty{}} 
\newtheorem{propo}[theorem]{Proposition}
%\renewcommand{\thepropo}{\empty{}} 
\newtheorem{prop}[theorem]{Propriété}
%\renewcommand{\theprop}{\empty{}} 
\newtheorem{coro}[theorem]{Corollaire}
\newtheorem{lemma}[theorem]{Lemme}

\theoremstyle{definition}
\newtheorem{definition}[theorem]{Définition}
%\renewcommand{\thedefinition}{\empty{}}
\newtheorem{exem}{Exemple}
\renewcommand{\theexem}{\empty{}}
\newtheorem{cours}{Question de cours}
\renewcommand{\thecours}{\empty{}} 
\newtheorem{exo}{Exercice}
%\renewcommand{\theexo}{\empty{}} 
\newtheorem{sol}{Solution de l'exercice}
%\renewcommand{\thesol}{\empty{}} 
\newtheorem{hypo}{Hypothesis}[section]
\newtheorem{remark}[theorem]{Remarque}%[section]


\theoremstyle{remark}
\newtheorem{dem}{Démonstration}
\renewcommand{\thedem}{\empty{}} 
\newtheorem{nota}{Notation}[section]

%%%%%%%%%%%%%%%%%%%%%%%%%%%%%%%%%%%   Raccourcis   %%%%%%%%%%%%%%%%%%%%%%%%%%%%%%%%%%%
 
% Ensembles
\newcommand{\R}{\mathbb{R}} 
\newcommand{\Q}{\mathbb{Q}}
\newcommand{\C}{\mathbb{C}}
\newcommand{\Z}{\mathbb{Z}}
\newcommand{\K}{\mathbb{K}}
\newcommand{\N}{\mathbb{N}}
\newcommand{\D}{\mathbb{D}}


% Lettres calligraphiques
\newcommand{\calP}{\mathcal{P}} 
\newcommand{\calF}{\mathcal{F}} 
\newcommand{\calD}{\mathcal{D}} 
\newcommand{\calQ}{\mathcal{Q}} 
\newcommand{\calT}{\mathcal{T}} 

% Noyau et image
\newcommand{\im}{\text{Im}} 
\newcommand{\noy}{\text{Ker}} 
\newcommand{\card}{\text{Card}} 

%Epsilon
\newcommand{\vare}{\varepsilon} 


% Fonctions usuelles
%\newcommand{\un}{\mathbb{1}} % Indicatrice en utilisant le package \usepackage{bbold} mais celui modifie les mathbb donc on définit l'indicatrice à la main
\def\un{{\mathchoice {\rm 1\mskip-4mu l} {\rm 1\mskip-4mu l}
{\rm 1\mskip-4.5mu l} {\rm 1\mskip-5mu l}}} % Indicatrice






\title{Corrigé - Colle 8 (Sujet 1)}
\author{MPSI2\\
Année 2021-2022}
%\author{Mathématiques P.A.S.S. 1}
\date{23 novembre 2021}

\begin{document}


   \maketitle
%      \rule{\linewidth}{0.5mm}
      \rule{\linewidth}{0.5mm}
%  \begin{center}
%  %    \rule{\linewidth}{0.5mm}\\[0.4cm]
%       { \huge \bfseries Mathématiques pour le P.A.S.S 1\\[0.4cm] }
%    \rule{\linewidth}{0.5mm}\\[4cm]
%     \end{center}


\begin{cours}
Démontrer que pour trois parties $A$, $B$ et $C$ d'un ensemble $E$,
\[ A \cap(B \cup C) = (A \cap B) \cup (A \cap C).\]
\end{cours}

\begin{exo}
%Difficulté : 1/5
%Chapitre : Applications
On considère la fonction 
$ f : x \mapsto \sqrt{x^2 + x - 2}$.
\begin{enumerate}
\item Déterminer l'ensemble de définition $\calD_f$ de $f$.
\item Déterminer l'image de $f$, i.e. $f(\calD_f)$.
\item L'application $f$ est-elle injective de $\calD_f$ dans $\R$ ?
\end{enumerate}
\end{exo}

\begin{sol}
\begin{enumerate}
\item L'ensemble de définition de $f$ est l'ensemble des points $x$ tels que $x^2 + x -2 \geqslant 0$. Or, $x^2 + x -2$ est un trinôme dont le discriminant est $\Delta = 9$ et dont les racines sont donc $x_1 = -2$ et $x_2 =1$. De plus, $x^2 +x -2$ est positif sauf entre les racines $x_1 = -2$ et $x_2= 1$. L'ensemble de définition de $f$ est donc $\calD_f = ]-\infty,-2] \cup [1, +\infty[$.
\item L'image de $f$ est donnée par $f(\calD_f) = [0,+\infty[$ (car le trinôme s'annule en $-2$ et en $1$, $x^2+x-2 \to + \infty$ lorsque $x \to + \infty$ et la fonction racine est continue sur $\R^+$).
\item Non, l'application $f$ n'est pas injective de $\calD_f$ dans $\R$. En effet, $f(-2) =  f(1)= 0$.
\end{enumerate}
\end{sol}



\begin{exo}
%Difficulté : 1/5
%Chapitre : Applications
Soient
\[  \begin{array}{ccccc}
f & : & \R & \to & \R \\
 & & x & \mapsto & \sqrt{x^2+1} - x
\end{array} \quad \text{et} \quad \begin{array}{ccccc}
g & : & \R^{\star} & \to & \R \\
 & & x& \mapsto & \frac{1-x^2}{2x}
\end{array}.\]
\begin{enumerate}
\item Montrer que tout réel $x$, $f(x) > 0$.
\item Montrer que la composée $f \circ g$ est bien définie sur $\R_+^{\star}$ et calculer $(f \circ g)(x)$ pour tout $x > 0$.
\item De même, montrer que la composée $g \circ f$ est bien définie sur $\R$ et calculer $(f \circ g)(x)$ pour tout $x \in \R$.
\item Que peut-on en conclure ?
\end{enumerate}
\end{exo}

\begin{sol}
\begin{enumerate}
\item On a
\[ f(x) > 0 \quad \Leftrightarrow \quad \sqrt{x^2+1} > x \quad \Leftrightarrow \quad  x^2 + 1 > x^2 \quad \Leftrightarrow \quad 1 > 0\]
ce qui est bien sûr vraie pour tout $x \in \R$. Ainsi, $f(x) > 0$ pour tout $x \in \R$.
\item $f\circ g$ est bien définie lorsque pour tout $x \in \R_+^{\star}$ puisque $f$ est définie sur $\R$ et $g$ est bien définie sur $\R^{\star}$ et donc sur $\R_+^{\star}$. De plus,
\[ (f \circ g)(x) = f(g(x)) = \sqrt{\left(\frac{1-x^2}{2x} \right)^2+1} - \frac{1-x^2}{2x} = \sqrt{\frac{1-2x^2+x^4}{4x^2} +1} - \frac{1-x^2}{2x}  .\]
Après calcul, on arrive à
\[ (f \circ g)(x)  = \sqrt{\frac{1+2x^2+x^4}{4x^2}} - \frac{1-x^2}{2x} =\sqrt{\frac{(1+x^2)^2}{4x^2}} - \frac{1-x^2}{2x}.\]
Or, $\sqrt{(1+x^2)^2} = |(1 +x^2)^2| = (1+x^2)^2$ car $(1+x^2)^2 \geqslant 0$ et de plus, puisque $x \in \R_+^{\star}$, $\sqrt{4x^2} = |2x| = 2x$. Ainsi, on a finalement,
\[ (f \circ g)(x)  =\frac{1+x^2}{2x} - \frac{1-x^2}{2x} = x.\]
\item La composée $ g \circ f$ est définie sur $\R$ car $f$ est définie sur $\R$, $f(x) > 0$ pour tout $x \in \R$ et $g$ est bien définie sur $\R^{\star}$ et donc sur $\R_+^{\star}$. De plus,
\[ (g \circ f)(x) = g(f(x)) = \frac{1-(\sqrt{x^2+1} - x)^2}{2(\sqrt{x^2+1} - x)} =  \frac{1-(x^2+1 -2 x\sqrt{x^2 +1} +x^2)}{2(\sqrt{x^2+1} - x)} .\]
Après calcul, on arrive à
\[ (g \circ f)(x) =  \frac{1-x^2-1 +2 x\sqrt{x^2 +1} -x^2}{2(\sqrt{x^2+1} - x)} = \frac{2x (\sqrt{x^2 +1}-x) }{2(\sqrt{x^2+1} - x)} =x.\]
\item On a montré que
\[  \begin{array}{ccccc}
f\circ g & : & \R_+^{\star} & \to & \R_+^{\star}  \\
 & & x & \mapsto & x
\end{array} \quad \quad \text{et} \quad \quad \begin{array}{ccccc}
g\circ f & : & \R & \to & \R \\
 & & x & \mapsto & x
\end{array} .\]
Autrement dit,
\[ 
f\circ g = Id_{\R_+^{\star}} \quad \quad \text{et} \quad \quad g \circ f = Id_{\R} .\]
On peut en déduire que l'on a en réalité montré que $f$ et $g$ sont des applications réciproques l'une de l'autre.
\end{enumerate}

\end{sol}

\begin{exo}
%Difficulté : 2/5
%Chapitre : Applications
On considère la fonction
$\displaystyle{h : x \mapsto \frac{2x+1}{x+2}}$.
\begin{enumerate}
\item Déterminer l'ensemble de définition $\calD_h$ de $h$.
\item Déterminer $\im(h) = h(\calD_h)$ et montrer que $h$ est bijective de $\calD_h$ dans $\im(h)$.
\item Donner son application réciproque.
\end{enumerate}
\end{exo}

\begin{sol}
\begin{enumerate}
\item $\calD_h = \R \setminus \{-2\}$ car c'est le seul endroit où le dénominateur s'annule (et le numérateur ne s'y annule pas).
\item Si $y \in \im(h)$, alors il existe $x \in \R \setminus \{-2\}$ tel que $h(x) = y$. Or,
\[h(x) = y \quad \Leftrightarrow \quad \frac{2x+1}{x+2} = y \quad \Leftrightarrow \quad 2x+1 = y(x+2) \quad \Leftrightarrow \quad (2-y)x = 2y-1.\]
Ainsi, si $y \neq 2$, $x = \frac{2y-1}{2-y}$ est un antécédent de $y$. Ceci montre que $\R \setminus \{2\} \subset \im(h)$. Afin de montrer que c'est une égalité, montrons à présent que $2$ n'a pas d'antécédent par $f$. Supposons par l'absurde qu'il existe $x$ tel que $h(x) = 2$. Alors,
\[ h(x) = 2 \quad \Leftrightarrow \quad \frac{2x+1}{x+2} =2 \quad \Leftrightarrow \quad 2x+1 = 2(x+2) \quad \Leftrightarrow \quad 1 = 4\]
ce qui est absurde. On a ainsi montré que $\im(h) = \R \setminus \{2\}$.\\
Montrons à présent que $h$ est bijective de $\calD_h$ dans $\im(h)$. Cela revient à montré que $h$ est injective. Soit $(x,y) \in (\R \setminus \{-2\})^2$ tel que $h(x) = h(y)$. Alors,
\[h(x) = h(y) \Leftrightarrow \quad \frac{2x+1}{x+2} = \frac{2y+1}{y+2} \quad \Leftrightarrow \quad (2x+1)(y+2) = (2y+1)(x+2).\]
Ainsi,
\[ h(x) = h(y) \quad \Leftrightarrow \quad 2xy + 4x + y + 2 = 2xy + 4y + x + 2 \quad \Leftrightarrow \quad   3(x-y)  = 0 .\]
On en déduit que $h(x) = h(y)$ si et seulement si $x =y$. $h$ est donc bijective de $\calD_h$ dans $\im(h)$.
\item Nous avons déjà montré que pour tout $y \neq 2$,
\[h(x) = y \quad \Leftrightarrow \quad x = \frac{2y-1}{2-y}.\]
L'application réciproque de $h$ est donc donnée par
\[  \begin{array}{ccccc}
h^{-1} & : & \R \setminus \{2\} & \to & \R \setminus \{-2\} \\
 & & x & \mapsto & \frac{2x-1}{2-x}
\end{array}.\]
\end{enumerate}
\end{sol}

\begin{exo}
%Difficulté : 2/5
%Chapitre : Applications
Soit $f :E \to F$ et $A \subset F$. Montrer que $ f^{-1}(f(A)) \subset A$. Trouver un contre-exemple pour l'autre inclusion. Que peut-on dire si $f$ est de plus surjective ?
\end{exo}

\begin{sol}
\begin{enumerate}
 \item Si $y\in f(f^{-1}(A))$ alors il existe $x\in f^{-1}(A)$ tel que $y=f(x)$. $x\in f^{-1}(A)$ donc $f(x)\in A$ d'où $y\in A$ donc on a bien $f(f^{-1}(A)) \subset A$. 
 \item Donnons un contre-exemple lorsque $f$ n'est pas surjective. On peut choisir $f\colon \mathbb R \to \mathbb R$ définie par $f(x)=\cos(x)$. Pour $A=[1,+\infty[$ on a $f^{-1}(A)=\{2k\pi, \ k\in \mathbb Z\}$ et $f(f^{-1}(A))=\{1\}$. Donc $A$ n'est pas inclu dans $f(f^{-1}(A))$.
 \item Supposons que $f$ est surjective. Soit $y\in A$. Alors, comme $y\in F$, il existe $x\in E$ tel que $y=f(x)$ par surjectivité de $f$. Pour montrer que $y\in f(f^{-1}(A))$ il reste à prouver que $x\in f^{-1}(A)$ c'est-à-dire que $f(x)\in A$. Or $f(x)=y \in A$ donc on a bien $A\subset f(f^{-1}(A))$. Ainsi, lorsque $f$ est surjective on a donc $f(f^{-1}(A))=A$.
\end{enumerate}
\end{sol}
   


\begin{exo}
%Difficulté : 2/5
%Chapitre : Ensembles
Pour tout $m \in \R$, on définit la droite $\calD_m$ par l'équation
\[ \calD_m \,: \, 12mx - 9y = 3m + 6.\]
Montrer que toutes les droites  $\calD_m$ sont concourantes en un unique point.
\end{exo}

\begin{sol}
On procède par analyse-synthèse.
\begin{itemize}
\item \textbf{Analyse :} Soient $m \neq m'$ deux valeurs du paramètre. Un point appartient aux deux droites $\mathcal{D}_m$ et $\mathcal{D}_{m'}$ si et seulement il vérifie les deux conditions simultanément. Autrement, dit
\[(x,y) \in \mathcal{D}_m \cap \mathcal{D}_{m'} \quad \Leftrightarrow \quad  \left\{
    \begin{array}{ccl}
        12mx-9y & = & 3m +6 \\
       12m'x-9y & = & 3m' +6 
    \end{array}
\right. \quad \Leftrightarrow \quad  \left\{
    \begin{array}{ccl}
        12mx-9y & = & 3m +6 \\
       12(m-m')x & = & 3 (m-m')
    \end{array}
\right..\]
Avec la deuxième ligne on obtient
\[12(m-m')x = 3 (m-m') \quad \Leftrightarrow \quad x = \frac{3}{12} = \frac{1}{4}\]
et en injectant dans la première ligne on arrive à
\[ 12m \frac{1}{4} - 9y = 3m + 6 \quad \Leftrightarrow \quad y = -\frac{6}{9} = -\frac{2}{3}.\]
Ainsi, si $(x,y)$ appartient à toutes les droites $\mathcal{D}_m$ alors $(x,y) = \left( \frac{1}{4} , -\frac{2}{3} \right)$.
\item \textbf{Synthèse.} Vérifions que si $(x,y) = \left( \frac{1}{4} , -\frac{2}{3} \right)$ alors ce point appartient à toutes les droites $\mathcal{D}_m$ :
\[ 12mx-9y  = 12 m \frac{1}{4} + 9 \frac{2}{3} = 3m +6.\]
Donc $(x,y) \in \mathcal{D}_m$.
\end{itemize}
On a donc bien montré que toutes les droites $\mathcal{D}_m$ sont concourantes au point $ \left( \frac{1}{4} , -\frac{2}{3} \right)$.
\end{sol}

















\end{document}
