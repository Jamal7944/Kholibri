%%%%%%%%%%%%%%%%%%%%%%%%%%%%%%%%%%%   Packages de base   %%%%%%%%%%%%%%%%%%%%%%%%%%%%%%%%%%%


% Classe du document (book : chapitre inclus, pratique pour les gros cours (PASS, Tremplin), article (8pt) ou amsart (12pt) : pratique pour les cours sur une seule thématique)
\documentclass[a4paper, 11pt,openany]{article}% openany évite de commencer les chapitres sur une page forcément impaire (évite les pages blanches)

% Encodage, langue et font
\usepackage[utf8]{inputenc} % Pour les accents (conversion entre ces caractères accentués et les commandes d’accentuation)
\usepackage[french]{babel} % Français
\usepackage[T1]{fontenc} % Permet d'afficher et de prendre correctement en charge ces caractères accentués du point de vue du fichier de sortie

% Si on veut faire apparaître le texte avec des caractères plus larges
%\usepackage[lf]{Baskervaldx} % Texte en plus "gras"
%\usepackage[bigdelims,vvarbb]{newtxmath} % Lettre mathématiques en plus "gras"
%\usepackage[cal=boondoxo]{mathalfa} % Style pour les \mathscal
%\renewcommand*\oldstylenums[1]{\textosf{#1}}

% Mise en page
% Version automatique mais problème d'écart entre l'en-tête et le texte.
%\usepackage{fullpage} % Numérotation bas de page (et mise en page pleine).
\usepackage{setspace} % Interligne
\onehalfspacing % Interligne
\usepackage[a4paper]{geometry}% Package pour mise en page
\geometry{hscale=0.8,vscale=0.8,centering,headsep=0.5cm} % Marges, Headsep permet de ne pas coller le texte à l'en-tête

\setlength{\parindent}{0pt} % Supprimer les identations par défaut

\usepackage{fancyhdr} %Package permettant de mettre des en-têtes et des pieds de pages
\pagestyle{fancy}
\fancyhead{} % Enleve ce qui est présent par défaut
\fancyfoot{}% Enleve ce qui est présent par défaut
\usepackage{extramarks} % Pour écrire proprement le chapitre dans l'en-tête
%\renewcommand{\chaptermark}[1]{\markboth{\chaptername\ \thechapter.\ #1}{}} % Redéfinition du chapitre pour écriture dans l'en-tête sous la forme "Chapitre n. Nom du chapitre"
%\renewcommand{\chaptermark}[1]{\markboth{\thechapter.\ #1}{}} % Forme "n. Nom du chapitre"
%\renewcommand{\chaptermark}[1]{\markboth{#1}{}} % Forme "Nom du chapitre"
\renewcommand{\headrulewidth}{0.6pt} % Epaisseur trait en haut
\renewcommand{\footrulewidth}{0.6pt} % Epaisseur trait en bas
\setlength{\headheight}{15pt}
% Placer les informations où on veut (à noter : E: Even page ; O: Odd page ; L: Left field ; C: Center field ; R: Right field ; H: Header ; F: Footer) :
\fancyhead[LO,LE]{\slshape  \textbf{MPSI2}}
\fancyhead[RO,RE]{\slshape \textbf{Corrigé - Colle 6 (Sujet 2)}}
%\fancyhead[CO,CE]{---Draft---}
\fancyfoot[C]{\thepage}
%\fancyfoot[LO, LE] {\slshape Damien GOBIN}
%\fancyfoot[RO, RE] {\slshape Année 2021-2022}

% Couleurs
\usepackage{xcolor}
\definecolor{BleuTresFonce}{rgb}{0.0,0.0,0.250}
%\definecolor{bleu}{rgb}{0.36, 0.54, 0.66}
\definecolor{bleu}{rgb}{0.47, 0.62, 0.8}
%\definecolor{vert}{rgb}{0.33, 0.42, 0.18}
%\definecolor{vert}{rgb}{0.52, 0.73, 0.4}
\definecolor{vert}{rgb}{0.56, 0.74, 0.56}
% Liens hypertextes
\usepackage[colorlinks,final,hyperindex]{hyperref}
\hypersetup{
	pdftex,
	linkcolor=BleuTresFonce,
	citecolor=BleuTresFonce,
	filecolor=BleuTresFonce,
	urlcolor=BleuTresFonce,
	pdftitle=TemplateCours,
	pdfauthor=Damien Gobin,
	pdfsubject=,
	pdfkeywords=
}


% Inclure des figures
\usepackage{graphicx} % Permet d'utiliser includegraphics
\usepackage{pdfpages} % Permet d'insérer des fichiers pdf

% Utiliser des commentaires
\usepackage{comment}
%\excludecomment{sol} %commenter cette ligne si on veut faire apparaitre les solutions d'exercices

%%%%%%%%%%%%%%%%%%%%%%%%%%%%%%%%%%%   Packages mathématiques  %%%%%%%%%%%%%%%%%%%%%%%%%%%%%%%%%%%

% Packages mathématiques de base
%\usepackage{amsfonts} % Permet de taper des ensembles 
\usepackage{amsmath} % Permet de taper des maths (contient cleveref)
\usepackage{amsthm} % Permet de définir une environnement pour les théorèmes
\usepackage{amssymb} % Donne accès a plus de symboles mathématiques (charge de façon automatique amsfonts)
% \usepackage{mathrsfs} % Ecriture ronde type mathcal
%\usepackage{bbold} % Permet d'avoir l'indicatrice mais change les textbb
\usepackage{stmaryrd} % Pour les intervalles entiers [[ ]]
\usepackage{pifont} % Pour avoir accès à plus de caractères

%Utiliser les symboles jeu de cartes
\DeclareSymbolFont{extraup}{U}{zavm}{m}{n}
\DeclareMathSymbol{\varheart}{\mathalpha}{extraup}{86}
\DeclareMathSymbol{\vardiamond}{\mathalpha}{extraup}{87}



% Référence dans le document
\usepackage[noabbrev,capitalize]{cleveref}

% Tableau
\usepackage{tabularx} % Tableau
\usepackage{multirow} % Pour créer des tableaux en subdivisant les lignes
\usepackage{diagbox} %Pour créer des crois dans les tableaux à double entrées
\usepackage{enumitem} %Permet de scinder une énumération et de reprendre au numéro suivant
\usepackage{multicol} % Création de document en colonne avec plusieurs colonnes
\multicolsep=5pt % supprime l'espace vertical

% Modification des itemizes
\setitemize{label=$\bullet$} % Utilise le package enumitem et met des points au lieu des tirets

% Ecrire des algorithmes en "français" (exemple issu du cours d'optimisation 2020/2021)
\usepackage[linesnumbered, french, frenchkw,ruled]{algorithm2e}
% Exemple :
%\begin{center}
%\begin{algorithm}
%\Entree{Un graphe $G = (V,E)$, $|V| = n$, $|E| = m$ et pour chaque arête $e$ de $E$ son poids $c(e)$;}
%\Sortie{Un arbre (ou une forêt) maximal $A = (V,F)$ et de poids minimum ;} 
%Trier et renuméroter les arêtes de $G$ dans l'ordre croissant de leur poids : $c(e_1) \leqslant c(e_2) \leqslant ... \leqslant c(e_m)$ \;
%Poser $F := \emptyset$ et $k = 0$\;
%\Tq{$k<m$ et $|F| < n-1$}{Si $e_{k+1}$ ne forme pas de cycle avec $F$ alors $F := F \cup \{ e_k \}$\;
%$k := k +1 $\;}
%\caption{Algorithme de Kruskal théorique (1956)}
%\end{algorithm}
%\end{center}

% Inclure du code Python avec coloration syntaxique et bloc gris (exemple issu du TP2 d'optimisation 2020/2021)
%\usepackage{xcolor} % Déjà importé plus haut
\usepackage{listings}
\lstset{backgroundcolor=\color{darkWhite},literate={á}{{\'a}}1 {é}{{\'e}}1 {í}{{\'i}}1 {ó}{{\'o}}1 {ú}{{\'u}}1{Á}{{\'A}}1 {É}{{\'E}}1 {Í}{{\'I}}1 {Ó}{{\'O}}1 {Ú}{{\'U}}1{à}{{\`a}}1 {è}{{\`e}}1 {ì}{{\`i}}1 {ò}{{\`o}}1 {ù}{{\`u}}1{À}{{\`A}}1 {È}{{\'E}}1 {Ì}{{\`I}}1 {Ò}{{\`O}}1 {Ù}{{\`U}}1{ä}{{\"a}}1 {ë}{{\"e}}1 {ï}{{\"i}}1 {ö}{{\"o}}1 {ü}{{\"u}}1{Ä}{{\"A}}1 {Ë}{{\"E}}1 {Ï}{{\"I}}1 {Ö}{{\"O}}1 {Ü}{{\"U}}1{â}{{\^a}}1 {ê}{{\^e}}1 {î}{{\^i}}1 {ô}{{\^o}}1 {û}{{\^u}}1{Â}{{\^A}}1 {Ê}{{\^E}}1 {Î}{{\^I}}1 {Ô}{{\^O}}1 {Û}{{\^U}}1{œ}{{\oe}}1 {Œ}{{\OE}}1 {æ}{{\ae}}1 {Æ}{{\AE}}1 {ß}{{\ss}}1{ű}{{\H{u}}}1 {Ű}{{\H{U}}}1 {ő}{{\H{o}}}1 {Ő}{{\H{O}}}1{ç}{{\c c}}1 {Ç}{{\c C}}1 {ø}{{\o}}1 {å}{{\r a}}1 {Å}{{\r A}}1{€}{{\EUR}}1 {£}{{\pounds}}1}
\lstdefinestyle{stylepython}{        language=Python,        basicstyle=\ttfamily,    commentstyle=\color{green},    keywordstyle=\color{blue},    stringstyle=\color{olive},    numberstyle=\tiny,        numbers=left,        stepnumber=1,         numbersep=5pt}
% Exemple :
%\begin{lstlisting}[style=stylepython]
%import networkx as nx #Cette bibliothèque permettra de manipuler des graphes
%import matplotlib.pyplot as plt #Cette bibliothèque permettra de les représenter
%import numpy as np
%\end{lstlisting}

%Package permettant de tracer des graphes, des courbes, etc.. (exemple issu du cours d'optimisation 2020/2021)
\usepackage{tikz}
\usepackage{tikz-cd}
%\usetikzlibrary{shapes,backgrounds}
\usetikzlibrary{
	decorations.pathmorphing,
	arrows,
	arrows.meta,
	calc,
	shapes,
	shapes.geometric,
	decorations.pathreplacing,}
\tikzcdset{arrow style=tikz, diagrams={>=stealth}}
\tikzset{>=stealth'}
\tikzstyle{rond}=[draw,circle,thick,fill=white]
% Exemple courbe :
%\begin{center}
%\begin{tikzpicture}
%	\def\shift{.5}
%	\def\xmax{6}
%	\def\ymax{7}
%	\draw[->] (-\shift,0) -- (\xmax+\shift,0) node[below] {$x$}; 
%	\foreach \x in {1,...,\xmax}{ \pgfmathtruncatemacro\xbis{100*\x}
%		\draw (\x,0) -- (\x,-.3*\shift) node[below] {$\scriptstyle{\xbis}$};
%	}
%	% axe ordonnees
%	\draw[->] (0,-\shift) -- (0,\ymax+\shift) node[left] {$y$};  
%	\foreach \y in {1,...,\ymax}{ \pgfmathtruncatemacro\ybis{100*\y}
%		\draw (0,\y) -- (-.3*\shift,\y) node[left] {$\scriptstyle{\ybis}$};
%	}
%	\draw[fill=yellow] (0,0) -- (3,0) -- (0,6) -- cycle;
%	
%	\draw[thick, purple] plot [domain=-.15:6.15, variable=\t] (\t,6-\t);
%	\node[above, text=purple] (A) at (6.9,.1) {$x+y = 600$};
%	%%
%	\draw[thick, purple] plot [domain=-.1:3.1, variable=\t] (\t,6-2*\t);
%	\node[right, text=purple] (B) at (1.1,5) {$2x+y = 600$};
%	%%
%	\draw[thick, blue] plot [domain=-.1:3.85, variable=\t] (\t,6-1.6*\t);
%	%%
%	\draw[fill=blue] (0,6) circle (3pt);
%	\node[right, text=blue] (B) at (0.1,6.3) {Solution optimale};
%\end{tikzpicture}
%\end{center}
% Exemple graphe :
%\begin{center}
%\begin{tikzpicture}[scale=0.9]
%    \draw[thick] (-1.5,0) node[rond] {$1$} coordinate (A) --
%        ++(3,0) node[rond] {$2$} coordinate (B) --
%        ++(-1,-1) node[rond] {$4$}  coordinate (D) --
%        ++(-1,0) node[rond] {$3$}  coordinate (C) --
%        ++(0,-1) node[rond] {$5$}  coordinate (E) --
%        ++(1,0) node[rond] {$6$}  coordinate (F) --
%        ++(1,-1) node[rond] {$8$}  coordinate (H) --
%        ++(-3,0) node[rond] {$7$}  coordinate (G) -- (A)
%        (A) -- (C)
%        (G) -- (E)
%        (D) -- (F)
%        (B) -- (H);
%\end{tikzpicture}
%\end{center}

%%%%%%%%%%%%%%%%%%%%%%%%%%%%%%%%%%%   Environnements  %%%%%%%%%%%%%%%%%%%%%%%%%%%%%%%%%%%

%Différents types à donner aux environnements théorèmes, définitions, etc...
%[theorem] permet de numéroter par rapport à la section en cours
%\renewcommand{\theexem}{\empty{}} permet de ne pas numéroterNe numérote pas les exemples

%\usepackage{thmtools} % Permet de créer un environnement de théorème avec des cadres
%\declaretheorem[thmbox=L]{boxtheorem L}
%\declaretheorem[thmbox=M]{boxtheorem M}
%\declaretheorem[thmbox=S]{boxtheorem S}
%
%%\usepackage[dvipsnames]{xcolor}
%%\declaretheorem[shaded={bgcolor=Lavender,textwidth=12em}]{BoxI}
%\declaretheorem[shaded={rulecolor=black,rulewidth=1pt}]{BoxII}
%
%
%\usepackage[leftmargin=0.1865cm,rightmargin=0.6cm]{thmbox}
% Voici ici pour les options https://ctan.mines-albi.fr/macros/latex/contrib/thmbox/thmbox.pdf

%% Ecrit de cette façon tout le monde a le même style : Titre en droit et texte en italique très léger.
%\newtheorem[L,leftmargin=0.1865cm,rightmargin=0.1865cm,cut=true]{thm}{Théorème}[subsection]
%%\renewcommand{\thetheorem}{\empty{}} 
%\newtheorem[M,leftmargin=0.1865cm,rightmargin=0.1865cm,cut=true]{propo}[thm]{Proposition}
%%\renewcommand{\thepropo}{\empty{}} 
%\newtheorem[M,leftmargin=0.1865cm,rightmargin=0.1865cm,cut=true]{prop}[thm]{Propriété}
%%\renewcommand{\theprop}{\empty{}} 
%\newtheorem[M,leftmargin=0.1865cm,rightmargin=0.1865cm,cut=true]{coro}[thm]{Corollaire}
%\newtheorem[M,leftmargin=0.1865cm,rightmargin=0.1865cm,cut=true]{lem}[thm]{Lemme}
%\newtheorem[M,leftmargin=0.1865cm,rightmargin=0.1865cm,cut=true]{defi}[theorem]{Définition}
%%\renewcommand{\thedefinition}{\empty{}}
%\newtheorem[S,leftmargin=0.1865cm,rightmargin=0.1865cm,cut=true]{exem}{Exemple}
%\renewcommand{\theexem}{\empty{}} 
%\newtheorem[S,leftmargin=0.1865cm,rightmargin=0.1865cm,cut=true]{exo}{Exercice}
%\renewcommand{\theexo}{\empty{}}
%\newtheorem[S,leftmargin=0.1865cm,rightmargin=0.1865cm,cut=true]{sol}{Solution}
%\renewcommand{\thesol}{\empty{}} 
%\newtheorem[S,leftmargin=0.1865cm,rightmargin=0.1865cm,cut=true]{hypo}{Hypothesis}[section]
%\newtheorem[S,leftmargin=0.1865cm,rightmargin=0.1865cm,cut=true]{remark}[thm]{Remarque}%[section]
%\newtheorem{dem}{Démonstration}
%\renewcommand{\thedem}{\empty{}} 
%\newtheorem[S]{nota}{Notation}[section]

\theoremstyle{plain}
\newtheorem{theorem}{Théorème}[subsection]
%\renewcommand{\thetheorem}{\empty{}} 
\newtheorem{propo}[theorem]{Proposition}
%\renewcommand{\thepropo}{\empty{}} 
\newtheorem{prop}[theorem]{Propriété}
%\renewcommand{\theprop}{\empty{}} 
\newtheorem{coro}[theorem]{Corollaire}
\newtheorem{lemma}[theorem]{Lemme}

\theoremstyle{definition}
\newtheorem{definition}[theorem]{Définition}
%\renewcommand{\thedefinition}{\empty{}}
\newtheorem{exem}{Exemple}
\renewcommand{\theexem}{\empty{}} 
\newtheorem{cours}{Question de cours}
\renewcommand{\thecours}{\empty{}} 
\newtheorem{exo}{Exercice}
%\renewcommand{\theexo}{\empty{}} 
\newtheorem{sol}{Solution de l'exercice}
%\renewcommand{\thesol}{\empty{}} 
\newtheorem{hypo}{Hypothesis}[section]
\newtheorem{remark}[theorem]{Remarque}%[section]


\theoremstyle{remark}
\newtheorem{dem}{Démonstration}
\renewcommand{\thedem}{\empty{}} 
\newtheorem{nota}{Notation}[section]

%%%%%%%%%%%%%%%%%%%%%%%%%%%%%%%%%%%   Raccourcis   %%%%%%%%%%%%%%%%%%%%%%%%%%%%%%%%%%%
 
% Ensembles
\newcommand{\R}{\mathbb{R}} 
\newcommand{\Q}{\mathbb{Q}}
\newcommand{\C}{\mathbb{C}}
\newcommand{\Z}{\mathbb{Z}}
\newcommand{\K}{\mathbb{K}}
\newcommand{\N}{\mathbb{N}}
\newcommand{\D}{\mathbb{D}}


% Lettres calligraphiques
\newcommand{\calP}{\mathcal{P}} 
\newcommand{\calF}{\mathcal{F}} 
\newcommand{\calD}{\mathcal{D}} 
\newcommand{\calQ}{\mathcal{Q}} 
\newcommand{\calT}{\mathcal{T}} 

% Noyau et image
\newcommand{\im}{\text{Im}} 
\newcommand{\noy}{\text{Ker}} 
\newcommand{\card}{\text{Card}} 

%Epsilon
\newcommand{\vare}{\varepsilon} 


% Fonctions usuelles
%\newcommand{\un}{\mathbb{1}} % Indicatrice en utilisant le package \usepackage{bbold} mais celui modifie les mathbb donc on définit l'indicatrice à la main
\def\un{{\mathchoice {\rm 1\mskip-4mu l} {\rm 1\mskip-4mu l}
{\rm 1\mskip-4.5mu l} {\rm 1\mskip-5mu l}}} % Indicatrice


\title{Corrigé - Colle 6 (Sujet 2)}
\author{MPSI2\\
Année 2021-2022}
%\author{Mathématiques P.A.S.S. 1}
\date{9 novembre 2021}

\begin{document}


   \maketitle
%      \rule{\linewidth}{0.5mm}
      \rule{\linewidth}{0.5mm}
%  \begin{center}
%  %    \rule{\linewidth}{0.5mm}\\[0.4cm]
%       { \huge \bfseries Mathématiques pour le P.A.S.S 1\\[0.4cm] }
%    \rule{\linewidth}{0.5mm}\\[4cm]
%     \end{center}


%\begin{cours}
%Que peut-on dire de l'intégrale d'une fonction paire sur un intervalle symétrique par rapport à $0$. Démontrer.
%\end{cours}
%
%\begin{exo}
%%Difficulté : 1/5
%Chapitre : Intégration
%Donner une primitive des fonctions suivantes :
%\[ f(x) = \frac{8x^2}{(x^3+2)^3}\quad ; \quad g(x) =  3x\sqrt{1 - 2x^2}\]
%\[ h(x) = \sin(x)^2\cos(x)\quad \text{et} \quad i(x) = \frac{e^x}{1+e^x}.\]
%\end{exo}

\begin{sol}
\begin{enumerate}
\item On remarque qu'en posant $u(x)=x^3+2$, on
a $u'(x)= 3\,x^2$, de sorte que $f(x)=\frac{8}{3}\,\frac{u'(x)}{u^{3}(x)}$. On a donc 
\[
 f(x)=
\frac{8}{3}\,
u'(x)\,u^{-3}(x)=\frac{d}{dx} \left(\frac{8}{3}\,\left(-\frac{1}{2}\right)\,u^{-2}(x)\right)
=\frac{d}{dx} \left(-\frac{4}{3}\,(x^3+2)^{-2}\right)=\frac{d}{dx} \left(
-\frac{4}{3\,(x^3+2)^2}\right).
\] 
\item On remarque qu'en
posant $u(x)=1-2\,x^2$, on a $u'(x)=- 4\,x$, de sorte que $g(x)=-\frac{3}{4}\,u'(x)\,\sqrt{u(x)}$. On a alors
\[
g(x)=-\frac{3}{4}\,  u'(x)
u^{\frac{1}{2}}(x)=\frac{d}{dx} \left(-\frac{1}{2}\,u^{\frac{3}{2}}(x)\right)=
\frac{d}{dx} \left(-\frac{1}{2}\,(1-2\,x^2)^{\frac{3}{2}}\right).
\] 
\item Pour $h$, on pose $u(x)=\sin(x)$, on alors $u'(x)=\cos(x)$, d'où $h(x)=u'(x)\,u^2(x)$. On a donc
\[
h(x)= u'(x)\,u^2(x)= \frac{d}{dx}\left(\frac{1}{3}\, u^3(x)\right)= \frac{d}{dx}\left(\frac{1}{3}\,\sin^3(x)\right).
\]
\item On remarque qu'en posant $u(x)=e^{x}+1$, on a $u'(x)=e^{x}$, de sorte que $i(x)=\frac{u'(x)}{u(x)}$. On a donc
\begin{align*}
 i(x)\, &=\frac{u'(x)}{u(x)}= \frac{d}{dx}\left( \ln |u(x)|\right)=\frac{d}{dx}\left(\ln(e^{x}+1)\right)=\frac{d}{dx}\left(\ln (e^{x}+1)\right).
\end{align*}
\end{enumerate}
\end{sol}



%\begin{exo}
%%Difficulté : 1/5
%Chapitre : Intégration
%Donner une primitive des fonctions suivantes :
%\[ f(x) = \frac{x}{\cos(x)^2}\quad ; \quad g(x) = x^2 e^{-3x}\quad \text{et} \quad h(x) = \sin(x)e^{2x}.\]
%\end{exo}
%
\begin{sol}
\begin{enumerate}
\item On remarque que $ \frac{1}{\cos^2(x)}=1+\tan^2(x)=\tan'(x)$, ce qui conduit à faire l'i.p.p. suivante
:\begin{align*}\int \frac{x} {\cos^2(x)}\, dx &= \int u(x)\, v'(x)\, dx \quad \text{où}\quad \begin{cases}
 u(x)=x\,,\;\text{donc}\; u'(x)=1\\
 v(x)=\tan(x)\,,\;\text{donc}\;v'(x)=\frac{1}{\cos^2(x)}\end{cases}\\
&=u(x)\,v(x)-\int u'(x)\, v(x)\, dx\\
&= x\, \tan(x)-\int \tan(x)\, dx=x\, \tan(x)+\int\frac{-\sin(x)}{\cos(x)}\,dx\\
&=x\, \tan(x)+\ln |\cos(x)|+C.
\end{align*}
\item On procède à deux i.p.p. successives où on dérive la partie polynomiale à chaque
étape.
\begin{align*}
\int x^2\,e^{-3x}\, dx &= \int u(x)\, v'(x)\, dx \quad \text{où}\quad \begin{cases}
 u(x)=x^2\,,\;\text{donc}\; u'(x)=2\,x\\
 v(x)=-\frac{1}{3}\,e^{-3x}\,,\;\text{donc}\;v'(x)=e^{-3x}\end{cases}\\
&=u(x)\,v(x)-\int u'(x)\, v(x)\, dx
= -\frac{1}{3}\,x^2\,e^{-3x}+\frac{2}{3}\,\int x\,e^{-3x}\,dx\,,
\end{align*}
\begin{align*}\int x\,e^{-3x}\, dx &= \int u(x)\, v'(x)\, dx \quad \text{où}\quad \begin{cases}
 u(x)=x\,,\;\text{donc}\; u'(x)=1\\
 v(x)=-\frac{1}{3}\,e^{-3x}\,,\;\text{donc}\;v'(x)=e^{-3x}\end{cases}\\
&=u(x)\,v(x)-\int u'(x)\, v(x)\, dx\\
&= -\frac{1}{3}\,x\,e^{-3x}+\frac{1}{3}\,\int e^{-3x}\,dx
=-\frac{1}{3}\,x\,e^{-3x}-\frac{1}{9}\,e^{-3x}+C.
\end{align*}
Finalement, 
\[
\int x^2\,e^{-3x}\, dx =-\frac{1}{27}\,(9\,x^2+6\,x+2)\, e^{-3x}+C.
\]
\item On a
\begin{align*}\int e^{2x}\, \sin(x)\, dx &= \int u(x)\, v'(x)\, dx \quad \text{où}\quad \begin{cases}
 u(x)=e^{2x}\,,\;\text{donc}\; u'(x)=2\,e^{2x} \\
 v(x)=-\cos(x)\,,\;\text{donc}\;v'(x)=\sin(x)\end{cases}\\
&=u(x)\,v(x)-\int u'(x)\, v(x)\, dx\\
&= - e^{2x}\,\cos(x) +2\, \int e^{2x}\, \cos(x) \,dx\,,
\end{align*}
\begin{align*}\int e^{2x}\, \cos(x)\, dx &= \int u(x)\, v'(x)\, dx \quad \text{où}\quad \begin{cases}
 u(x)=e^{2x}\,,\;\text{donc}\; u'(x)=2\,e^{2x} \\
 v(x)=\sin(x)\,,\;\text{donc}\;v'(x)=\cos(x)\end{cases}\\
&=u(x)\,v(x)-\int u'(x)\, v(x)\, dx\\
&=  e^{2x}\,\sin(x) -2\, \int e^{2x}\, \sin(x) \,dx.
\end{align*}
D'où l'on déduit
\[
\int e^{2x}\, \sin(x)\,
dx=-\frac{1}{5}\,e^{2x}\,\cos(x)+\frac{2}{5}\,e^{2x}\,\sin(x)+C.
\]
\end{enumerate}
\end{sol}


%\begin{exo}
%%Difficulté : 1/5
%Chapitre : Intégration
%Calculer
%\[ \int_1^2 \frac{1}{x \sqrt{2x+1}} \, dx.\]
%\end{exo}

\begin{sol}
On procède au changement de variable $t = \sqrt{2x +1}$. Alors, $dt = \frac{1}{\sqrt{2x +1}}dx$, $x= \frac{1}{2}(t^2 - 1)$ et donc 
\[  \int_1^2 \frac{1}{x \sqrt{2x+1}} \, dx = 
\int_{\sqrt{3}}^{\sqrt{5}} \frac{1}{\frac{1}{2}(t^2 - 1)}\, dt= 2 \int_{\sqrt{3}}^{\sqrt{5}} \frac{1}{t^2 - 1}\, dt  = 2 \int_{\sqrt{3}}^{\sqrt{5}} \frac{\frac{1}{2}}{t - 1} - \frac{\frac{1}{2}}{t+1} \, dt.\]
Finalement,
\[  \int_1^2 \frac{1}{x \sqrt{2x+1}} \, dx =\int_{\sqrt{3}}^{\sqrt{5}} \frac{1}{t - 1} - \frac{1}{t+1} \, dt = [ \ln(|t-1|) - \ln(|t+1|)]_{\sqrt{3}}^{\sqrt{5}}\]
et donc 
\[  \int_1^2 \frac{1}{x \sqrt{2x+1}} \, dx = \ln(\sqrt{5}-1) - \ln(\sqrt{5}+1) - \ln(\sqrt{3}-1) + \ln(\sqrt{3}+1) = \ln \left( \frac{(\sqrt{5}-1)(\sqrt{3}+1)}{(\sqrt{5}+1)(\sqrt{3}-1)} \right).\]
\end{sol}


%\begin{exo}
%%Difficulté : 2/5
%Chapitre : Intégration
Pour tout $n \in \N^{\star}$, on pose \[ I_n= \int_0^1 \frac{dx}{(x^2+1)^n}.\]
\begin{enumerate}
\item Exprimer $I_{n+1}$ en fonction de $I_n$ pour tout $n \in \N^{\star}$.
\item En déduire la valeur de $I_3$.
\end{enumerate}
%\end{exo}

\begin{sol}
\begin{enumerate}
\item Une intégration par parties donne, en posant $u(x)=(x^2+1)^{-n}$ et $v'(x)=1$, \[ I_n = \left[ \frac{x}{(x^2+1)^n} \right]_0^1 +2n \int_0^1 \frac{x^2}{(x^2+1)^{n+1}} \, dx
= \frac{1}{2^n}+2n \int_0^1  \frac{x^2}{(x^2+1)^{n+1}} \,dx.\]
Or, \[ \int_0^1  \frac{x^2}{(x^2+1)^{n+1}} \,dx = \int_0^1  \frac{x^2 +1}{(x^2+1)^{n+1}} \,dx  - 
\int_0^1  \frac{1}{(x^2+1)^{n+1}} \,dx = I_n - I_{n+1}.\]
Regroupant les termes, on trouve
\[ 2n I_{n+1} =(2n - 1)I_n + \frac{1}{2^n}\]
et donc
\[ I_{n+1} = \frac{2n-1}{2n} I_n + \frac{1}{n2^{n+1}}.\]
\item Sachant que
\[ I_1= [\arctan(x)]_0^1 = \frac{\pi}{4},\]
on trouve 
\[ I_2=\frac{\pi}{8} + \frac{1}{4} \quad \text{et} \quad I_3= \frac{3 \pi}{32} + \frac{1}{4}.\]
\end{enumerate}
\end{sol}




%\begin{exo}
%%Difficulté : 5/5
%Chapitre : Intégration
%Étudier la fonction définie sur $\R$ par : \[ f:x \mapsto \int_{0}^{\sin(x)^2}  \arcsin(\sqrt{t}) \, dt +\int_{0}^{\cos(x)^2}  \arccos(\sqrt{t}) \, dt.\]
%\end{exo}

\begin{sol}
Commençons par remarquer que $f$ est $\pi$-périodique, car $sin(\pi+x)^2=sin(x)^2$ et $cos(\pi+x)^2=cos(x)^2$. Il suffit donc d'étudier $f$ sur $[0,\pi]$. De plus, on a également $f(\pi-x)=f(x)$ et donc il suffit d'étudier $f$ sur $\left[0, \frac{\pi}{2} \right]$. Soit $u(x)=\int_0^x \arcsin(\sqrt{t}) \,dt$ et $v(x)=\int_0^x \arccos(\sqrt{t}) \,dt$. Puisque les fonctions $\arcsin$ et $\arccos$ sont continues sur $[0,1]$, $u$ et $v$ sont de classe $C^1$ sur $[0,1]$, avec $u'(x)=\arcsin(\sqrt{x})$ et $v'(x)=\arccos(\sqrt{x})$. De plus, $f(x)=u(\sin(x)^2)+v(cos(x)^2)$. Par composition, $f$ est de classe $C^1$ sur $\left[0,\frac{\pi}{2} \right]$, par conséquent sur $\R$, et sa dérivée est 
\[ f'(x)=2\sin(x) \cos(x) \arcsin(\sqrt{\sin(x)^2}) - 2\sin(x) \cos(x) \arccos(\sqrt{\cos(x)^2}).\]
On peut se restreindre à $x$ dans l'intervalle $\left[ 0 ,\frac{\pi}{2} \right]$], et pour $x$ dans cet intervalle, tout se passe bien, à savoir $\sqrt{\sin(x)^2}=\sin(x)$, $\sqrt{\cos(x)^2}=\cos(x)$ et $\arcsin(\sin(x))=x$, $\arccos(\cos(x))=x$. Il vient :
\[ f'(x)=2x\sin(x)\cos(x)-2x\sin(x)\cos(x)=0.\] Ainsi, la fonction $f$ est constante sur $\R$. De plus, on a, pour tout $x \in \R$, 
\[ f(x)=f \left( \frac{\pi}{4} \right) =  \int_0^{\frac{1}{2}} \arccos(\sqrt{t}) + \arcsin(\sqrt{t}) \, dt= \int_0^{\frac{1}{2}} \frac{\pi}{2} \, dt = \frac{\pi}{4},\]
puisque l'on sait que pour tout $x \in [-1,1]$, $\arcsin(x)+\arccos(x)= \frac{\pi}{2}$.
\end{sol}
















\end{document}
