%%%%%%%%%%%%%%%%%%%%%%%%%%%%%%%%%%%   Packages de base   %%%%%%%%%%%%%%%%%%%%%%%%%%%%%%%%%%%


% Classe du document (book : chapitre inclus, pratique pour les gros cours (PASS, Tremplin), article (8pt) ou amsart (12pt) : pratique pour les cours sur une seule thématique)
\documentclass[a4paper, 11pt,openany]{article}% openany évite de commencer les chapitres sur une page forcément impaire (évite les pages blanches)

% Encodage, langue et font
\usepackage[utf8]{inputenc} % Pour les accents (conversion entre ces caractères accentués et les commandes d’accentuation)
\usepackage[french]{babel} % Français
\usepackage[T1]{fontenc} % Permet d'afficher et de prendre correctement en charge ces caractères accentués du point de vue du fichier de sortie

% Si on veut faire apparaître le texte avec des caractères plus larges
%\usepackage[lf]{Baskervaldx} % Texte en plus "gras"
%\usepackage[bigdelims,vvarbb]{newtxmath} % Lettre mathématiques en plus "gras"
%\usepackage[cal=boondoxo]{mathalfa} % Style pour les \mathscal
%\renewcommand*\oldstylenums[1]{\textosf{#1}}

% Mise en page
% Version automatique mais problème d'écart entre l'en-tête et le texte.
%\usepackage{fullpage} % Numérotation bas de page (et mise en page pleine).
\usepackage{setspace} % Interligne
\onehalfspacing % Interligne
\usepackage[a4paper]{geometry}% Package pour mise en page
\geometry{hscale=0.8,vscale=0.8,centering,headsep=0.5cm} % Marges, Headsep permet de ne pas coller le texte à l'en-tête

\setlength{\parindent}{0pt} % Supprimer les identations par défaut

\usepackage{fancyhdr} %Package permettant de mettre des en-têtes et des pieds de pages
\pagestyle{fancy}
\fancyhead{} % Enleve ce qui est présent par défaut
\fancyfoot{}% Enleve ce qui est présent par défaut
\usepackage{extramarks} % Pour écrire proprement le chapitre dans l'en-tête
%\renewcommand{\chaptermark}[1]{\markboth{\chaptername\ \thechapter.\ #1}{}} % Redéfinition du chapitre pour écriture dans l'en-tête sous la forme "Chapitre n. Nom du chapitre"
%\renewcommand{\chaptermark}[1]{\markboth{\thechapter.\ #1}{}} % Forme "n. Nom du chapitre"
%\renewcommand{\chaptermark}[1]{\markboth{#1}{}} % Forme "Nom du chapitre"
\renewcommand{\headrulewidth}{0.6pt} % Epaisseur trait en haut
\renewcommand{\footrulewidth}{0.6pt} % Epaisseur trait en bas
\setlength{\headheight}{15pt}
% Placer les informations où on veut (à noter : E: Even page ; O: Odd page ; L: Left field ; C: Center field ; R: Right field ; H: Header ; F: Footer) :
\fancyhead[LO,LE]{\slshape  \textbf{MPSI2}}
\fancyhead[RO,RE]{\slshape \textbf{Corrigé - Colle 6 (Sujet 3)}}
%\fancyhead[CO,CE]{---Draft---}
\fancyfoot[C]{\thepage}
%\fancyfoot[LO, LE] {\slshape Damien GOBIN}
%\fancyfoot[RO, RE] {\slshape Année 2021-2022}

% Couleurs
\usepackage{xcolor}
\definecolor{BleuTresFonce}{rgb}{0.0,0.0,0.250}
%\definecolor{bleu}{rgb}{0.36, 0.54, 0.66}
\definecolor{bleu}{rgb}{0.47, 0.62, 0.8}
%\definecolor{vert}{rgb}{0.33, 0.42, 0.18}
%\definecolor{vert}{rgb}{0.52, 0.73, 0.4}
\definecolor{vert}{rgb}{0.56, 0.74, 0.56}
% Liens hypertextes
\usepackage[colorlinks,final,hyperindex]{hyperref}
\hypersetup{
	pdftex,
	linkcolor=BleuTresFonce,
	citecolor=BleuTresFonce,
	filecolor=BleuTresFonce,
	urlcolor=BleuTresFonce,
	pdftitle=TemplateCours,
	pdfauthor=Damien Gobin,
	pdfsubject=,
	pdfkeywords=
}


% Inclure des figures
\usepackage{graphicx} % Permet d'utiliser includegraphics
\usepackage{pdfpages} % Permet d'insérer des fichiers pdf

% Utiliser des commentaires
\usepackage{comment}
%\excludecomment{sol} %commenter cette ligne si on veut faire apparaitre les solutions d'exercices

%%%%%%%%%%%%%%%%%%%%%%%%%%%%%%%%%%%   Packages mathématiques  %%%%%%%%%%%%%%%%%%%%%%%%%%%%%%%%%%%

% Packages mathématiques de base
%\usepackage{amsfonts} % Permet de taper des ensembles 
\usepackage{amsmath} % Permet de taper des maths (contient cleveref)
\usepackage{amsthm} % Permet de définir une environnement pour les théorèmes
\usepackage{amssymb} % Donne accès a plus de symboles mathématiques (charge de façon automatique amsfonts)
% \usepackage{mathrsfs} % Ecriture ronde type mathcal
%\usepackage{bbold} % Permet d'avoir l'indicatrice mais change les textbb
\usepackage{stmaryrd} % Pour les intervalles entiers [[ ]]
\usepackage{pifont} % Pour avoir accès à plus de caractères

%Utiliser les symboles jeu de cartes
\DeclareSymbolFont{extraup}{U}{zavm}{m}{n}
\DeclareMathSymbol{\varheart}{\mathalpha}{extraup}{86}
\DeclareMathSymbol{\vardiamond}{\mathalpha}{extraup}{87}



% Référence dans le document
\usepackage[noabbrev,capitalize]{cleveref}

% Tableau
\usepackage{tabularx} % Tableau
\usepackage{multirow} % Pour créer des tableaux en subdivisant les lignes
\usepackage{diagbox} %Pour créer des crois dans les tableaux à double entrées
\usepackage{enumitem} %Permet de scinder une énumération et de reprendre au numéro suivant
\usepackage{multicol} % Création de document en colonne avec plusieurs colonnes
\multicolsep=5pt % supprime l'espace vertical

% Modification des itemizes
\setitemize{label=$\bullet$} % Utilise le package enumitem et met des points au lieu des tirets

% Ecrire des algorithmes en "français" (exemple issu du cours d'optimisation 2020/2021)
\usepackage[linesnumbered, french, frenchkw,ruled]{algorithm2e}
% Exemple :
%\begin{center}
%\begin{algorithm}
%\Entree{Un graphe $G = (V,E)$, $|V| = n$, $|E| = m$ et pour chaque arête $e$ de $E$ son poids $c(e)$;}
%\Sortie{Un arbre (ou une forêt) maximal $A = (V,F)$ et de poids minimum ;} 
%Trier et renuméroter les arêtes de $G$ dans l'ordre croissant de leur poids : $c(e_1) \leqslant c(e_2) \leqslant ... \leqslant c(e_m)$ \;
%Poser $F := \emptyset$ et $k = 0$\;
%\Tq{$k<m$ et $|F| < n-1$}{Si $e_{k+1}$ ne forme pas de cycle avec $F$ alors $F := F \cup \{ e_k \}$\;
%$k := k +1 $\;}
%\caption{Algorithme de Kruskal théorique (1956)}
%\end{algorithm}
%\end{center}

% Inclure du code Python avec coloration syntaxique et bloc gris (exemple issu du TP2 d'optimisation 2020/2021)
%\usepackage{xcolor} % Déjà importé plus haut
\usepackage{listings}
\lstset{backgroundcolor=\color{darkWhite},literate={á}{{\'a}}1 {é}{{\'e}}1 {í}{{\'i}}1 {ó}{{\'o}}1 {ú}{{\'u}}1{Á}{{\'A}}1 {É}{{\'E}}1 {Í}{{\'I}}1 {Ó}{{\'O}}1 {Ú}{{\'U}}1{à}{{\`a}}1 {è}{{\`e}}1 {ì}{{\`i}}1 {ò}{{\`o}}1 {ù}{{\`u}}1{À}{{\`A}}1 {È}{{\'E}}1 {Ì}{{\`I}}1 {Ò}{{\`O}}1 {Ù}{{\`U}}1{ä}{{\"a}}1 {ë}{{\"e}}1 {ï}{{\"i}}1 {ö}{{\"o}}1 {ü}{{\"u}}1{Ä}{{\"A}}1 {Ë}{{\"E}}1 {Ï}{{\"I}}1 {Ö}{{\"O}}1 {Ü}{{\"U}}1{â}{{\^a}}1 {ê}{{\^e}}1 {î}{{\^i}}1 {ô}{{\^o}}1 {û}{{\^u}}1{Â}{{\^A}}1 {Ê}{{\^E}}1 {Î}{{\^I}}1 {Ô}{{\^O}}1 {Û}{{\^U}}1{œ}{{\oe}}1 {Œ}{{\OE}}1 {æ}{{\ae}}1 {Æ}{{\AE}}1 {ß}{{\ss}}1{ű}{{\H{u}}}1 {Ű}{{\H{U}}}1 {ő}{{\H{o}}}1 {Ő}{{\H{O}}}1{ç}{{\c c}}1 {Ç}{{\c C}}1 {ø}{{\o}}1 {å}{{\r a}}1 {Å}{{\r A}}1{€}{{\EUR}}1 {£}{{\pounds}}1}
\lstdefinestyle{stylepython}{        language=Python,        basicstyle=\ttfamily,    commentstyle=\color{green},    keywordstyle=\color{blue},    stringstyle=\color{olive},    numberstyle=\tiny,        numbers=left,        stepnumber=1,         numbersep=5pt}
% Exemple :
%\begin{lstlisting}[style=stylepython]
%import networkx as nx #Cette bibliothèque permettra de manipuler des graphes
%import matplotlib.pyplot as plt #Cette bibliothèque permettra de les représenter
%import numpy as np
%\end{lstlisting}

%Package permettant de tracer des graphes, des courbes, etc.. (exemple issu du cours d'optimisation 2020/2021)
\usepackage{tikz}
\usepackage{tikz-cd}
\usepackage{tkz-tab} % Pour les tableaux de signes
%\usetikzlibrary{shapes,backgrounds}
\usetikzlibrary{
	decorations.pathmorphing,
	arrows,
	arrows.meta,
	calc,
	shapes,
	shapes.geometric,
	decorations.pathreplacing,}
\tikzcdset{arrow style=tikz, diagrams={>=stealth}}
\tikzset{>=stealth'}
\tikzstyle{rond}=[draw,circle,thick,fill=white]
% Exemple courbe :
%\begin{center}
%\begin{tikzpicture}
%	\def\shift{.5}
%	\def\xmax{6}
%	\def\ymax{7}
%	\draw[->] (-\shift,0) -- (\xmax+\shift,0) node[below] {$x$}; 
%	\foreach \x in {1,...,\xmax}{ \pgfmathtruncatemacro\xbis{100*\x}
%		\draw (\x,0) -- (\x,-.3*\shift) node[below] {$\scriptstyle{\xbis}$};
%	}
%	% axe ordonnees
%	\draw[->] (0,-\shift) -- (0,\ymax+\shift) node[left] {$y$};  
%	\foreach \y in {1,...,\ymax}{ \pgfmathtruncatemacro\ybis{100*\y}
%		\draw (0,\y) -- (-.3*\shift,\y) node[left] {$\scriptstyle{\ybis}$};
%	}
%	\draw[fill=yellow] (0,0) -- (3,0) -- (0,6) -- cycle;
%	
%	\draw[thick, purple] plot [domain=-.15:6.15, variable=\t] (\t,6-\t);
%	\node[above, text=purple] (A) at (6.9,.1) {$x+y = 600$};
%	%%
%	\draw[thick, purple] plot [domain=-.1:3.1, variable=\t] (\t,6-2*\t);
%	\node[right, text=purple] (B) at (1.1,5) {$2x+y = 600$};
%	%%
%	\draw[thick, blue] plot [domain=-.1:3.85, variable=\t] (\t,6-1.6*\t);
%	%%
%	\draw[fill=blue] (0,6) circle (3pt);
%	\node[right, text=blue] (B) at (0.1,6.3) {Solution optimale};
%\end{tikzpicture}
%\end{center}
% Exemple graphe :
%\begin{center}
%\begin{tikzpicture}[scale=0.9]
%    \draw[thick] (-1.5,0) node[rond] {$1$} coordinate (A) --
%        ++(3,0) node[rond] {$2$} coordinate (B) --
%        ++(-1,-1) node[rond] {$4$}  coordinate (D) --
%        ++(-1,0) node[rond] {$3$}  coordinate (C) --
%        ++(0,-1) node[rond] {$5$}  coordinate (E) --
%        ++(1,0) node[rond] {$6$}  coordinate (F) --
%        ++(1,-1) node[rond] {$8$}  coordinate (H) --
%        ++(-3,0) node[rond] {$7$}  coordinate (G) -- (A)
%        (A) -- (C)
%        (G) -- (E)
%        (D) -- (F)
%        (B) -- (H);
%\end{tikzpicture}
%\end{center}

%%%%%%%%%%%%%%%%%%%%%%%%%%%%%%%%%%%   Environnements  %%%%%%%%%%%%%%%%%%%%%%%%%%%%%%%%%%%

%Différents types à donner aux environnements théorèmes, définitions, etc...
%[theorem] permet de numéroter par rapport à la section en cours
%\renewcommand{\theexem}{\empty{}} permet de ne pas numéroterNe numérote pas les exemples

%\usepackage{thmtools} % Permet de créer un environnement de théorème avec des cadres
%\declaretheorem[thmbox=L]{boxtheorem L}
%\declaretheorem[thmbox=M]{boxtheorem M}
%\declaretheorem[thmbox=S]{boxtheorem S}
%
%%\usepackage[dvipsnames]{xcolor}
%%\declaretheorem[shaded={bgcolor=Lavender,textwidth=12em}]{BoxI}
%\declaretheorem[shaded={rulecolor=black,rulewidth=1pt}]{BoxII}
%
%
%\usepackage[leftmargin=0.1865cm,rightmargin=0.6cm]{thmbox}
% Voici ici pour les options https://ctan.mines-albi.fr/macros/latex/contrib/thmbox/thmbox.pdf

%% Ecrit de cette façon tout le monde a le même style : Titre en droit et texte en italique très léger.
%\newtheorem[L,leftmargin=0.1865cm,rightmargin=0.1865cm,cut=true]{thm}{Théorème}[subsection]
%%\renewcommand{\thetheorem}{\empty{}} 
%\newtheorem[M,leftmargin=0.1865cm,rightmargin=0.1865cm,cut=true]{propo}[thm]{Proposition}
%%\renewcommand{\thepropo}{\empty{}} 
%\newtheorem[M,leftmargin=0.1865cm,rightmargin=0.1865cm,cut=true]{prop}[thm]{Propriété}
%%\renewcommand{\theprop}{\empty{}} 
%\newtheorem[M,leftmargin=0.1865cm,rightmargin=0.1865cm,cut=true]{coro}[thm]{Corollaire}
%\newtheorem[M,leftmargin=0.1865cm,rightmargin=0.1865cm,cut=true]{lem}[thm]{Lemme}
%\newtheorem[M,leftmargin=0.1865cm,rightmargin=0.1865cm,cut=true]{defi}[theorem]{Définition}
%%\renewcommand{\thedefinition}{\empty{}}
%\newtheorem[S,leftmargin=0.1865cm,rightmargin=0.1865cm,cut=true]{exem}{Exemple}
%\renewcommand{\theexem}{\empty{}} 
%\newtheorem[S,leftmargin=0.1865cm,rightmargin=0.1865cm,cut=true]{exo}{Exercice}
%\renewcommand{\theexo}{\empty{}}
%\newtheorem[S,leftmargin=0.1865cm,rightmargin=0.1865cm,cut=true]{sol}{Solution}
%\renewcommand{\thesol}{\empty{}} 
%\newtheorem[S,leftmargin=0.1865cm,rightmargin=0.1865cm,cut=true]{hypo}{Hypothesis}[section]
%\newtheorem[S,leftmargin=0.1865cm,rightmargin=0.1865cm,cut=true]{remark}[thm]{Remarque}%[section]
%\newtheorem{dem}{Démonstration}
%\renewcommand{\thedem}{\empty{}} 
%\newtheorem[S]{nota}{Notation}[section]

\theoremstyle{plain}
\newtheorem{theorem}{Théorème}[subsection]
%\renewcommand{\thetheorem}{\empty{}} 
\newtheorem{propo}[theorem]{Proposition}
%\renewcommand{\thepropo}{\empty{}} 
\newtheorem{prop}[theorem]{Propriété}
%\renewcommand{\theprop}{\empty{}} 
\newtheorem{coro}[theorem]{Corollaire}
\newtheorem{lemma}[theorem]{Lemme}

\theoremstyle{definition}
\newtheorem{definition}[theorem]{Définition}
%\renewcommand{\thedefinition}{\empty{}}
\newtheorem{exem}{Exemple}
\renewcommand{\theexem}{\empty{}} 
\newtheorem{cours}{Question de cours}
\renewcommand{\thecours}{\empty{}} 
\newtheorem{exo}{Exercice}
%\renewcommand{\theexo}{\empty{}} 
\newtheorem{sol}{Solution de l'exercice}
%\renewcommand{\thesol}{\empty{}} 
\newtheorem{hypo}{Hypothesis}[section]
\newtheorem{remark}[theorem]{Remarque}%[section]


\theoremstyle{remark}
\newtheorem{dem}{Démonstration}
\renewcommand{\thedem}{\empty{}} 
\newtheorem{nota}{Notation}[section]

%%%%%%%%%%%%%%%%%%%%%%%%%%%%%%%%%%%   Raccourcis   %%%%%%%%%%%%%%%%%%%%%%%%%%%%%%%%%%%
 
% Ensembles
\newcommand{\R}{\mathbb{R}} 
\newcommand{\Q}{\mathbb{Q}}
\newcommand{\C}{\mathbb{C}}
\newcommand{\Z}{\mathbb{Z}}
\newcommand{\K}{\mathbb{K}}
\newcommand{\N}{\mathbb{N}}
\newcommand{\D}{\mathbb{D}}


% Lettres calligraphiques
\newcommand{\calP}{\mathcal{P}} 
\newcommand{\calF}{\mathcal{F}} 
\newcommand{\calD}{\mathcal{D}} 
\newcommand{\calQ}{\mathcal{Q}} 
\newcommand{\calT}{\mathcal{T}} 

% Noyau et image
\newcommand{\im}{\text{Im}} 
\newcommand{\noy}{\text{Ker}} 
\newcommand{\card}{\text{Card}} 

%Epsilon
\newcommand{\vare}{\varepsilon} 


% Fonctions usuelles
%\newcommand{\un}{\mathbb{1}} % Indicatrice en utilisant le package \usepackage{bbold} mais celui modifie les mathbb donc on définit l'indicatrice à la main
\def\un{{\mathchoice {\rm 1\mskip-4mu l} {\rm 1\mskip-4mu l}
{\rm 1\mskip-4.5mu l} {\rm 1\mskip-5mu l}}} % Indicatrice


\title{Corrigé - Colle 6 (Sujet 3)}
\author{MPSI2\\
Année 2021-2022}
%\author{Mathématiques P.A.S.S. 1}
\date{9 novembre 2021}


\begin{document}


   \maketitle
%      \rule{\linewidth}{0.5mm}
      \rule{\linewidth}{0.5mm}
%  \begin{center}
%  %    \rule{\linewidth}{0.5mm}\\[0.4cm]
%       { \huge \bfseries Mathématiques pour le P.A.S.S 1\\[0.4cm] }
%    \rule{\linewidth}{0.5mm}\\[4cm]
%     \end{center}

%\begin{cours}
%Que peut-on dire de l'intégrale d'une fonction périodique. Démontrer.
%\end{cours}

%\begin{exo}
%%Difficulté : 1/5
%Chapitre : Intégration
%Donner une primitive des fonctions suivantes :
%\[ f(x) = \frac{1}{2x-3} \quad ; \quad  g(x) =  \frac{x+3}{\sqrt{x^2 + 6x}}\]
%\[ h(x) = (e^x +1)^3 e^x \quad \text{et} \quad i(x) = \frac{1}{1+e^x}.\]
%\end{exo}

\begin{sol}
\begin{enumerate}
\item On remarque qu'en
posant $u(x)=2\,x-3$, on a $u'(x)= 2$, de sorte que $f(x)=\frac{1}{2}\,\frac{u'(x)}{u(x)}$. On a donc 
\[
f(x)= \frac{1}{2}\,\frac{u'(x)}{u(x)}=\frac{d}{dx} \left(\frac{1}{2}\, \ln
(|u(x)|)\right)=\frac{d}{dx} \left(\frac{1}{2}\,\ln|2\,x-3|\right).
\] 
\item On remarque qu'en posant $u(x)=x^2+6\,x$, on a $u'(x)=2(x+3)$, de sorte que
$g(x)=\frac{1}{2}\,\frac{u'(x)}{u^{\frac{1}{2}}(x)}$. On a alors 
\[
g(x)= \frac{1}{2}\,
u'(x)\,u^{-\frac{1}{2}}(x)=\frac{d}{dx} \left( u^{\frac{1}{2}}(x)\right)
=\frac{d}{dx} \left( (x^2+6\,x)^{\frac{1}{2}}\right).
\]
\item On remarque qu'en posant $h(x)=e^{x}+1$, on a $u'(x)=e^{x}$, de sorte que $h(x)=
u'(x)\, u^3(x)$. On a donc
\begin{align*}
f_{25}(x)\, &= u'(x)\, u^3(x)\,  = \frac{d}{dx}\left(\frac{1}{4}\, u^{4}(x)\right)= 
 \frac{d}{dx}\left(\frac{1}{4}\, (e^{x}+1)^{4}\right)\,.
\end{align*}
\item On a 
\[
i(x)\,  = \frac{1+e^{x}-e^{x}}{e^{x}+1}=1-\frac{e^{x}}{e^{x}+1}=\frac{d}{dx}\left(x-\ln (e^{x}+1)\right)\,.
\]
\end{enumerate}
\end{sol}

%\begin{exo}
%%Difficulté : 1/5
%Chapitre : Intégration
%Donner une primitive des fonctions suivantes :
%\[ f(x) = x^n \ln(x)\quad ; \quad
%g(x) = x\sqrt{1+x} \quad \text{et} \quad h(x) = \sin(\ln(x))\]
%\end{exo}


\begin{sol}
\begin{enumerate}
\item  A priori, on ne connaît pas
de primitive de la fonction $\ln$.
\begin{align*}\int x^n\,\ln|x| dx &= \int u(x)\, v'(x)\, dx \quad \text{où}\quad \begin{cases}
 u(x)=\ln|x|\,,\;\text{donc}\; u'(x)=\frac{1}{x}\\
 v(x)=\frac{x^{n+1}}{n+1}\,,\;\text{donc}\;v'(x)=x^n\end{cases}\\
&=u(x)\,v(x)-\int u'(x)\, v(x)\, dx\\
&= \left(\frac{x^{n+1}}{n+1}\right)\,\ln|x|-\int\left(\frac{x^{n}}{n+1}\right)\,dx
=\frac{x^{n+1}}{n+1}\,\left(\ln|x|-\frac{1}{n+1}\right)+C.
\end{align*}
Remarque : pour $n=0$, on obtient $\displaystyle{\int \ln|x|\,dx=x\,(\ln|x|-1)+C}$.
\item On fait l'i.p.p. suivante, en remarquant que $\sqrt{1+x}=w'(x)\,\sqrt{w(x)}$, où $w(x)=1+x$ :
\begin{align*}
\int x\,\sqrt{1+x}\, dx &= \int u(x)\, v'(x)\, dx \quad \text{où}\quad \begin{cases}
 u(x)=x\,,\;\text{donc}\; u'(x)=1\\
 v(x)=\frac{2}{3}\,(1+x)^{\frac{3}{2}}\,,\;\text{donc}\;v'(x)=\sqrt{1+x}\end{cases}\\
&=u(x)\,v(x)-\int u'(x)\, v(x)\, dx\\
&= \frac{2}{3}\,x\, (1+x)^{\frac{3}{2}}-\frac{2}{3}\,\int(1+x)^{\frac{3}{2}}\,dx
=\frac{2}{3}\,x\, (1+x)^{\frac{3}{2}}-\frac{4}{15}\, (1+x)^{\frac{5}{2}}+C.
\end{align*}
\item L'astuce consiste ici à faire deux i.p.p. successives afin de faire apparaître $-\int h(x)\,dx$ dans la
deuxième i.p.p.
\begin{align*}
  \int \sin(\ln|x|)\, dx &= \int u(x) v'(x)\, dx\mbox{\hspace{1cm} avec } u(x)=\sin(\ln|x|), v(x)=x\\
&= u(x) v(x) - \int  u'(x) v(x)\\
  &= x \sin(\ln|x|) - \int x \cos(\ln|x|) \, \\frac{1}{x}\, dx = x
  \sin(\ln|x|) -\int \cos(\ln|x|)\, dx
\end{align*}
\begin{align*}
  \int \cos(\ln|x|)\, dx &= \int u(x) v'(x)\, dx 
\mbox{\hspace{1cm} avec } u(x)=\cos(\ln|x|),v(x)=x\\
&= u(x) v(x) 
- \int u'(x) v(x)\, dx= 
x \cos(\ln|x|) - \int x (-\sin(\ln|x|)) \, \frac{1}{x}\, dx\\
&= x \cos(\ln|x|) +\int \sin(\ln|x|)\, dx
\end{align*}
et donc
\[
\int \sin(\ln|x|)\, dx = \frac{x}{2}(\sin(\ln|x|)-\cos(\ln|x|))+C.
\]
\end{enumerate}
\end{sol}
   
%
%\begin{exo}
%%Difficulté : 1/5
%Chapitre : Intégration
%Calculer
%\[ \int_0^1 \frac{1}{x^2 + x + \frac{5}{4}} \, dx.\]
%\end{exo}

\begin{sol}
On commence par remarquer que
\[ x^2 + x + \frac{5}{4} = \left( x + \frac{1}{2} \right)^2 +1.\]
On utilise alors le changement de variable $t = x+\frac{1}{2}$ et on obtient
\[ \int_0^1 \frac{1}{x^2 + x + \frac{5}{4}} \, dx= \int_0^1 \frac{1}{\left( x +\frac{1}{2} \right)^2 +1} \, dx = \int_{\frac{1}{2}}^{\frac{3}{2}} \frac{1}{t^2 +1} \, dt\]
et finalement
\[ \int_0^1 \frac{1}{x^2 + x + \frac{5}{4}} \, dx = \left[ \arctan(t) \right]_{\frac{1}{2}}^{\frac{3}{2}} = \arctan \left( \frac{3}{2} \right) - \arctan \left( \frac{1}{2} \right). \]
\end{sol}


%\begin{exo}
%%Difficulté : 1/5
%Chapitre : Intégration
%Calculer 
%\[ I = \int_1^{\frac{5}{2}} \sqrt{-x^2 + 2x + 8} \, dx.\]
%\end{exo}

\begin{sol}
On a
\[ - x^2 + 2x + 8 = -(x^2 - 2x - 8) = -((x-1)^2 - 9) = 9 - (x-1)^2.\]
 On a 
 \[ I= \int_1^{\frac{5}{2}} \sqrt{9- (x-1)^2} \, dx = 3 \int_{1}^{\frac{5}{2}} \sqrt{1 - \left( \frac{x-1}{3} \right)^2} \, dx.\]
 On pose alors $u=\frac{x-1}{3}$ pour se ramener à une intégrale du type $\int \sqrt{1 - u^2}\, du$, que l'on sait calculer. On a $dx=3du$, et le changement de variables est affine, donc bijectif. On en déduit que 
 \[ I=9 \frac{0}{\frac{1}{2}} \sqrt{1-u^2} \, du.\]
 On pose ensuite $u=\sin(t)$. La fonction $\sin$ réalisant une bijection de l'intervalle $\left[0,\frac{\pi}{6} \right]$ sur l'intervalle $\left[0, \frac{1}{2} \right]$, on en déduit que 
 \[ I=9 \int_0^{\frac{\pi}{6}} \sqrt{1-\sin(t)^2} \cos(t) \, dt = 9\int_0^{\frac{\pi}{6}} \cos(t)^2 \, dt = \frac{9}{2} \int_0^{\frac{\pi}{6}} 1+\cos(2t) \, dt.\]
 et donc
 \[ I = \frac{9}{2} \left[ t + \frac{1}{2} \sin(2t) \right]_0^{\frac{\pi}{6}} = \frac{3 \pi}{4} + \frac{9 \sqrt{3}}{8}.\]
\end{sol}


%\begin{exo}
%%Difficulté : 3/5
%Chapitre : Intégration
%Pour $n \in \N$, on pose 
%\[ I_n= \int_0^1 (1-t^2)^n \, dt.\]
%\begin{enumerate}
%\item Montrer que la suite $(I_n)$ est strictement décroissante.
%\item Montrer que, pour tout $u \in [0,1]$, on a $0 \leqslant 1-u \leqslant e^{-u}$.
%\item En déduire une majoration de $I_n$
%à l'aide de 
%$J_n = \int_0^{\sqrt{n}} e^{-x^2} \, dx$.
%\item Montrer que la suite $(J_n)$
%est majorée. En déduire que la suite $(I_n)$
%converge vers une limite que l'on calculera.
%\item Démontrer que, pour tout $n \geqslant 1$, $I_n= \frac{2n}{2n+1} I_{n-1}$.
%\item En déduire, pour $n \geqslant 0$, une expression de $I_n$ à l'aide de factorielles.
%\item En déduire une expression de $\int_0^{\frac{\pi}{2}} \cos(\theta)^{2n+1} \, d\theta$ pour tout $n\in \N$.
%\end{enumerate}
%\end{exo}

\begin{sol}
\begin{enumerate}
\item On va d'abord démontrer que la suite est décroissante. Soit $n \in \N^{\star}$. Pour tout $t \in [0,1]$, on a $(1-t^2)\in [0,1]$ et donc $(1-t^2)^{n+1} \leqslant (1-t^2)^n$. On intègre cette inégalité entre $0$ et $1$ et on trouve que $I_{n+1} \leqslant I_n$. Supposons maintenant que $I_n=I_{n+1}$. Alors, on a \[ \int_0^1 ((1-t^2)^n - (1-t^2)^{n+1}) \, dt=0.\]
Or, la fonction $t \mapsto (1-t^2)^n-(1-t^2)^{n+1}$ est continue et positive sur $[0,1]$ Son intégrale est nulle, donc cette fonction est identiquement nulle. Cela est faux si $t=\frac{1}{2}$. Donc $I_n \neq I_{n+1}$ et donc $I_n > I_{n+1}$.
\item La première inégalité est évidente. Pour la seconde, on peut étudier la fonction $\phi$
définie par $\phi(u)=e^{-u}-1+u$ sur l'intervalle $[0,1]$. Sa dérivée est $\phi'(u)=1-e^{-u}$ et elle est positive sur $[0,1]$, donc $\phi$ est croissante sur cet intervalle. Comme $\phi(0)=0$, on en déduit le résultat. Une autre façon de procéder, pour ceux qui connaissent les fonctions convexes, est de remarquer que la fonction $u \mapsto e^{-u}$ est convexe, et donc que sa courbe représentative est au-dessus de sa tangente à l'origine, dont l'équation est justement $y=1-x$.
\item D'après l'inégalité précédente, on a \[ I_n \leqslant \int_0^1 e^{-nu^2} \, du = \int_0^{\sqrt{n}} e^{-nt^2} \, dt = \frac{1}{\sqrt{n}} J_n\]
où on a effectué le changement de variables $t=u\sqrt{n}$.
\item Si $x\geqslant 1$, alors $x^2\geqslant x$ et donc $e^{-x^2} \leqslant e^{-x}$. On en déduit que \[J_n \leqslant \int_0^1 e^{-x^2} \, dx +  \int_1^{\sqrt{n}} e^{-x} \, dx = \int_0^1 e^{-x^2} \, dx + [ e^{-x} ]_1^{\sqrt{n}} = \int_0^1 e^{-x^2} \, dx + e^{-1} - e^{-\sqrt{n}}\]
et donc
\[ J_n \leqslant \int_0^1 e^{-x^2} \, dx + e^{-1}.\]
Ceci montre bien que $(J_n)$ est majorée. De l'inégalité $0 \leqslant I_n \leqslant \frac{J_n}{\sqrt{n}}$, on déduit par le théorème des gendarmes que $(I_n)$ converge vers $0$.
\item On va intégrer par parties, en écrivant $(1-t^2)^n=(1-t^2)^n \times 1$, en dérivant $(1-t^2)^n$ et en intégrant $1$. On trouve \[ I_n=[(1-t^2)^n t]_0^1 +2n \int_0^1 t^2 (1-t^2)^{n-1} \, dt=0+2n \int_0^1 (t^2-1+1) (1-t^2)^{n-1} \, dt= 2nI_{n-1} - 2n I_n.\]
On en déduit immédiatement le résultat demandé.
\item On remarque que $I_0=1$. On en déduit que \[ I_n=\frac{(2n)(2n-2)...2}{(2n+1)(2n-1)...1}I_0=\frac{((2n)(2n-2)...2)^2}{(2n+1)(2n)(2n-1)... 1}=\frac{(2^n n!)^2}{(2n+1)!}.\]
\item On va encore faire un changement de variables en écrivant $\cos(\theta)^{2n+1}=(1-\sin(\theta)^2)^n \cos(\theta)$ et en posant $t=\sin(\theta)$ et $dt=\cos(\theta) d \theta$. On en déduit que 
\[ \int_0^{\frac{\pi}{2}} \cos(\theta)^{2n+1} \, d\theta = \int_0^1(1-t^2)^n \, dt = I_n = \frac{(2^n n!)^2}{(2n+1)!}.\]
\end{enumerate}
\end{sol}



















\end{document}
