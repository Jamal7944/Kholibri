%%%%%%%%%%%%%%%%%%%%%%%%%%%%%%%%%%%   Packages de base   %%%%%%%%%%%%%%%%%%%%%%%%%%%%%%%%%%%


% Classe du document (book : chapitre inclus, pratique pour les gros cours (PASS, Tremplin), article (8pt) ou amsart (12pt) : pratique pour les cours sur une seule thématique)
\documentclass[a4paper, 11pt,openany]{article}% openany évite de commencer les chapitres sur une page forcément impaire (évite les pages blanches)

% Encodage, langue et font
\usepackage[utf8]{inputenc} % Pour les accents (conversion entre ces caractères accentués et les commandes d’accentuation)
\usepackage[french]{babel} % Français
\usepackage[T1]{fontenc} % Permet d'afficher et de prendre correctement en charge ces caractères accentués du point de vue du fichier de sortie

% Si on veut faire apparaître le texte avec des caractères plus larges
%\usepackage[lf]{Baskervaldx} % Texte en plus "gras"
%\usepackage[bigdelims,vvarbb]{newtxmath} % Lettre mathématiques en plus "gras"
%\usepackage[cal=boondoxo]{mathalfa} % Style pour les \mathscal
%\renewcommand*\oldstylenums[1]{\textosf{#1}}

% Mise en page
% Version automatique mais problème d'écart entre l'en-tête et le texte.
%\usepackage{fullpage} % Numérotation bas de page (et mise en page pleine).
\usepackage{setspace} % Interligne
\onehalfspacing % Interligne
\usepackage[a4paper]{geometry}% Package pour mise en page
\geometry{hscale=0.8,vscale=0.8,centering,headsep=0.5cm} % Marges, Headsep permet de ne pas coller le texte à l'en-tête

\setlength{\parindent}{0pt} % Supprimer les identations par défaut

\usepackage{fancyhdr} %Package permettant de mettre des en-têtes et des pieds de pages
\pagestyle{fancy}
\fancyhead{} % Enleve ce qui est présent par défaut
\fancyfoot{}% Enleve ce qui est présent par défaut
\usepackage{extramarks} % Pour écrire proprement le chapitre dans l'en-tête
%\renewcommand{\chaptermark}[1]{\markboth{\chaptername\ \thechapter.\ #1}{}} % Redéfinition du chapitre pour écriture dans l'en-tête sous la forme "Chapitre n. Nom du chapitre"
%\renewcommand{\chaptermark}[1]{\markboth{\thechapter.\ #1}{}} % Forme "n. Nom du chapitre"
%\renewcommand{\chaptermark}[1]{\markboth{#1}{}} % Forme "Nom du chapitre"
\renewcommand{\headrulewidth}{0.6pt} % Epaisseur trait en haut
\renewcommand{\footrulewidth}{0.6pt} % Epaisseur trait en bas
\setlength{\headheight}{15pt}
% Placer les informations où on veut (à noter : E: Even page ; O: Odd page ; L: Left field ; C: Center field ; R: Right field ; H: Header ; F: Footer) :
\fancyhead[LO,LE]{\slshape  \textbf{MPSI2}}
\fancyhead[RO,RE]{\slshape \textbf{Corrigé - Colle 9 (Sujet 2)}}
%\fancyhead[CO,CE]{---Draft---}
\fancyfoot[C]{\thepage}
%\fancyfoot[LO, LE] {\slshape Damien GOBIN}
%\fancyfoot[RO, RE] {\slshape Année 2021-2022}

% Couleurs
\usepackage{xcolor}
\definecolor{BleuTresFonce}{rgb}{0.0,0.0,0.250}
%\definecolor{bleu}{rgb}{0.36, 0.54, 0.66}
\definecolor{bleu}{rgb}{0.47, 0.62, 0.8}
%\definecolor{vert}{rgb}{0.33, 0.42, 0.18}
%\definecolor{vert}{rgb}{0.52, 0.73, 0.4}
\definecolor{vert}{rgb}{0.56, 0.74, 0.56}
% Liens hypertextes
\usepackage[colorlinks,final,hyperindex]{hyperref}
\hypersetup{
	pdftex,
	linkcolor=BleuTresFonce,
	citecolor=BleuTresFonce,
	filecolor=BleuTresFonce,
	urlcolor=BleuTresFonce,
	pdftitle=TemplateCours,
	pdfauthor=Damien Gobin,
	pdfsubject=,
	pdfkeywords=
}


% Inclure des figures
\usepackage{graphicx} % Permet d'utiliser includegraphics
\usepackage{pdfpages} % Permet d'insérer des fichiers pdf

% Utiliser des commentaires
\usepackage{comment}
%\excludecomment{sol} %commenter cette ligne si on veut faire apparaitre les solutions d'exercices

%%%%%%%%%%%%%%%%%%%%%%%%%%%%%%%%%%%   Packages mathématiques  %%%%%%%%%%%%%%%%%%%%%%%%%%%%%%%%%%%

% Packages mathématiques de base
%\usepackage{amsfonts} % Permet de taper des ensembles 
\usepackage{amsmath} % Permet de taper des maths (contient cleveref)
\usepackage{amsthm} % Permet de définir une environnement pour les théorèmes
\usepackage{amssymb} % Donne accès a plus de symboles mathématiques (charge de façon automatique amsfonts)
% \usepackage{mathrsfs} % Ecriture ronde type mathcal
%\usepackage{bbold} % Permet d'avoir l'indicatrice mais change les textbb
\usepackage{stmaryrd} % Pour les intervalles entiers [[ ]]
\usepackage{pifont} % Pour avoir accès à plus de caractères

%Utiliser les symboles jeu de cartes
\DeclareSymbolFont{extraup}{U}{zavm}{m}{n}
\DeclareMathSymbol{\varheart}{\mathalpha}{extraup}{86}
\DeclareMathSymbol{\vardiamond}{\mathalpha}{extraup}{87}



% Référence dans le document
\usepackage[noabbrev,capitalize]{cleveref}

% Tableau
\usepackage{tabularx} % Tableau
\usepackage{multirow} % Pour créer des tableaux en subdivisant les lignes
\usepackage{diagbox} %Pour créer des crois dans les tableaux à double entrées
\usepackage{enumitem} %Permet de scinder une énumération et de reprendre au numéro suivant
\usepackage{multicol} % Création de document en colonne avec plusieurs colonnes
\multicolsep=5pt % supprime l'espace vertical

% Modification des itemizes
\setitemize{label=$\bullet$} % Utilise le package enumitem et met des points au lieu des tirets

% Ecrire des algorithmes en "français" (exemple issu du cours d'optimisation 2020/2021)
\usepackage[linesnumbered, french, frenchkw,ruled]{algorithm2e}
% Exemple :
%\begin{center}
%\begin{algorithm}
%\Entree{Un graphe $G = (V,E)$, $|V| = n$, $|E| = m$ et pour chaque arête $e$ de $E$ son poids $c(e)$;}
%\Sortie{Un arbre (ou une forêt) maximal $A = (V,F)$ et de poids minimum ;} 
%Trier et renuméroter les arêtes de $G$ dans l'ordre croissant de leur poids : $c(e_1) \leqslant c(e_2) \leqslant ... \leqslant c(e_m)$ \;
%Poser $F := \emptyset$ et $k = 0$\;
%\Tq{$k<m$ et $|F| < n-1$}{Si $e_{k+1}$ ne forme pas de cycle avec $F$ alors $F := F \cup \{ e_k \}$\;
%$k := k +1 $\;}
%\caption{Algorithme de Kruskal théorique (1956)}
%\end{algorithm}
%\end{center}

% Inclure du code Python avec coloration syntaxique et bloc gris (exemple issu du TP2 d'optimisation 2020/2021)
%\usepackage{xcolor} % Déjà importé plus haut
\usepackage{listings}
\lstset{backgroundcolor=\color{darkWhite},literate={á}{{\'a}}1 {é}{{\'e}}1 {í}{{\'i}}1 {ó}{{\'o}}1 {ú}{{\'u}}1{Á}{{\'A}}1 {É}{{\'E}}1 {Í}{{\'I}}1 {Ó}{{\'O}}1 {Ú}{{\'U}}1{à}{{\`a}}1 {è}{{\`e}}1 {ì}{{\`i}}1 {ò}{{\`o}}1 {ù}{{\`u}}1{À}{{\`A}}1 {È}{{\'E}}1 {Ì}{{\`I}}1 {Ò}{{\`O}}1 {Ù}{{\`U}}1{ä}{{\"a}}1 {ë}{{\"e}}1 {ï}{{\"i}}1 {ö}{{\"o}}1 {ü}{{\"u}}1{Ä}{{\"A}}1 {Ë}{{\"E}}1 {Ï}{{\"I}}1 {Ö}{{\"O}}1 {Ü}{{\"U}}1{â}{{\^a}}1 {ê}{{\^e}}1 {î}{{\^i}}1 {ô}{{\^o}}1 {û}{{\^u}}1{Â}{{\^A}}1 {Ê}{{\^E}}1 {Î}{{\^I}}1 {Ô}{{\^O}}1 {Û}{{\^U}}1{œ}{{\oe}}1 {Œ}{{\OE}}1 {æ}{{\ae}}1 {Æ}{{\AE}}1 {ß}{{\ss}}1{ű}{{\H{u}}}1 {Ű}{{\H{U}}}1 {ő}{{\H{o}}}1 {Ő}{{\H{O}}}1{ç}{{\c c}}1 {Ç}{{\c C}}1 {ø}{{\o}}1 {å}{{\r a}}1 {Å}{{\r A}}1{€}{{\EUR}}1 {£}{{\pounds}}1}
\lstdefinestyle{stylepython}{        language=Python,        basicstyle=\ttfamily,    commentstyle=\color{green},    keywordstyle=\color{blue},    stringstyle=\color{olive},    numberstyle=\tiny,        numbers=left,        stepnumber=1,         numbersep=5pt}
% Exemple :
%\begin{lstlisting}[style=stylepython]
%import networkx as nx #Cette bibliothèque permettra de manipuler des graphes
%import matplotlib.pyplot as plt #Cette bibliothèque permettra de les représenter
%import numpy as np
%\end{lstlisting}

%Package permettant de tracer des graphes, des courbes, etc.. (exemple issu du cours d'optimisation 2020/2021)
\usepackage{tikz}
\usepackage{tikz-cd}
%\usetikzlibrary{shapes,backgrounds}
\usetikzlibrary{
	decorations.pathmorphing,
	arrows,
	arrows.meta,
	calc,
	shapes,
	shapes.geometric,
	decorations.pathreplacing,}
\tikzcdset{arrow style=tikz, diagrams={>=stealth}}
\tikzset{>=stealth'}
\tikzstyle{rond}=[draw,circle,thick,fill=white]
% Exemple courbe :
%\begin{center}
%\begin{tikzpicture}
%	\def\shift{.5}
%	\def\xmax{6}
%	\def\ymax{7}
%	\draw[->] (-\shift,0) -- (\xmax+\shift,0) node[below] {$x$}; 
%	\foreach \x in {1,...,\xmax}{ \pgfmathtruncatemacro\xbis{100*\x}
%		\draw (\x,0) -- (\x,-.3*\shift) node[below] {$\scriptstyle{\xbis}$};
%	}
%	% axe ordonnees
%	\draw[->] (0,-\shift) -- (0,\ymax+\shift) node[left] {$y$};  
%	\foreach \y in {1,...,\ymax}{ \pgfmathtruncatemacro\ybis{100*\y}
%		\draw (0,\y) -- (-.3*\shift,\y) node[left] {$\scriptstyle{\ybis}$};
%	}
%	\draw[fill=yellow] (0,0) -- (3,0) -- (0,6) -- cycle;
%	
%	\draw[thick, purple] plot [domain=-.15:6.15, variable=\t] (\t,6-\t);
%	\node[above, text=purple] (A) at (6.9,.1) {$x+y = 600$};
%	%%
%	\draw[thick, purple] plot [domain=-.1:3.1, variable=\t] (\t,6-2*\t);
%	\node[right, text=purple] (B) at (1.1,5) {$2x+y = 600$};
%	%%
%	\draw[thick, blue] plot [domain=-.1:3.85, variable=\t] (\t,6-1.6*\t);
%	%%
%	\draw[fill=blue] (0,6) circle (3pt);
%	\node[right, text=blue] (B) at (0.1,6.3) {Solution optimale};
%\end{tikzpicture}
%\end{center}
% Exemple graphe :
%\begin{center}
%\begin{tikzpicture}[scale=0.9]
%    \draw[thick] (-1.5,0) node[rond] {$1$} coordinate (A) --
%        ++(3,0) node[rond] {$2$} coordinate (B) --
%        ++(-1,-1) node[rond] {$4$}  coordinate (D) --
%        ++(-1,0) node[rond] {$3$}  coordinate (C) --
%        ++(0,-1) node[rond] {$5$}  coordinate (E) --
%        ++(1,0) node[rond] {$6$}  coordinate (F) --
%        ++(1,-1) node[rond] {$8$}  coordinate (H) --
%        ++(-3,0) node[rond] {$7$}  coordinate (G) -- (A)
%        (A) -- (C)
%        (G) -- (E)
%        (D) -- (F)
%        (B) -- (H);
%\end{tikzpicture}
%\end{center}

%%%%%%%%%%%%%%%%%%%%%%%%%%%%%%%%%%%   Environnements  %%%%%%%%%%%%%%%%%%%%%%%%%%%%%%%%%%%

%Différents types à donner aux environnements théorèmes, définitions, etc...
%[theorem] permet de numéroter par rapport à la section en cours
%\renewcommand{\theexem}{\empty{}} permet de ne pas numéroterNe numérote pas les exemples

%\usepackage{thmtools} % Permet de créer un environnement de théorème avec des cadres
%\declaretheorem[thmbox=L]{boxtheorem L}
%\declaretheorem[thmbox=M]{boxtheorem M}
%\declaretheorem[thmbox=S]{boxtheorem S}
%
%%\usepackage[dvipsnames]{xcolor}
%%\declaretheorem[shaded={bgcolor=Lavender,textwidth=12em}]{BoxI}
%\declaretheorem[shaded={rulecolor=black,rulewidth=1pt}]{BoxII}
%
%
%\usepackage[leftmargin=0.1865cm,rightmargin=0.6cm]{thmbox}
% Voici ici pour les options https://ctan.mines-albi.fr/macros/latex/contrib/thmbox/thmbox.pdf

%% Ecrit de cette façon tout le monde a le même style : Titre en droit et texte en italique très léger.
%\newtheorem[L,leftmargin=0.1865cm,rightmargin=0.1865cm,cut=true]{thm}{Théorème}[subsection]
%%\renewcommand{\thetheorem}{\empty{}} 
%\newtheorem[M,leftmargin=0.1865cm,rightmargin=0.1865cm,cut=true]{propo}[thm]{Proposition}
%%\renewcommand{\thepropo}{\empty{}} 
%\newtheorem[M,leftmargin=0.1865cm,rightmargin=0.1865cm,cut=true]{prop}[thm]{Propriété}
%%\renewcommand{\theprop}{\empty{}} 
%\newtheorem[M,leftmargin=0.1865cm,rightmargin=0.1865cm,cut=true]{coro}[thm]{Corollaire}
%\newtheorem[M,leftmargin=0.1865cm,rightmargin=0.1865cm,cut=true]{lem}[thm]{Lemme}
%\newtheorem[M,leftmargin=0.1865cm,rightmargin=0.1865cm,cut=true]{defi}[theorem]{Définition}
%%\renewcommand{\thedefinition}{\empty{}}
%\newtheorem[S,leftmargin=0.1865cm,rightmargin=0.1865cm,cut=true]{exem}{Exemple}
%\renewcommand{\theexem}{\empty{}} 
%\newtheorem[S,leftmargin=0.1865cm,rightmargin=0.1865cm,cut=true]{exo}{Exercice}
%\renewcommand{\theexo}{\empty{}}
%\newtheorem[S,leftmargin=0.1865cm,rightmargin=0.1865cm,cut=true]{sol}{Solution}
%\renewcommand{\thesol}{\empty{}} 
%\newtheorem[S,leftmargin=0.1865cm,rightmargin=0.1865cm,cut=true]{hypo}{Hypothesis}[section]
%\newtheorem[S,leftmargin=0.1865cm,rightmargin=0.1865cm,cut=true]{remark}[thm]{Remarque}%[section]
%\newtheorem{dem}{Démonstration}
%\renewcommand{\thedem}{\empty{}} 
%\newtheorem[S]{nota}{Notation}[section]

\theoremstyle{plain}
\newtheorem{theorem}{Théorème}[subsection]
%\renewcommand{\thetheorem}{\empty{}} 
\newtheorem{propo}[theorem]{Proposition}
%\renewcommand{\thepropo}{\empty{}} 
\newtheorem{prop}[theorem]{Propriété}
%\renewcommand{\theprop}{\empty{}} 
\newtheorem{coro}[theorem]{Corollaire}
\newtheorem{lemma}[theorem]{Lemme}

\theoremstyle{definition}
\newtheorem{definition}[theorem]{Définition}
%\renewcommand{\thedefinition}{\empty{}}
\newtheorem{exem}{Exemple}
\renewcommand{\theexem}{\empty{}} 
\newtheorem{cours}{Question de cours}
\renewcommand{\thecours}{\empty{}} 
\newtheorem{exo}{Exercice}
%\renewcommand{\theexo}{\empty{}} 
\newtheorem{sol}{Solution de l'exercice}
%\renewcommand{\thesol}{\empty{}} 
\newtheorem{hypo}{Hypothesis}[section]
\newtheorem{remark}[theorem]{Remarque}%[section]


\theoremstyle{remark}
\newtheorem{dem}{Démonstration}
\renewcommand{\thedem}{\empty{}} 
\newtheorem{nota}{Notation}[section]

%%%%%%%%%%%%%%%%%%%%%%%%%%%%%%%%%%%   Raccourcis   %%%%%%%%%%%%%%%%%%%%%%%%%%%%%%%%%%%
 
% Ensembles
\newcommand{\R}{\mathbb{R}} 
\newcommand{\Q}{\mathbb{Q}}
\newcommand{\C}{\mathbb{C}}
\newcommand{\Z}{\mathbb{Z}}
\newcommand{\K}{\mathbb{K}}
\newcommand{\N}{\mathbb{N}}
\newcommand{\D}{\mathbb{D}}


% Lettres calligraphiques
\newcommand{\calP}{\mathcal{P}} 
\newcommand{\calF}{\mathcal{F}} 
\newcommand{\calD}{\mathcal{D}} 
\newcommand{\calQ}{\mathcal{Q}} 
\newcommand{\calT}{\mathcal{T}} 

% Noyau et image
\newcommand{\im}{\text{Im}} 
\newcommand{\noy}{\text{Ker}} 
\newcommand{\card}{\text{Card}} 

%Epsilon
\newcommand{\vare}{\varepsilon} 


% Fonctions usuelles
%\newcommand{\un}{\mathbb{1}} % Indicatrice en utilisant le package \usepackage{bbold} mais celui modifie les mathbb donc on définit l'indicatrice à la main
\def\un{{\mathchoice {\rm 1\mskip-4mu l} {\rm 1\mskip-4mu l}
{\rm 1\mskip-4.5mu l} {\rm 1\mskip-5mu l}}} % Indicatrice


\title{Corrigé - Colle 9 (Sujet 2)}
\author{MPSI2\\
Année 2021-2022}
%\author{Mathématiques P.A.S.S. 1}
\date{30 novembre 2021}

\begin{document}


   \maketitle
%      \rule{\linewidth}{0.5mm}
      \rule{\linewidth}{0.5mm}
%  \begin{center}
%  %    \rule{\linewidth}{0.5mm}\\[0.4cm]
%       { \huge \bfseries Mathématiques pour le P.A.S.S 1\\[0.4cm] }
%    \rule{\linewidth}{0.5mm}\\[4cm]
%     \end{center}


\begin{cours}
Montrer que $\Q$ et $\R \setminus \Q$ sont dense dans $\R$.
\end{cours}

\begin{exo}
%Difficulté : 2/5
%Chapitre : Applications
\begin{enumerate}
\item Représenter graphiquement la fonction suivante sur son domaine de définition :
\[ g : x \mapsto   \left\{
    \begin{array}{ll}
        1 + \frac{1}{x} & \mbox{si } x > 0 \\
        \sqrt{-x} & \mbox{si } x \leqslant 0
    \end{array}
\right..\]
\item Répondre aux questions suivantes en lisant sur les graphes concernées :
\begin{enumerate}
\item Donner $g(]0,2])$ et $g([1,+\infty[)$.
\item $g : \R \to [0,+\infty[$ est-elle surjective ? injective ? Déterminer deux intervalles $I \subset \R$ et $J \subset [0,+\infty[$ tels que $g : I \to J$ soit bijective.
\end{enumerate}
\end{enumerate}
\end{exo}

\begin{sol}
\begin{enumerate}
\item On a
\begin{center}
    \begin{tikzpicture}[scale=0.4]
     \draw [very thick, ->] (-5,0)--(6,0);
     \draw [very thick, ->] (0,-1)--(0,10);
     \draw [help lines] (-5,-1) grid (6,10);
     \draw [smooth,very thick,bleu!150, samples=100,domain=0.11:6]plot({\x},{1+1/(\x)});
     \draw [smooth,very thick,bleu!150, samples=100,domain=-5:0]plot({\x},{sqrt(-\x)});
               \draw (0.5,-1.5) node {\textbf{Graphe de $g$}};
    \end{tikzpicture}
\end{center}
\item 
\begin{enumerate}
\item $g(]0,2]) = [2,+\infty[$ et $g([1,+\infty[)=]1,2]$.
\item $g$ est surjective de $\R$ dans $[0,+\infty[$. En effet, si $y \in [0,+\infty[$ alors $g(-y^2) = \sqrt{-(-y^2)} = \sqrt{y^2} =y$ car $y \geqslant 0$. En revanche, $g$ n'est pas injective, par exemple car $g(-4) = 2 = g(1)$. $g$ est, par exemple, bijective de $]0,+\infty[$ dans $]1,+\infty[$.
\end{enumerate}
\end{enumerate}
\end{sol}



\begin{exo}
%Difficulté : 1/5
%Chapitre : Applications
Soient $E$ et $F$ deux ensembles et soit $f:E \to F$. Soient également $A$ et $B$ deux parties de $F$.
\begin{enumerate}
\item Démontrer que $A \subset B \Rightarrow f^{-1}(A) \subset f^{-1}(B)$. La réciproque est-elle vraie?
\item Démontrer que $f^{-1}(A \cap B)=f^{-1}(A) \cap f^{-1}(B)$.
\item Démontrer que $f^{-1}(A \cup B)=f^{-1}(A) \cup f^{-1}(B)$.
\end{enumerate}
\end{exo}

\begin{sol}
\begin{enumerate}
\item Prenons $x\in f^{-1}(A)$. Alors $f(x) \in A$ et donc $f(x) \in B$, et donc $x \in f^{-1}(B)$, ce qui prouve l'inclusion $f^{-1}(A) \subset f^{-1}(B)$. La réciproque n'est pas toujours vraie. Prenons $E=\{1\}$, $F=\{1,2\}$, $f:E \to F$ définie par $f(1)=1$, $A=\{2\}$ et $B=\{1\}$. Alors $f^{-1}(A)= \emptyset \subset f^{-1}(B)=\{1\}$. Et pourtant, $A$ n'est pas inclus dans $B$.
\item On a $A \cap B\subset A$, et donc $f^{-1}(A \cap B) \subset f^{-1}(A)$. De même, $f^{-1}(A \cap B)\subset f^{-1}(B)$, et donc $f^{-1}(A\cap B) \subset f^{-1}(A) \cap f^{-1}(B)$.\\
Réciproquement, soit $x\in f^{-1}(A) \cap f^{-1}(B)$. Alors on a à la fois $f(x) \in A$ et $f(x) \in B$, ce qui entraîne $f(x) \in A \cap B$. Ainsi, $x \in f^{-1}(A\cap B)$, et donc $f^{-1}(A) \cap f^{-1}(B) \subset f^{-1}(A\cap B)$.
\item On a $A \subset A \cup B$ et donc $f^{-1}(A) \subset f^{-1}(A \cup B)$. De même, $f^{-1}(B) \subset f^{-1}(A \cup B)$ et donc $f^{-1}(A) \cup f^{-1}(B) \subset f^{-1}(A \cup B)$.\\
Réciproquement, si $x \in f^{-1}(A \cup B)$, alors $f(x) \in A \cup B$. Ainsi, ou bien $f(x) \in A$, ou bien $f(x) \in B$. Mais si $f(x) \in A$, on a $x \in f^{-1}(A) \subset f^{-1}(A) \cup f^{-1}(B)$ et de même, si $f(x) \in B$, on a $x \in f^{-1}(B) \subset f^{-1}(A) \cup f^{-1}(B)$. Dans tous les cas, on a prouvé que $x\in f^{-1}(A) \cup f^{-1}(B)$ et donc l'inclusion $f^{-1}(A \cup B) \subset f^{-1}(A) \cup f^{-1}(B)$. 
\end{enumerate}
\end{sol}


\begin{exo}
%Difficulté : 3/5
%Chapitre : Nombres réels
On considère $\displaystyle{ A = \left\{ \frac{n}{mn +1}, \ (m,n) \in \N^{\star} \right\}}$. $A$ est-elle majorée, minorée ? Déterminer, s'il y a lieu, la borne inférieure et la borne supérieure et dire s'il s'agit d'un minimum ou d'un maximum.
\end{exo}

\begin{sol}
On a, pour tout $n,m \in \N^{\star}$,
\[0 \leqslant \frac{n}{nm+1} \leqslant \frac{n}{nm} \leqslant \frac{1}{m} \leqslant 1.\]
$A$ est donc minorée par 0 et majorée par 1. Montrons que $\inf(A)=0$. Si $c>0$ est un minorant de $A$, alors pour tout couple $(m,n) \in (\N^{\star})2$, on a 
\[ c \leqslant \frac{n}{nm+1}.\]
Prenons $n=1$, on obtient 
\[ c \leqslant \frac{1}{m+1} \quad \Leftrightarrow \quad m \leqslant \frac{1}{c} - 1.\]
Comme ceci doit être vrai pour tout entier $m\geqslant 1$, c'est une contradiction car $\N$ n'est pas majoré. Ainsi, 0 est le plus grand des minorants de $A$, et $\inf(A)=0$. Démontrons de même que $1=\sup(A)$. Si $d<1$ est un majorant de $A$, alors pour tout couple $(m,n) \in (\N^{\star})^2$, on a \[ d \geqslant \frac{n}{nm+1}.\]
Pour $m=1$, on obtient 
\[ d \geqslant \frac{n}{n+1} \quad \Leftrightarrow \quad d \geqslant n(1-d)  \quad \Leftrightarrow \quad  n \leqslant \frac{d}{1-d}.\]
Cette inégalité est impossible à réaliser pour tout entier $n$, et donc $\sup(A)=1$. De plus, 0 n'est pas un élément de $A$ (c'est trivial), et 1 non plus car on a toujours $nm+1>n$ pour $n,m\geqslant 1$.
\end{sol}




\begin{exo}
%Difficulté : 2/5
%Chapitre : Applications
On considère
\[ \begin{array}{ccccc}
f & : & \R & \to & \C \\
 & & x & \mapsto & \frac{1+ix}{1-ix}
\end{array} .\]
\begin{enumerate}
 \item Montrer que $f$ est bien définie. $f$ est-elle injective ? $f$ est-elle surjective ?
 \item Déterminer $f^{-1}(\mathbb R)$.
 \item Montrer que $f(\mathbb R)=\{z\in\mathbb C,\ |z|=1,\ z\neq-1\}$.
\end{enumerate}
\end{exo}

\begin{sol}
\begin{enumerate}
 \item $f$ est bien définie car le dénominateur ne s'annule pas. En effet, puisque $x \in \R$, $ix \in i \R$ et donc $1 - ix \neq 0$.\\
 Etudions l'injectivité de $f$. Soit $(x,y) \in \R^2$ tels que $f(x) = f(y)$. Alors
 \[ \frac{1+ix}{1-ix} = \frac{1+iy}{1-iy} \quad \Leftrightarrow \quad (1+ix)(1-iy) = (1-ix)(1+iy) \quad \Leftrightarrow \quad 1+ix - iy +xy = 1-ix + iy + xy.\]
 On a donc $2i(x-y) = 0$ ce qui impose que $x=y$. $f$ est donc injective.\\
 Etudions à présent la surjectivité de $f$. Supposons qu'il existe $x \in \R$ tel que $f(x) = 0$ alors $1+ix = 0$ ce qui impose que $x=i$ ce qui est absurde puisque $x$ est réel. $f$ n'est donc pas surjective.
 \item On chercher l'ensemble des éléments $x \in \R$ tels que $f(x) \in \R$. Or,
 \[ f(x) = a \in \R \quad \Leftrightarrow \quad 1+ix = a(1-ix) \quad \Leftrightarrow \quad 1+ ix = a - iax.\]
 En identifiant les parties réelles et imaginaire on arrive à $a=1$ et $ax = -x$ ce qui peut être satisfait seulement si $x = 0$ (puisque $a = 1$). On peut alors vérifier que $f(0) = 1 \in \R$. Ainsi, $f^{-1}(\R) = \{0 \}$.
 \item Procédons par double-inclusion.
 \begin{itemize}
 \item \textbf{Sens direct.} Soit $z \in f(\R)$ alors il existe $x \in\R$ tel que $z = f(x)$. Autrement dit, $z = \frac{1+ix}{1-ix}$. Montrons à présent que $|z| = 1$ et que $z \neq -1$. On a,
 \[ |z| = \left| \frac{1+ix}{1-ix} \right| = \frac{\sqrt{1+x^2}}{\sqrt{1+x^2}} = 1.\]
 De plus, supposons qu'il existe $x \in \R$ tel que $f(x) = -1$. Alors,
 \[ f(x) = -1 \quad \Leftrightarrow \quad  \frac{1+ix}{1-ix} = -1 \quad \Leftrightarrow \quad  1+ix = -1 + ix \quad \Leftrightarrow \quad  2 =0\]
 ce qui est bien entendu absurde. Donc, pour tout $x \in \R$, $f(x) \neq -1$. On a donc montré l'inclusion directe.
 \item \textbf{Sens réciproque.} Soit $z \in \C$ tel que $|z| =1$ et $z \neq -1$. Pour montrer que $z \in f(\R)$ nous devons trouver $x \in \R$ tel que $f(x) = z$. Or,
 \[ f(x) = z \quad \Leftrightarrow \quad  \frac{1+ix}{1-ix} = z \quad \Leftrightarrow \quad  1+ix = (1 - ix)z \quad \Leftrightarrow \quad  x(i +iz) = -1 + z.\]
 Puisque $z \neq -1$, on obtient
 \[ x = \frac{-1+z}{i(1+z)}.\]
 Donc, $z \in f(\R)$.
 \end{itemize}
 On a donc bien montré que $f(\mathbb R)=\{z\in\mathbb C,\ |z|=1,\ z\neq-1\}$ par double-inclusion.
\end{enumerate}
\end{sol}


\begin{exo}
%Difficulté : 3/5
%Chapitre : Nombres réels
Soit $f:[0,1] \to [0,1]$ une application croissante. On note $E=\{x \in [0,1]; \ f(x)\geqslant x\}$.
\begin{enumerate}
\item Montrer que $E$ admet une borne supérieure $b$.
\item Prouver que $f(b)=b$.
\end{enumerate}
\end{exo}

\begin{sol}
\begin{enumerate}
\item $E$ est une partie non-vide (car elle contient 0), et majorée (par 1) de $\R$. Elle admet donc une borne supérieure que l'on note $b$. 
\item On va raisonner par l'absurde pour démontrer que $f(b)=b$.
\begin{itemize}
\item \textbf{Si $f(b)<b$.} Comme $b$ est le plus petit des majorants de $E$, $f(b)$ ne majore pas $E$. Il existe donc un élément $c$ de $E$ tel que $f(b) < c \leqslant b$. Mais alors $f(c)\geqslant c > f(b)$ alors que $c \leqslant b$. Ceci contredit que $f$ est croissante.
\item \textbf{Si $f(b)>b$.} Comme $f$ est croissante, on a $f(f(b)) \leqslant f(b)$, et donc $f(b) \in E$, ce qui est impossible puisque $f(b)$ est strictement supérieur à la borne supérieure de $E$.
\end{itemize}
\end{enumerate}
\end{sol}












\end{document}
