On commence par remarquer que cette égalité n'est bien définie que pour $x > 1$ car $\ln$ n'est bien définie que sur $\R_+^{\star}$. De plus,
	\[		2 \ln x +\ln(2x-1) = \ln(2x+8) + 2\ln(x-1) \quad \Leftrightarrow  \quad \ln(x^2) +\ln(2x-1) = \ln(2x+8) + \ln((x-1)^2)\]
	et, toujours en utilisant les propriétés de la fonction $\ln$,
	\[ 2 \ln x +\ln(2x-1) = \ln(2x+8) + 2\ln(x-1)  \quad \Leftrightarrow  \quad \ln(x^2(2x-1)) = \ln((2x+8)(x-1)^2).\]
	Ainsi, après calcul on arrive à l'inéquation
	\[ \ln(2x^3-x^2) = \ln(2x^3 + 4x^2  - 14x + 8).\]
	On applique alors la fonction exponentielle et on obtient
		\[ 2x^3-x^2 = 2x^3 + 4x^2  - 14x + 8 \quad \Leftrightarrow  \quad 5x^2 - 14x + 8 = 0.\]
Or, le discriminant de ce trinôme est donné par $\Delta = 196 - 160 = 36$ et les racines de ce trinôme sont donc
\[ x_1 = \frac{14 - \sqrt{36}}{10} = \frac{8}{10} = \frac{4}{5} \quad \Leftrightarrow \quad 
x_2 = \frac{14 + \sqrt{36}}{10} = \frac{20}{10} = 2.\]
Or, $x_1 \leqslant 1$ donc la seule solution de l'équation étudiée est $x = 2$.