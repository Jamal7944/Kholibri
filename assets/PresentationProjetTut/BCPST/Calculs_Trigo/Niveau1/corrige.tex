On procède par disjonction de cas pour se débarasser des valeurs absolues. Ainsi,
\begin{itemize}
\item \textbf{Si $x < -1$ :} Alors $|x-2| = -(x-2)$ et $|x+1| = -(x+1)$ et donc 
\[|x-2| - 2 |x+1| = 0  \quad \Leftrightarrow \quad
-x+2 + 2 (x+1) = 0 \quad \Leftrightarrow \quad
x = -4\]
et $-4$ vérifie bien la condition $x < -1$.
\item \textbf{Si $-1 \leqslant x \leqslant 2$ :} Alors $|x-2| = -(x-2)$ et $|x+1| = x+1$ et donc 
\[|x-2| - 2 |x+1| = 0  \quad \Leftrightarrow \quad
-x+2 - 2 (x+1) = 0 \quad \Leftrightarrow \quad
-3x = 0 \quad \Leftrightarrow \quad
x = 0 \]
et $0$ vérifie bien la condition $-1 \leqslant x \leqslant 2$.
\item \textbf{Si $x > 2$ :} Alors $|x-2| = x-2$ et $|x+1| = x+1$ et donc 
\[|x-2| - 2 |x+1| = 0  \quad \Leftrightarrow \quad
x-2 - 2 (x+1) = 0 \quad \Leftrightarrow \quad
-x -4 = 0 \quad \Leftrightarrow \quad
x = -4 \]
mais $-4$ ne satisfait pas la condition $x > 2$. Nous ne tenons donc pas compte de cette solution.
\end{itemize}
Finalement, nous avons montré que les solutions de l'équation $|x-2| - 2 |x+1| = 0$ sont $-4$ et $0$.