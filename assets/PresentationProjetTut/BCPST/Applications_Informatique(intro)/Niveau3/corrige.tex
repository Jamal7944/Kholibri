Pour $x>0$, posons 
\[ f(x)=x^{ 1/x} = e^{\frac{\ln(x)}{x}}\]
de sorte que $\sqrt[n]{n}=f(n)$. $f$ est dérivable sur l'intervalle $]0,+\infty[$ et on a 
\[ f'(x)=\frac{1-\ln(x)}{x^2}e^{\frac{\ln(x)}{x}}.\]
Pour $x>0$, $f'(x)$ est du signe de $1-\ln(x)$, donc $f'(x)>0$ si $x \in ]0,e[$ et $f'(x)<0$ si $x\in ]e,+\infty[$. Puisque $3>e$, on en déduit que la fonction $f$ est strictement décroissante sur $[3,+\infty[$. En particulier, pour $n \geqslant 3$, on a $f(n) \geqslant f(3)$, et donc la plus grande valeur de $\sqrt[n]{n}$ est atteinte pour $n=2$ ou pour $n=3$. Comme $\sqrt{2} \simeq 1,41$ et $\sqrt[3]{3} \simeq 1,44$ la valeur maximale vaut $\sqrt[3]{3}$.