\begin{enumerate}
      \item On rappelle que
      \[ \begin{pmatrix} k \\ p \end{pmatrix}  =   \frac{k!}{p! (k-p)!}.\]
      Ainsi,
      \[ \begin{pmatrix} k+1 \\ p+1 \end{pmatrix} - \begin{pmatrix} k \\ p+1 \end{pmatrix} = \frac{(k+1)!}{(p+1)! (k+1-(p+1))!} - \frac{k!}{(p+1)! (k-(p+1))!}\]
      et donc
      \[ \begin{pmatrix} k+1 \\ p+1 \end{pmatrix} - \begin{pmatrix} k \\ p+1 \end{pmatrix}
      = \frac{(k+1)!}{(p+1)! (k-p)!} - \frac{k!}{(p+1)! (k-p-1)!}.\]
      On met donc sur le même dénominateur et on obtient
      \[ \begin{pmatrix} k+1 \\ p+1 \end{pmatrix} - \begin{pmatrix} k \\ p+1 \end{pmatrix}
      = \frac{(k+1)! - k! (k-p)}{(p+1)! (k-p)!}  = \frac{k! (k+1  - (k-p))}{(p+1)! (k-p)!}   = \frac{k! (p+1)}{(p+1)! (k-p)!}.\]
      On arrive donc bien à
      \[ \begin{pmatrix} k+1 \\ p+1 \end{pmatrix} - \begin{pmatrix} k \\ p+1 \end{pmatrix} = \begin{pmatrix} k \\ p \end{pmatrix}.\]
      \item On note que
      \[ \sum_{k=p}^n \begin{pmatrix} k \\ p \end{pmatrix} = \sum_{k=p}^n \left( \begin{pmatrix} k+1 \\ p+1 \end{pmatrix} - \begin{pmatrix} k \\ p+1 \end{pmatrix} \right)\]
      Autrement dit,
     
\begin{align*}
\underset{k=p}{\overset{n}{\sum}} \begin{pmatrix} k \\ p \end{pmatrix}  &= 1 + \underset{k=p+1}{\overset{n}{\sum}} \begin{pmatrix} k \\ p \end{pmatrix} \\
  & =  1+ \left( \begin{pmatrix} p+2 \\ p+1 \end{pmatrix} - \begin{pmatrix} p+1\\ p+1 \end{pmatrix} \right) + \left( \begin{pmatrix} p+3 \\ p+1 \end{pmatrix} - \begin{pmatrix} p+2\\ p+1 \end{pmatrix} \right)\\ 
 &  \quad + \dots  \\ 
 &\quad + \left( \begin{pmatrix} n \\ p+1 \end{pmatrix} - \begin{pmatrix} n-1 \\ p+1 \end{pmatrix} \right) + \left( \begin{pmatrix} n+1 \\ p+1 \end{pmatrix} - \begin{pmatrix} n\\ p+1 \end{pmatrix}\right) \\
&= 1 + \begin{pmatrix} n+1 \\ p+1 \end{pmatrix} - \begin{pmatrix} p+1\\ p+1 \end{pmatrix} \\
&= \begin{pmatrix} n+1 \\ p+1 \end{pmatrix}.
\end{align*}

     \end{enumerate}