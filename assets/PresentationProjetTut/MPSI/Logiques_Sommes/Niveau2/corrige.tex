	Soit $x\in \R$ et $n\in \N^*$.
	\begin{enumerate}
		\item Si $x=1$, alors $S_n = n$. Si $x\ne 1$, la somme $S_n$ est la somme des $n$ premier termes d'une suite géométrique d'où 
		\[
			S_n(x) = \frac{1-x^{n+1}}{1-x} \ .
		\]
		\item Pour tout $n$ et pour tout $x$, $S_n$ est une fonction dérivable (c'est un polynôme) donc, pour tout $n\geqslant 1$ et pour tout réel $x\ne 1$
		\[
			S'_n(x) = \sum_{k=1}^n kx^{k-1} = \sum_{k=1}^n (k-1)x^{k-1} + \sum_{k=1}^n x^{k-1}
		\]
	\end{enumerate}