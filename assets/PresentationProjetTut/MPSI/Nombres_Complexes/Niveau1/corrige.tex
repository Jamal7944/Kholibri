On commence par écrire $1+i\sqrt{3}$ sous forme exponentielle : $1+i\sqrt{3}=2e^{\frac{i \pi }{3}}$. En prenant la puissance $n$-ième, on obtient $(1+i\sqrt{3})^n =2^n e^{\frac{in\pi}{3}}$. Ceci est un réel positif si et seulement si $\sin\left( \frac{n \pi}{3} \right) =0$ et $\cos \left( \frac{n \pi}{3} \right) \geqslant 0$. Or, $\sin\left( \frac{n \pi}{3} \right)=0$ si et seulement si $n=3k$, $k \in \Z$. Mais, pour ces valeurs de $n$, on a $\cos \left( \frac{n \pi}{3} \right)=\cos(k\pi)$, et ceci est positif si et seulement si $k$ est pair. Ainsi, les entiers qui conviennent sont les multiples de 6.